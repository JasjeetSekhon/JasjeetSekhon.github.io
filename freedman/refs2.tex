% checked 5/17 by DC and JM

%refsall.tex refs for vol 1 & 3  25 pp before theboil down

\input mtplainx

\magnification1050
\hsize=290pt
\vsize=500pt
\hoffset72pt
\nopagenumbers
\pageno=343


\font\num=frw at 20pt
\font\tit=frw at 14pt
\font\au=frw at 12pt
\font\hel=frw at 10pt
\font\hi=frwi at 10pt
\font\sc tirsc at 10pt

\frenchspacing
\font\sm=tir at 9.5pt\sm
\font\it=tii at 9.5pt

\vglue160pt

\noindent
{\tit References}
\bigskip\bigskip\medskip

%{\parskip0pt
\noindent
Aalen, O.~O. (1978).
Nonparametric inference for a family of counting processes.
{\it Annals of Statistics\/} 6: 701--26.

\smallskip\noindent
Abbott, A. (1997).
Of time and space: The contemporary relevance of the Chicago school.
{\it Social Forces\/} 75: 1149--82.

\smallskip\noindent
Abbott, A. (1998).
The causal devolution.
{\it Sociological Methods and Research\/} 27: 148--81.

\smallskip\noindent
Achen, C.~H. and Shively, W.~P. (1995).
{\it Cross-Level Inference\/}.
Chicago: University of Chicago Press.

\smallskip\noindent
Alderman, M.~H., Madhavan, S., Cohen, H.~et al.~(1995).
Low urinary sodium is associated with greater risk of
myocardial infarction among treated hypertensive men.
{\it Hypertension\/} 25: 1144--52.

\smallskip\noindent
Alderman, M.~H., Madhavan, S., Ooi, W.~L.~et al.~(1991).
Association of the renin-sodium profile with the risk of
myocardial infarction in patients with hypertension.
{\it New England Journal of Medicine\/} 324: 1098--1104.

\smallskip\noindent
Altman, D.~G. and de Stavola, B.~L. (1994).
Practical problems in fitting a proportional hazards model to
data with updated measurements of the covariates.
{\it Statistics in Medicine\/} 13: 301--41.

\smallskip\noindent
Altman, D.~G., Schulz, K.~F., Moher, D. et al.~(2001).
The revised CONSORT statement for reporting randomized trials:
Explanation and elaboration.
{\it Annals of Internal Medicine\/} 134: 663--94.

\smallskip\noindent
Amemiya, T. (1981).
Qualitative response models: A survey.
{\it Journal of Economic Literature\/} {19}: 1483--1536.

\smallskip\noindent
Amemiya, T. (1985).
{\it Advanced Econometrics\/}.
Cambridge, MA: Harvard University Press.
%\par}
%\vfil\eject

\smallskip\noindent
Andersen, P.~K. (1991).
Survival analysis 1982--1991: The second decade of the
proportional hazards regression model.
{\it Statistics in Medicine\/} 10: 1931--41.

\smallskip\noindent
Andersen, P.~K., Borgan, \O., Gill, R.~D., and Keiding, N. (1996).
{\it Statistical Models Based on Counting Processes\/}.
Corr.~4th printing. New York: Springer-Verlag.

\headline{\ifodd\pageno\rightheadline \else\leftheadline\fi}
\def\rightheadline{\sc References \hfill \folio}
\def\leftheadline{\sc\folio \hfill David A.~Freedman}
\voffset=2\baselineskip


\smallskip\noindent
Andersen, P.~K. and Keiding, N., eds. (2006).
{\it Survival and Event History Analysis\/}.
Chichester, U.K.: Wiley.



\smallskip\noindent
Anderson, M., Ensher, J.~R., Matthews, M.~R., Wieman, C.~E., and Cornell, E.~A. (1995).
Observation of Bose-Einstein condensation in a dilute atomic vapor.
{\it Science\/} {269}: 198--201.

\smallskip\noindent
Anderson, M. and Fienberg, S.~E. (1999).
{\it Who Counts? The Politics of Census-Taking in Contemporary America\/}.
New York: Russell Sage Foundation.

\smallskip\noindent
Angrist, J.~D. (2001).
Estimation of limited dependent variable models with binary
endogenous regressors: Simple strategies for empirical practice.
{\it Journal of Business and Economic Statistics\/} 19: 2--16.

\smallskip\noindent
Angrist, J.~D, Imbens, G.~W., and Rubin, D.~B. (1996).
Identification of causal effects using instrumental variables.
{\it Journal of the American Statistical Association\/} 91: 444--72.

\smallskip\noindent
Angrist, J.~D. and Krueger, A.~B. (2001).
Instrumental variables and the search for identification:
From supply and demand to natural experiments.
{\it Journal of Economic Persepctives\/} 15: 69--85.

\smallskip\noindent
Appel, L.~J., Moore, T.~J., Obarzanek, E.~et al.~(1997).
A clinical trial of the effects of dietary patterns on blood pressure.
DASH Collaborative Research Group.
{\it New England Journal of Medicine\/} 336: 1117--24.

\smallskip\noindent
Arceneaux, K., Gerber, A.~S., and Green, D.~P. (2006).
Comparing experimental and matching methods using a large-scale voter mobilization experiment.
{\it Political Analysis\/} 14: 37--62.

\smallskip\noindent
Archer J. (2000).
Sex differences in aggression between heterosexual partners: A meta-analytic review.
{\it Psychological Bulletin\/} 126: 651--80.

\smallskip\noindent
Aris, E.~M.~D., Hagenaars, J.~A.~P., Croon, M., and Vermunt, J.~K. (2000).
The use of randomization for logit and logistic models.
In J. Blasius, J. Hox, E. de Leuw, and P. Smidt, eds.
{\it Proceedings of the Fifth International Conference on Social Science Methodology\/}.
Cologne: TT Publications.

\smallskip\noindent
Arminger, G. and Bohrnstedt, G.~W. (1987).
Making it count even more: A review and critique of Stanley Lieberson's
{\it Making It Count: The Improvement of Social Theory and Research\/}.
In C. Clogg, ed. {\it Sociological Methdology 1987\/}.
Washington, D.C.: American Sociological Association, pp. 198--201.

\smallskip\noindent
Bahry, D. and Silver, B.~D. (1987).
Intimidation and the symbolic uses of terror in the USSR.
{\it American Political Science Review\/} 81: 1065--98.

\smallskip\noindent
Bailar, J.~C. (1997).
The promise and problems of meta-analysis.
{\it New England Journal of Medicine\/} 337: 559--61.

\smallskip\noindent
Bailar, J.~C. (1999).
Passive smoking, coronary heart disease, and meta-analysis.
{\it New England Journal of Medicine\/} 340: 958--59.

\smallskip\noindent
Bang, H. and Robins, J.~M. (2005).
Doubly robust estimation in missing data and causal inference models.
{\it Biometrics\/} 61: 962--72.

\smallskip\noindent
Baron, J. (1838).
{\it The Life of Edward Jenner\/}. Vol. I.
London: Henry Colburn.

\smallskip\noindent
Barrett-Connor, E. (1991).
Postmenopausal estrogen and prevention bias.
{\it Annals of Internal Medicine\/} 115: 455--56.

\smallskip\noindent
Bayes, T. (1764).
An essay towards solving a problem in the doctrine of chances.
{\it Philosophical Transactions of the Royal Society of London\/} 53: 370--418.

\smallskip\noindent
Bayles, K.~W. (2000).
The bactericidal action of penicillin: New clues to an unsolved mystery.
{\it Trends in Microbiology\/} 8: 274--78.

\smallskip\noindent
Bazin, H. (2000).
{\it The Eradication of Smallpox\/}.
London: Academic Press.

\smallskip\noindent
Beck, N., Katz,  J.~N., Alvarez, R.~M., Garrett, G., and Lange, P. (1993).
Government partisanship, labor organization, and macroeconomic performance.
{\it American Political Science Review\/} 87: 945--48.

\smallskip \noindent
Belsley, D.~A., Kuh, E., and Welsch, R.~E. (2004).
{\it Regression Diagnostics: Identifying Influential Data and Sources of Collinearity\/}.
New York: Wiley.

\smallskip\noindent
Berger, J. (1985).
{\it Statistical Decision Theory and Bayesian Analysis\/}. 2nd edn.
New York: Springer-Verlag.

\smallskip\noindent
Berger, J. and Wolpert, R. (1988).
{\it The Likelihood Principle\/}. 2nd edn.
Hayward, CA: Institute of Mathematical Statistics.

\smallskip\noindent
Berk, R.~A. (1988).
Causal inference for statistical data.
In N.~J. Smelser, ed. {\it Handbook of Sociology}.
Beverly Hills: Sage Publications, pp. 155--72.

\smallskip\noindent
Berk, R.~A. (1991).
Toward a methodology for mere mortals.
In P.~V. Marsden, ed. {\it Sociological Methodology 1991\/}.
Washington, D. C.: The American Sociological Association, pp. 315--24.

\smallskip\noindent
Berk, R.~A. (2004).
{\it Regression Analysis: A Constructive Critique\/}.
Thousand Oaks, CA: Sage Publications.

\smallskip\noindent
Berk, R.~A. and Campbell, A. (1993).
Preliminary data on race and crack charging practices in Los Angeles.
{\it Federal Sentencing Reporter\/} 6(1): 36--38.

\smallskip\noindent
Berk, R.~A. and Freedman, D.~A. (2003).
Statistical assumptions as empirical commitments.
In T.~G. Blomberg and S. Cohen, eds.
{\it Law, Punishment, and Social Control: Essays in Honor of Sheldon Messinger\/}. 2nd edn.
New York: Aldine de Gruyter, pp.~235--54.

\smallskip\noindent
Berk, R.~A. and Freedman, D.~A. (2008).
On weighting regressions by propensity scores.
{\it Evaluation Review\/} 32: 392--409.

\smallskip\noindent
Berkson, J. (1944).
Application of the logistic function to bio-assay.
{\it Journal of the American Statistical Association\/} {39}: 357--65.

\smallskip\noindent
Bernoulli, D. (1760).
Essai d'une nouvelle analyse de la mortalit\'e caus\'ee par la petite variole, et des
avantages de l'inoculation pour la pr\'evenir.
{\it M\'emoires de Math\'ematique et de Physique de l'Acad\'emie
Royale des Sciences\/}, Paris, pp. 1--45.
Reprinted in {\it Histoire de l'Acad\'emie Royale des Sciences\/} (1766).

\smallskip\noindent
Bhattacharya, J., Goldman, D., and McCaffrey, D. (2006).
Estimating probit models with self-selected treatments.
{\it Statistics in Medicine\/} 25: 389--413.

\smallskip\noindent
Bickel, P.~J. and Doksum, K.~A. (1977).
{\it  Mathematical Statistics: Basic Ideas and Selected Topics\/}.
San Francisco: Holden-Day.

\smallskip\noindent
Blau, P.~M. and Duncan. O.~D. (1967).
{\it The American Occupational Structure\/}.
New York: Wiley.

\smallskip\noindent
Bluthenthal, R.~N., Ridgeway, G., Schell, T.~et al.~(2006).
Examination of the association between Syringe Exchange Program (SEP) dispensation
policy and SEP client-level syringe coverage among injection drug users.
{\it Addiction\/} 102: 638--46.

\smallskip\noindent
Brady, H.~E. and Collier, D., eds. (2004).
{\it Rethinking Social Inquiry: Diverse Tools, Shared Standards\/}.
Lanham, MD: Rowman \& Littlefield.

\smallskip\noindent
Brady, H.~E., Collier, D., and Seawright, J. (2004).
Refocusing the discussion of methodology.
In Brady and Collier (2004), pp.~3--20.

\smallskip\noindent
Brant, R. (1996).
Digesting logistic regression results.
{\it The American Statistician\/} {50}: 117--19.

\smallskip\noindent
Briggs, D.~C. (2004).
Causal inference and the Heckman model.
{\it Journal of Educational and Behavioral Statistics\/} 29: 397--420.

\smallskip\noindent
Bristowe, J.~S. and Hutchinson, J.~S. (1876).
{\it A Treatise on the Theory and Practice of Medicine\/}.
Philadelphia, PA: Henry C.~Lea.

\smallskip\noindent
Bristowe, J.~S., Wardell, J.~R., Begbie, J.~W.~et al.~(1879).
{\it Diseases of the Intestines and Peritoneum\/}.
New York: William Wood and Company.

\smallskip\noindent
Brown, L.~D., Eaton, M.~L., Freedman, D.~A.~et al.~(1999).
Statistical controversies in Census 2000.
{\it Jurimetrics\/} 39: 347--75.

\smallskip\noindent
Buck, C., Llopis, A., N\'ajera, E., and Terris, M., eds. (1989).
{\it The Challenge of Epidemiology: Issues and Selected Readings\/}.
Geneva: World Health Organization, Scientific Publication No. 505.

\smallskip\noindent
Budd, W. (1873).
{\it Typhoid Fever: Its Nature, Mode of Spreading, and Prevention\/}.
London: Longmans, Green, and Co.
Reprinted in 1977 by Ayer Publishing.

\noindent \hskip 10pt
http://www.deltaomega.org/typhoid.pdf

\smallskip\noindent
Bulloch, W. (1938).
{\it The History of Bacteriology\/}.
Oxford: Oxford University Press.

\smallskip\noindent
Bushway, S., Johnson, B.~D., and Slocum, L.~A. (2007).
Is the magic still there? The use of the Heckman two-step correction for selection bias in criminology.
{\it Journal of Quantitative Criminology\/} 23: 151--78.

\smallskip\noindent
Carmelli, D. and Page, W.~F. (1996).
24-year mortality in smok\-ing-dis\-cor\-dant World War II
U.S.~male veteran twins.
{\it International Journal of Epidemiology\/} 25: 554--59.

\smallskip\noindent
Carpenter, K.~J. (1981).
{\it Pellagra\/}.
Stroudsberg, PA: Hutchinson Ross.

\smallskip\noindent
Carpenter, K.~J. (2000).
{\it Beriberi, White Rice, and Vitamin B\/}.
Berkeley: University of California Press.

\smallskip\noindent
Cartwright, N. (1989).
{\it Nature's Capacities and Their Measurement\/}.
Oxford: Clarendon Press.

\smallskip\noindent
Casella, G. and Berger, R.~L. (1990).
{\it Statistical Inference\/}.
Pacific Grove, CA: Wadsworth \& Brooks/Cole.

\smallskip\noindent
Centers for Disease Control (1999).
Fluoridation of drinking water to prevent dental caries.
{\it Morbidity and Mortality Weekly Report\/}, October 22, Vol. 48, No. 41, 933--40.
U.S.~Department of Health and Human Services.

\smallskip\noindent
Chattopadhyay, R. and Duflo, E. (2004).
Women as policy makers: Evidence from a randomized policy experiment in India.
{\it Econometrica\/} 72: 1409--43.

\smallskip\noindent
Cho, W.~K.~Tam (1998).
Iff the assumption fits: A comment on the King ecological inference solution.
{\it Political Analysis\/} 7: 143--63.

\smallskip\noindent
Chobanian, A.~V. and Hill, M. (2000).
National Heart, Lung, and Blood Institute Workshop on Sodium and Blood Pressure:
A critical review of current scientific evidence.
{\it Hypertension\/} 35: 858--63.
Quotes are from the online unabridged version (www.nhlbi.nih.gov).

\smallskip\noindent
Cholera Inquiry Committee (1855).
{\it Report on the Cholera Outbreak in the Parish of St. James, Westminster during the Autumn of 1854\/}.
London: Chur\-chill.

\smallskip\noindent
Chrystal, G. (1889).
{\it Algebra: An Elementary Text Book for the Higher Classes of Secondary Schools and for Colleges\/}. Part II.
Edinburgh: Adam and Charles Black.

\smallskip\noindent
Churchill, H.~V. (1931).
Occurrence of fluorides in some waters of the United States.
{\it Journal of Industrial and Engineering Chemistry\/} 23: 996--98.

\smallskip\noindent
Citro, C.~F., Cork, D.~L., and Norwood, J.~L., eds. (2004).
{\it The 2000 Census: Counting Under Adversity\/}.
Washington, D.C.: National Academy Press.

\smallskip \noindent
Clifford, P. (1977).
Nonidentifiability in stochastic models of illness and death.
{\it Proceedings of the National Academy of Sciences\/} {\it USA\/} 74: 1338--40.

\smallskip\noindent
Clogg, C.~C. and Haritou, A. (1997).
The regression method of causal inference and a dilemma confronting this method.
In V. McKim and S. Turner, eds. {\it Causality in Crisis?\/}
Notre Dame, IN: University of Notre Dame Press, pp.~83--112.

\smallskip\noindent
Cochran, W.~G. (1957).
Analysis of covariance: Its nature and uses.
{\it Biometrics} 13: 261--81.

\smallskip\noindent
Cohen, J. (1988).
{\it Statistical Power Analysis for the Behavioral Sciences\/}. 2nd edn.
Hillsdale, NJ: Lawrence Erlbaum.

\smallskip\noindent
Cohen, M.~L., White, A.~A., and Rust, K.~F., eds. (1999).
{\it  Measuring a Changing Nation:  Modern Methods for the 2000 Census\/}.
Washington, D.C.: National Academy Press.

\smallskip\noindent
Colton, T. and Greenberg, E.~R. (1982).
Epidemiologic evidence for adverse effects of DES exposure during pregnancy.
{\it The American Statistician\/} 36: 268--72.

\smallskip\noindent
Cook, N.~R., Cutler, J.~A., Obarzanek, E.~et al.~(2007).
Long term effects of dietary sodium reduction on cardiovascular disease outcomes:
Observational followup of the trials of hypertension prevention.
{\it British Medical Journal\/} 334: 885--92.

\smallskip\noindent
Cook, T.~D. and Campbell, D.~T. (1979).
{\it Quasi-Experimentation: Design \& Analysis Issues for Field Settings\/}.
Boston, MA: Houghton Mifflin.

\smallskip\noindent
Copas, J.~B. and Li, H.~G. (1997).
Inference for non-random samples.
{\it Journal of the Royal Statistical Society\/}, Series B, 59: 55--77.

\smallskip\noindent
Cork, D.~L., Cohen, M.~L., and King, B.~F., eds. (2004).
{\it Reengineering the 2010 Census: Risks and Challenges\/}.
Washington, D.C.: National Academy Press.

\smallskip\noindent
Cornfield, J., Haenszel, W., Hammond, E.~C., Lilienfeld, A.~M., Shimkin, M.~B., and Wynder, E.~L. (1959).
Smoking and lung cancer: Recent evidence and a discussion of some questions.
{\it Journal of the National Cancer Institute\/} 22: 173--203.

\smallskip\noindent
Cournot, A.~A. (1843).
{\it Exposition de la th\'eorie des chances et des probabilit\'es\/}.
Paris: Hachette.
Reprinted in Bernard Bru, ed. (1984). {\it Oeuvres com\-pl\`etes de Cournot\/}.
Paris: J. Vrin, vol. 1.

\smallskip\noindent
Cox, D.~R. (1956).
A note on weighted randomization.
{\it Annals of Mathematical Statistics\/} 27: 1144--51.

\smallskip\noindent
Cox, D.~R. (1972).
Regression models and lifetables.
{\it Journal of the Royal Statistical Society\/}, Series~B, 34: 187--220 (with discussion).

\smallskip\noindent
Crump, R.~K., Hotz, V.~J., Imbens, G.~W., Mitnik, O.~A. (2007).
Dealing with limited overlap in estimation of average treatment effects.
Technical report, Department of Economics, University of California, Berkeley.
%biometrika, to appear?

\smallskip\noindent
Cutler, J.~A., Follmann, D., and Allender, P.~S. (1997).
Randomized trials of sodium reduction: An overview.
{\it American Journal of Clinical Nutrition\/} 65, Supplement: S643--51.

\smallskip\noindent
Dabrowska, D. and Speed, T.~P. (1990).
On the application of probability theory to agricultural experiments. Essay on principles.
(English translation of Neyman 1923.)
{\it Statistical Science\/} 5: 463--80 (with discussion).

\smallskip\noindent
Darga, K. (2000).
{\it Fixing the Census Until It Breaks\/}.
Lansing, MI: Michigan Information Center.

\smallskip\noindent
Dawid, A.~P. (2000).
Causal inference without counterfactuals.
{\it Journal of the American Statistical Association\/} 95: 407--48.

\smallskip\noindent
Dean, H.~T. (1938).
Endemic fluorosis and its relation to dental caries.
{\it Public Health Reports\/} 53: 1443--52.

\smallskip\noindent
de Finetti, B. (1959).
{\it La probabilit{\`a}, la statistica, nei
rapporti con l'indu\-zione, secondo diversi punti di vista\/}.
Rome: Centro Internazionale Matematica Estivo Cremonese.
English translation in de Finetti (1972).

\smallskip\noindent
de Finetti, B. (1972).
{\it Probability, Induction, and Statistics\/}.
New York: Wiley.

\smallskip\noindent
de Moivre, A. (1697).
A method of raising an infinite multinomial to any given power, or extracting any given root of the same.
{\it Philosophical Transactions of the Royal Society of London\/} {19}, no.~230: 619--25.

\smallskip\noindent
Diaconis, P. and Freedman, D.~A. (1980a).
De Finetti's generalizations of exchangeability.
In R.~C. Jeffrey, ed. {\it Studies in Inductive Logic and Probability\/}.
Berkeley: University of California Press, vol. 2, pp.~233--50.

\smallskip\noindent
Diaconis, P. and Freedman, D.~A. (1980b).
Finite exchangeable sequences.
{\it Annals of Probability\/} 8: 745--64.

\smallskip\noindent
Diaconis, P. and Freedman, D.~A. (1981).
Partial exchangeability and sufficiency.
In {\it Proceedings of the Indian Statistical Institute Golden Jubilee International Conference on Statistics:
Applications and New Directions\/}.
Calcutta: Indian Statistical Institute, pp.~205--36.

\smallskip\noindent
Diaconis, P. and Freedman, D.~A. (1986).
On the consistency of Bayes' estimates.
{\it Annals of Statistics\/} 14: 1--87 (with discussion).

\smallskip\noindent
Diaconis, P. and Freedman, D.~A. (1988).
Conditional limit theorems for exponential families and finite versions of de Finetti's theorem.
{\it Journal of Theoretical Probability\/} 1: 381--410.

\smallskip\noindent
Diaconis, P. and Freedman, D.~A. (1990).
Cauchy's equation and de Finetti's theorem.
{\it Scandinavian Journal of Statistics\/} 17: 235--50.

\smallskip\noindent
Dietz, K. and Heesterbeek, J.~A.~P. (2002).
Daniel Bernoulli's epidemiological model revisited.
{\it Mathematical Biosciences\/} 180: 1--21.

\smallskip\noindent
Doss, H. and Sethuraman, J. (1989).
The price of bias reduction when there is no unbiased estimate.
{\it Annals of Statistics\/} 17: 440--42.

\smallskip\noindent
Duch, R.~M. and Palmer, H.~D. (2004).
It's not whether you win or lose, but how you play the game.
{\it American Political Science Review\/} {98}: 437--52.

\smallskip\noindent
Ducharme, G.~R. and Lepage, Y. (1986).
Testing collapsibility in contingency tables.
{\it Journal of the Royal Statistical Society\/}, Series B, {48}: 197--205.

\smallskip\noindent
Duncan, O.~D. (1984).
{\it Notes on Social Measurement\/}.
New York: Russell Sage.

\smallskip\noindent
Dunford, N. and Schwartz, J.~T. (1958).
{\it Linear Operators: Part I\/}, {\it General Theory\/}.
New York: Wiley.

\smallskip\noindent
Dunnill, M.~S. (2001).
{\it The Plato of Praed Street: The Life and Times of Almroth Wright\/}.
London: Royal Society of Medicine Press.

\smallskip\noindent
Dunning, T. and Freedman, D.~A. (2007).
Modeling selection effects.
In S. Turner and W. Outhwaite, eds.
{\it The Handbook of Social Science Methodology\/}.
London: Sage Publications, pp.~225--31.

\smallskip\noindent
Dyer, A.~R., Elliott, P., Marmot, M.~et al.~(1996).
Commentary: Strength and importance of the relation of dietary salt to blood pressure.
{\it British Journal of Medicine\/} 312: 1663--65.

\smallskip\noindent
Eaton, M.~L. and Sudderth, W.~D. (1999).
Consistency and strong inconsistency of group-invariant predictive inferences.
{\it Bernoulli\/} 5: 833--54.

\smallskip\noindent
Ebrahim, S. and Davey-Smith, G. (1998).
Lowering blood pressure: A systematic review of sustained effects of non-pharmacological interventions.
{\it Journal of Public Health Medicine\/} 20: 441--48.

\smallskip\noindent
Efron, B. (1986).
Why isn't everyone a Bayesian?
{\it The American Statistician\/} 40: 1--11 (with discussion).

\smallskip\noindent
Ehrenberg, A.~S.~C. and Bound, J.~A. (1993).
Predictability and prediction.
{\it Journal of the Royal Statistical Society\/}, Series A, 156, Part 2: 167--206 (with discussion).

\smallskip\noindent
Eliot, C.~W., ed. (1910).
{\it Scientific Papers: Physiology, Medicine, Surgery, Geology\/}.
Vol. 38 in {\it The Harvard Classics\/}.
New York: P.~F.~Collier \& Son; originally published 1897.

\smallskip\noindent
Elliott, P., Stamler, J., Nichols, R.~et al.~(1996).
Intersalt revisited:
Further analyses of 24 hour sodium excretion and blood pressure within and across populations.
{\it British Journal of Medicine\/} 312: 1249--53.

\smallskip\noindent
Ellsworth, W., Matthews, M., Nadeau, R., Nishenko, S., Reasonberg, P., and Simpson, R. (1998).
A physically-based earthquake recurrence model for estimation of long-term earthquake probabilities.
In {\it Proceedings of the Second Joint Meeting of the UJNR Panel on Earthquake Research\/}, pp.~135--49.

\smallskip\noindent
Engle, R.~F., Hendry, D.~F., and Richard, J.~F. (1983).
Exogeneity.
{\it Econometrica\/} 51: 277--304.

\smallskip\noindent
Erikson, R.~S., McIver, J.~P., and Wright, Jr., G.~C. (1987).
State political culture and public opinion.
{\it American Political Science Review\/} 81: 797--813.

\smallskip\noindent
Evans, R.~J. (1987).
{\it Death in Hamburg: Society and Politics in the Cholera Years\/}.
Oxford: Oxford University Press.

\smallskip\noindent
Evans, S.~N. and Stark, P.~B. (2002).
Inverse problems as statistics.
{\it Inverse Problems\/} 18: R1--43.

\smallskip\noindent
Evans, W.~N. and Schwab, R.~M. (1995).
Finishing high school and starting college: Do Catholic schools make a difference?
{\it Quarterly Journal of Economics\/} {110}: 941--74.

\smallskip\noindent
Eyler, J.~M. (1979).
{\it Victorian Social Medicine: The Ideas and Methods of William Farr\/}.
Baltimore, MD: Johns Hopkins University Press.

\smallskip\noindent
Fahrmeir, L. and Kaufmann, H. (1985).
Consistency and asymptotic normality of the maximum likelihood estimator in generalized linear models.
{\it The Annals of Statistics\/} 13: 342--68.

\smallskip\noindent
Fearon, J.~D. and Laitin, D.~D. (2008).
Integrating qualitative and quantitative methods.
In J.~M.~Box-Steffensmeier, H.~E.~Brady, and D.~Collier, eds.
{\it The Oxford Handbook of Political Methodology\/}.
Oxford: Oxford University Press, pp.~756--76.

\smallskip\noindent
Feller, W. (1968).
{\it An Introduction to Probability Theory and Its Applications, Vol. I\/}. 3rd edn.
New York: Wiley.

\smallskip\noindent
Fenner, F., Henderson, D.~A., Arita, I., Jezek, Z., and Ladnyi, I.~D. (1988).
{\it Smallpox and its Eradication\/}.
Geneva: World Health Organization.

\noindent\hskip 10pt
http://whqlibdoc.who.int/smallpox/9241561106.pdf

\smallskip\noindent
Ferguson, T. (1967).
{\it  Mathematical Statistics:  A Decision Theoretic Approach\/}.
New York: Academic Press.

\smallskip\noindent
Finlay, B.~B., Heffron, F., and Falkow, S. (1989).
Epithelial cell surfaces induce Salmonella proteins required for bacterial adherence and invasion.
{\it Science\/} 243: 940--43.

\smallskip\noindent
Fisher, F.~M. (1980).
Multiple regression in legal proceedings.
{\it Columbia Law Review\/} 80: 702--36.

\smallskip\noindent
Fisher, R.~A. (1958).
{\it Statistical Models for Research Workers\/}. 13th edn.
Edinburgh: Oliver and Boyd.

\smallskip\noindent
Fleming, A. (1929).
On the antibacterial action of cultures of a penicillium,
with special reference to their use in the isolation of {\it B. influenzae\/}.
{\it British Journal of Experimental Pathology\/} 10: 226--36.

\smallskip\noindent
Fleming, T.~R. and Harrington, D.~P. (2005).
{\it Counting Processes and Survival Analysis\/}.
2nd rev. edn.
New York: John Wiley \& Sons.

\smallskip\noindent
Fox, J. (1991).
{\it Regression Diagnostics: An Introduction\/}.
Thousand Oaks, CA: Sage Publications, Inc.

\smallskip\noindent
Fraker, T. and Maynard, R. (1987).
The adequacy of comparison group designs for evaluations of employment-related programs.
{\it Journal of Human Resources\/} 22: 194--227.

\smallskip\noindent
Francesconi, M. and Nicoletti, C. (2006).
Intergenerational mobility and sample selection in short panels.
{\it Journal of Applied Econometrics\/} 21: 1265--93.

\smallskip\noindent
Freedman, D.~A. (1971).
{\it Markov Chains\/}.
San Francisco: Holden-Day.
Re\-printed in 1983 by Springer-Verlag, New York.

\smallskip\noindent
Freedman, D.~A. (1983).
A note on screening regression equations.
{\it The American Statistician\/} 37: 152--55.

\smallskip\noindent
Freedman, D.~A. (1985).
Statistics and the scientific method.
In W.~M.~Mason and S.~E.~Fienberg, eds.
{\it Cohort Analysis in Social Research: Beyond the Identification Problem\/}.
New York: Spring\-er-Verlag, pp.~343--90 (with discussion).

\smallskip\noindent
Freedman, D.~A. (1987).
As others see us: A case study in path analysis.
{\it Journal of Educational Statistics\/} 12(2): 101--28 (with discussion).

\smallskip\noindent
Freedman, D.~A. (1991).
Statistical models and shoe leather.
In P.~V. Marsden, ed.
{\it Sociological Methodology 1991\/} (with discussion).
Washington, D.C.: American Sociological Association.

\smallskip\noindent
Freedman, D.~A. (1995).
Some issues in the foundation of statistics.
{\it Foundations of Science\/} 1: 19--83 (with discussion).
Reprinted in B.~C.~van Fraassen, ed.
{\it Some Issues in the Foundation of Statistics\/}.
Dordrecht: Kluwer, pp. 19--83.

\smallskip\noindent
Freedman, D.~A. (1997).
From association to causation via regression.
In V. McKim and S. Turner, eds.
{\it Causality in Crisis:
Statistical Methods and the Search for Causal Knowledge in the Social Sciences\/}.
Notre Dame, IN: University of Notre Dame Press, pp. 113--61.

\smallskip\noindent
Freedman, D.~A. (1999).
From association to causation: Some remarks on the history of statistics.
{\it Statistical Science\/} 14: 243--58.
Reprinted in {\it Journal de la Soci\'et\'e Fran\c caise de Statistique\/} (1999) 140: 5--32;
and in J. Panaretos, ed. (2003)
{\it Stochastic Musings: Perspectives from the Pioneers of the Late 20th century\/}.
Mahwah, NJ: Lawrence Erlbaum Associates, pp.~45--71.

\smallskip\noindent
Freedman, D.~A. (2003).
Structural equation models: A critical review.
Technical Report No.~651, Department of Statistics,
University of California, Berkeley.

\smallskip\noindent
Freedman, D.~A. (2004).
On specifying graphical models for causation and the identification problem.
{\it Evaluation Review\/} 26: 267--93.
Reprinted in D.~W.~K. Andrews and J.~H. Stock, eds. (2005).
{\it Identification and Inference for Econometric Models: Essays in Honor of Thomas Rothenberg\/}.
Cambridge: Cambridge University Press, pp.~56--79.

\smallskip\noindent
Freedman, D.~A. (2005).
{\it Statistical Models: Theory and Practice\/}.
New York: Cambridge University Press.

\smallskip\noindent
Freedman, D.~A. (2006a).
On the so-called ``Huber Sandwich Estimator'' and ``robust standard errors.''
{\it The American Statistician\/} 60: 299--302.

\smallskip\noindent
Freedman, D.~A. (2006b).
Statistical models for causation:
What inferential leverage do they provide?
{\it Evaluation Review\/} 30: 691--713.

\smallskip\noindent
Freedman, D.~A. (2007).
How can the score test be inconsistent?
{\it The American Statistician\/} 61: 291--95.

\smallskip\noindent
Freedman, D.~A. (2008a).
On regression adjustments to experimental data.
{\it Advances in Applied Mathematics\/} 40: 180--93.

\smallskip\noindent
Freedman, D.~A. (2008b).
On regression adjustments in experiments with several treatments.
{\it Annals of Applied Statistics\/} 2: 176--96.

\smallskip\noindent
Freedman, D.~A. (2008c).
Randomization does not justify logistic regression.
{\it Statistical Science\/} 23: 237--50.

\smallskip\noindent
Freedman, D.~A. (2008e).
Survival analysis: A primer.
{\it The American Statistician\/} 62: 110--19.

\smallskip\noindent
Freedman, D.~A. (2008f).
On types of scientific enquiry: The role of qualitative reasoning.
In J.~M. Box-Steffensmeier, H.~E. Brady, and D. Collier, eds.
{\it The Oxford Handbook of Political Methodology\/}.
Oxford: Oxford University Press, pp.~300--18.

\smallskip\noindent
Freedman, D.~A. (2008g).
Diagnostics cannot have much power against general alternatives.
To appear in {\it International Journal of Forecasting\/}.

\noindent\hskip 10pt
http://www.stat.berkeley.edu/users/census/nopower.pdf

\smallskip\noindent
Freedman, D.~A. (2008h).
Some general theory for weighted regressions.

\noindent\hskip 10pt
http://www.stat.berkeley.edu/users/census/wtheory.pdf

\smallskip\noindent
Freedman, D.~A. and Humphreys, P. (1999).
Are there algorithms that discover causal structure?
{\it Synthese\/} 121: 29--54.

\smallskip\noindent
Freedman, D.~A., Klein, S.~P., Ostland, M., and Roberts, M.~R. (1998).
Review of {\it A Solution to the Ecological Inference Problem\/}.
{\it Journal of the American Statistical Association\/} 93: 1518--22;
with discussion, vol.~94 (1999): 352--57.

\smallskip\noindent
Freedman, D.~A., Klein, S.~P., Sacks, J., Smyth, C.~A., and Everett, C.~G. (1991).
Ecological regression and voting rights.
{\it Evaluation Review\/} 15: 659--817 (with discussion).

\smallskip\noindent
Freedman, D.~A. and Lane, D. (1983).
A nonstochastic interpretation of reported significance levels.
{\it Journal of Business and Economic Statistics\/} 1: 292--98.

\smallskip\noindent
Freedman, D.~A. and Navidi, W. (1989).
Multistage models for carcinogenesis.
{\it Environmental Health Perspectives\/} 81: 169--88.

\smallskip\noindent
Freedman, D.~A. and Petitti, D.~B. (2001).
Salt and blood pressure: Conventional wisdom reconsidered.
{\it Evaluation Review\/} 25: 267--87.

\smallskip\noindent
Freedman, D.~A.,  Petitti, D.~B., and Robins, J.~M. (2004).
On the efficacy of screening for breast cancer.
{\it International Journal of Epidemiology\/} 33: 43--73 (with discussion).
Correspondence, 1404--06.

\smallskip\noindent
Freedman, D.~A., Pisani, R., and Purves, R.~A. (2007).
{\it Statistics\/}. 4th edn.
New York: Norton.

\smallskip\noindent
Freedman, D.~A. and Purves, R.~A. (1969).
Bayes method for bookies.
{\it Annals of Mathematical Statistics\/} 40: 1177--86.

\smallskip\noindent
Freedman, D.~A., Rothenberg, T., and Sutch, R. (1983).
On energy policy models.
{\it Journal of Business and Economic Statistics\/} 1: 24--36 (with discussion).

\smallskip\noindent
Freedman, D.~A. and Stark, P.~B. (1999).
The swine flu vaccine and Guillain-Barr\'e syndrome:
A case study in relative risk and specific causation.
{\it Evaluation Review\/} 23: 619--47.

\smallskip\noindent
Freedman, D.~A. and Stark, P.~B. (2003).
What is the chance of an earthquake?
In F. Mulargia and R.~J. Geller, eds.
{\it Earthquake Science and Seismic Risk Reduction\/}.
NATO Science Series IV: Earth and Environmental Sciences, vol.~32.
Dordrecht, the Netherlands: Kluwer, pp.~201--16.

\smallskip\noindent
Freedman, D.~A., Stark, P.~B., and Wachter, K.~W. (2001).
A probability model for census adjustment.
{\it Mathematical Population Studies\/} 9: 165--80.

\smallskip\noindent
Freedman, D.~A. and Wachter, K.~W. (1994).
Heterogeneity and census adjustment for the intercensal base.
{\it Statistical Science\/} 9: 458--537 (with discussion).

\smallskip\noindent
Freedman, D.~A. and Wachter, K.~W. (2003).
On the likelihood of improving the accuracy of the census through statistical adjustment.
In D.~R. Goldstein, ed.
{\it Science and Statistics: A Festscrift for Terry Speed\/}.
IMS Monograph 40, pp.~197--230.

\smallskip\noindent
Freedman, D.~A. and Wachter, K.~W. (2007).
Methods for Census 2000 and statistical adjustments.
In S. Turner and W. Outhwaite, eds.
{\it Handbook of Social Science Methodology\/}.
Thousand Oaks, CA: Sage Publications, pp.~232--45.

\smallskip\noindent
Freedman, D.~A., Wachter, K.~W., Coster, D.~C., Cutler, R.~C., and Klein, S.~P. (1993).
Adjusting the Census of 1990: The smoothing model.
{\it Evaluation Review\/} 17: 371--443.

\smallskip\noindent
Freedman, D.~A., Wachter, K.~W., Cutler, R.~C., and Klein, S.~P. (1994).
Adjusting the Census of 1990: Loss functions.
{\it Evaluation Review\/} 18: 243--80.

\smallskip\noindent
Fremantle, N., Calvert, M., Wood, J.~et al.~(2003).
Composite outcomes in randomized trials: Greater precision but with greater uncertainty?
{\it Journal of the American Medical Association\/} 289: 2554--59.

\smallskip\noindent
Frey, B.~S. and Meier, S. (2004).
Social comparisons and pro-social behavior:
Testing ``conditional cooperation'' in a field experiment.
{\it American Economic Review\/} {94}: 1717--22.

\smallskip\noindent
Gail, M.~H. (1986).
Adjusting for covariates that have the same distribution in exposed and unexposed cohorts.
In S.~H. Moolgavkar and R.~L. Prentice, eds.
{\it Modern Statistical Methods in Chronic Disease Epidemiology\/}.
New York: Wiley, pp.~3--18.

\smallskip\noindent
Gail, M.~H. (1988).
The effect of pooling across strata in perfectly balanced studies.
{\it Biometrics\/} {44}: 151--62.

\smallskip\noindent
Gani, J. (1978).
Some problems of epidemic theory.
{\it Journal of the Royal Statistical Society\/},
Series A, 141: 323--47 (with discussion).

\smallskip\noindent
Gartner, S.~S. and Segura, G.~M. (2000).
Race, casualties, and opinion in the Vietnam war.
{\it Journal of Politics\/} 62: 115--46.

\smallskip\noindent
Geiger, D., Verma, T., and Pearl, J. (1990).
Identifying independence in Bayes\-ian networks.
{\it Networks\/} 20: 507--34.
New York: Wiley.

\smallskip\noindent
Geller, N.~L., Sorlie, P., Coady, S.~et al.~(2004).
Limited access data sets from studies funded by the National Heart, Lung, and Blood Institute.
{\it Clinical Trials\/} 1: 517--24.

\smallskip\noindent
Gertler, P. (2004).
Do conditional cash transfers improve child health?
Evidence from PROGRESA's control randomized experiment.
{\it American Economic Review\/} 94: 336--41.

\smallskip\noindent
Gibson, J.~L. (1988).
Political intolerance and political repression during the McCarthy Red Scare.
{\it American Political Science Review\/} 82: 511--29.

\smallskip\noindent
Giesbrecht, P., Kersten, T., Maidhof, H., and Wecke, J. (1998).
Staphylococcal cell wall: Morphogenesis and fatal variations in the presence of penicillin.
{\it Microbiology and Molecular Biology Reviews\/} 62: 1371--1414.

\smallskip\noindent
Gigerenzer, G. (1996).
On narrow norms and vague heuristics.
{\it Psychological Review\/} 103: 592--96.

{\overfullrule0pt
\smallskip\noindent
Gilens, M. (2001).
Political ignorance and collective policy preferences.
{\it American Political Science Review\/} {95}: 379--96.\par}

\smallskip\noindent
Gill, R.~D. and Robins, J.~M. (2004).
Causal inference for complex longitudinal data: The continuous case.
Technical report, School of Public Health, Harvard University.

\smallskip\noindent
Glazerman, S., Levy, D.~M., and Myers, D. (2003).
Nonexperimental versus experimental estimates of earnings impacts.
{\it Annals of the American Academy of Political and Social Science\/} 589: 63--93.

\smallskip\noindent
Godlee, F. (1996).
The food industry fights for salt.
{\it British Journal of Medicine\/} 312: 1239--40.

\smallskip\noindent
Goertz, G. (2008).
Choosing cases for case studies: A qualitative logic.
{\it Qualitative Methods\/} Fall: 9--12.

\smallskip\noindent
Goldberger, J. (1914).
The etiology of pellagra.
{\it Public Health Reports\/} 29: 1683--86.
Reprinted in Buck et al.~(1989), pp.~99--102; and in Terris (1964), pp.~19--22.

\smallskip\noindent
Goldberger, J., Wheeler, G.~A., Lillie, R.~D., and Rogers, L.~M. (1926).
A further study of butter, fresh beef, and yeast as pellagra preventives,
with consideration of the relation of factor P-P of pellegra (and black
tongue of dogs) to vitamin B1.
{\it Public Health Reports\/} 41: 297--318.
Reprinted in Terris (1964), pp.~351--70.

\smallskip\noindent
Goldsmith, M. (1946).
{\it The Road to Penicillin\/}.
London: Lindsay Drummond.

\smallskip\noindent
Goldthorpe, J.~H. (1999).
{\it Causation, Statistics and Sociology\/}.
Twenty-ninth Geary Lecture, Nuffield College, Oxford.
Published by the Economic and Social Research Institute, Dublin, Ireland.

\smallskip\noindent
Goldthorpe, J.~H. (2001).
Causation, statistics, and sociology.
{\it European Sociological Review\/} 17: 1--20.

\smallskip\noindent
Good, I.~J. (1967).
The white shoe is a red herring.
{\it The British Journal for the Philosophy of Science\/} 17: 322.

\smallskip\noindent
Good, I.~J. (1968).
The white shoe {\it qua\/} herring is pink.
{\it The British Journal for the Philosophy of Science\/} 19: 156--57.

\smallskip\noindent
Goodman, L. (1953).
Ecological regression and the behavior of individuals.
{\it American Sociological Review\/} 18: 663--64.

\smallskip\noindent
Goodman, L. (1959).
Some alternatives to ecological correlation.
{\it American Journal of Sociology\/} 64: 610--25.

\smallskip\noindent
Gordis, L. (2008).
{\it Epidemiology\/}. 4th edn.
Philadelphia, PA: Elsevier-
Saunders.

\smallskip\noindent
Gould, E.~D., Lavy, V., and Passerman, M.~D. (2004).
Immigrating to opportunity:
Estimating the effect of school quality using a natural experiment on Ethiopians in Israel.
{\it Quarterly Journal of Economics\/} 119: 489--526.

\smallskip\noindent
Gozalo, P.~L. and Miller, S.~C. (2007).
Predictors of mortality:
Hospice enrollment and evaluation of its causal effect on
hospitalization of dying nursing home patients.
{\it Health Services Research\/} 42: 587--610.

\smallskip\noindent
Grace, N.~D., Muench, H., and Chalmers, T.~C. (1966).
The present status of shunts for portal hypertension in cirrhosis.
{\it Gastroenterology\/} 50: 684--91.

\smallskip\noindent
Graudal, N.~A., Gall\o e, A.~M., and Garred, P. (1998).
Effects of sodium restriction on blood pressure, renin, aldosterone,
catecholamines, cholesterols, and triglyceride: A meta-analysis.
{\it Journal of the American Medical Association\/} 279: 1383--91.

\smallskip\noindent
Graunt, J. (1662).
{\it Natural and Political Observations Mentioned in a Following Index,
and Made upon the Bills of Mortality\/}.
London. Printed by Tho.~Roycroft, for John Martin, James Allestry, and Tho.~Dicas,
at the Sign of the Bell in St. Paul's Church-yard, MDCLXII.
Reprinted in 2006 by
Ayer Company Publishers, NH.

\noindent\hskip 10pt
http://www.ac.wwu.edu/\string~stephan/Graunt/

\smallskip\noindent
Greene, W.~H. (2007).
{\it Econometric Analysis\/}. 6th edn.
Prentice Hall, NJ: Upper Saddle River.

\smallskip\noindent
Greenland, S., Pearl, J., and Robins, J. (1999).
Causal diagrams for epidemiologic research.
{\it Epidemiology} 10: 37--48.

\smallskip\noindent
Gregg, N.~M. (1941).
Congenital cataract following German measles in the mother.
{\it Transactions of the Ophthalmological Society of Australia\/} 3: 35--46.
Reprinted in Buck et al.~(1989), pp.~426--34.

\smallskip\noindent
Gregg, N.~M. (1944).
Further observations on congenital defects in infants following maternal rubella.
{\it Transactions of the Ophthalmological Society of Australia\/} 4: 119--31.

\smallskip\noindent
Grodstein, F., Stampfer, M.~J., Colditz, G.~A.~et al.~(1997).
Postmenopausal hormone therapy and mortality.
{\it New England Journal of Medicine\/} 336: 1769--75.

\smallskip\noindent
Grodstein, F., Stampfer, M.~J., Manson, J.~et al.~(1996).
Postmenopausal estrogen and progestin use and the risk of cardiovascular disease.
{\it New England Journal of Medicine\/} 335: 453--61.

\smallskip\noindent
Grofman, B. (1991).
Statistics without substance.
{\it Evaluation Review\/} 15: 746--69.

\smallskip\noindent
Gross, S.~R. and  Mauro R. (1989).
{\it Death and Discrimination\/}.
Boston, MA: Northeastern University Press.

\smallskip\noindent
Guo, G.~H. and Geng, Z. (1995).
Collapsibility of logistic regression coefficients.
{\it Journal of the Royal Statistical Society\/}, Series B, {57}: 263--67.

\smallskip\noindent
Hald, A. (2005).
{\it A History of Probability and Statistics and Their Applications before 1750\/}.
New York: Wiley.

\smallskip\noindent
Halley, E. (1693).
An estimate of the mortality of mankind, drawn from curious tables of the births
and funerals at the city of Breslaw;
with an attempt to ascertain the price of annuities upon lives.
{\it Philosophical Transactions of the Royal Society of London\/} 196: 596--610, 654--56.

\smallskip\noindent
Hanneman, R.~L. (1996).
Intersalt: Hypertension rise with age revisited.
{\it British Journal of Medicine\/} 312: 1283--84.

\smallskip\noindent
Hare, R. (1970).
{\it The Birth of Penicillin and the Disarming of Microbes\/}.
London: Allen \& Unwin.

\smallskip\noindent
Harsha, D.~W., Lin, P.~H., Obarzanek, E.~et al.~(1999).
Dietary approaches to stop hypertension: A summary of study results.
{\it Journal of the American Dietetic Association\/} 99, Supplement: 35--39.

\smallskip\noindent
Hart, H.~L.~A. and  Honor\'e, A.~M. (1985).
{\it Causation in the Law\/}. 2nd edn.
Oxford: Oxford University Press.

\smallskip\noindent
Hartigan, J. (1983).
{\it Bayes Theory\/}.
New York: Springer-Verlag.

\smallskip\noindent
He, J., Ogden, L.~G., Vupputuri, S.~et al.~(1999).
Dietary sodium intake and subsequent risk of cardiovascular disease in overweight adults.
{\it Journal of the American Medical Association\/} 282: 2027--34.

\smallskip\noindent
Heckman, J.~J. (1978).
Dummy endogenous variables in a simultaneous equation system.
{\it Econometrica\/} 46: 931--59.

\smallskip\noindent
Heckman, J.~J. (1979).
Sample selection bias as a specification error.
{\it Econometrica\/} 47: 153--61.

\smallskip\noindent
Heckman, J.~J. (2000).
Causal parameters and policy analysis in economics: A twentieth century retrospective.
{\it The Quarterly Journal of Economics\/} 115: 45--97.

\smallskip\noindent
Heckman, J.~J. (2001a).
Micro data, heterogeneity, and the evaluation of public policy: Nobel lecture.
{\it Journal of Political Economy\/} 109: 673--748.

\smallskip\noindent
Heckman, J.~J. (2001b).
Econometrics and empirical economics.
{\it Journal of Econometrics\/} 100: 3--5.

\smallskip\noindent
Heckman, J.~J. and Hotz, V.~J. (1989).
Choosing among alternative nonexperimental methods for estimating the impact of social programs:
The case of manpower training.
{\it Journal of the American Statistical Association\/} 84: 862--80.

\smallskip\noindent
Hedges, L.~V. and Olkin, I. (1985).
{\it Statistical Methods for Meta-Analysis\/}.
New York: Academic Press.

\smallskip\noindent
Hedstr\"om, P. and Swedberg, R., eds. (1998).
{\it Social Mechanisms\/}.
Cambridge: Cambridge University Press.

\smallskip\noindent
Heiss, G., Wallace, R., Anderson, G.~L.~et al.~(2008).
Health risks and benefits 3 years after stopping randomized treatment with estrogen and progestin.
{\it Journal of the American Medical Association\/} 299: 1036--45.

\smallskip\noindent
Hempel, C.~G. (1945).
Studies in the logic of confirmation.
{\it Mind\/} 54: 1--26, 97--121.

\smallskip\noindent
Hempel, C.~G. (1967).
The white shoe: No red herring.
{\it The British Journal for the Philosophy of Science\/} 18: 239--40.

\smallskip\noindent
Hendry, D.~F. (1980).
Econometrics---alchemy or science?
{\it Economica\/} 47: 387--406.
Reprinted in D.~F. Hendry (2000), {\it Econometrics---Alchemy or Science?\/}
Oxford: Blackwell, chapter 1. Page cites are to the journal article.

\smallskip\noindent
Henschke, C.~I., Yankelevitz, D.~F., Libby, D.~M.~et al.~(2006).
The International Early Lung Cancer Action Program Investigators.
Survival of patients with Stage I lung cancer detected on CT screening.
{\it New England Journal of Medicine\/} 355: 1763--71.

\smallskip\noindent
Herbst, A.~L. and Scully, R.~E. (1970).
Adenocarcinoma of the vagina in adolescence:
A report of 7 cases including 6 clear cell carcinomas.
{\it Cancer\/} 25: 745--57.

\smallskip\noindent
Herbst, A.~L., Ulfelder, H., and Poskanzer, D.~C. (1971).
Adenocarcinoma of the vagina: Association of maternal stilbestrol therapy with tumor appearance in young women.
{\it New England Journal of Medicine\/} 284: 878--81.
Reprinted in Buck et al.~(1989), pp.~446--50.

\smallskip\noindent
Hern\'an, M.~A., Brumback, B., Robins, J.~M. (2001).
Marginal structural models to estimate the joint causal effects of nonrandomized treatments.
{\it Journal of the American Statistical Association\/} 96: 440--48.

\smallskip\noindent
Hill, A.~B. (1961).
{\it Principles of Medical Statistics\/}. 7th edn.
London: The Lancet.

\smallskip\noindent
Hirano, K. and Imbens, G.~W. (2001).
Estimation of causal effects using propensity score weighting:
An application to data on right heart catheterization.
{\it Health Services and Outcomes Research Methodology\/} 2: 259--78.

\smallskip\noindent
Hodges, J.~L., Jr. and Lehmann, E. (1964).
{\it Basic Concepts of Probability and Statistics\/}.
San Francisco: Holden-Day.

\smallskip\noindent
Hoeffding, W. (1951).
A combinatorial central limit theorem.
{\it Annals of Mathematical Statistics\/} 22: 558--66.

\smallskip\noindent
Hoeffding, H. (1963).
Probability inequalities for sums of bounded random variables.
{\it Journal of the American Statistical Association\/} 58: 13--30.

\smallskip\noindent
H\"oglund, T. (1978).
Sampling from a finite population: A remainder term estimate.
{\it Scandinavian Journal of Statistics\/} 5: 69--71.

\smallskip\noindent
Holland, P.~W. (1986).
Statistics and causal inference.
{\it Journal of the American Statistical Association\/} 8: 945--70 (with discussion).

\smallskip\noindent
Holland, P.~W. (1988).
Causal inference, path analysis, and recursive structural equation models.
In C. Clogg, ed. {\it Sociological Methodology 1988\/}.
Washington, D.C.: American Sociological Association, pp.~449--93.

\smallskip\noindent
Howard-Jones, N. (1975).
{\it The Scientific Background of the International Sanitary Conferences 1851--1938\/}.
Geneva: World Health Organization.

\smallskip\noindent
Hu, W.-Y. (2003).
Marriage and economic incentives: Evidence from a welfare experiment.
{\it Journal of Human Resources\/} {38}: 942--63.

\smallskip\noindent
Huber, P.~J. (1967).
The behavior of maximum likelihood estimates under nonstandard conditions.
{\it Proceedings of the Fifth Berkeley Symposium on Mathematical Statistics and Probability\/},
vol.~I, pp.~221--33.

\smallskip\noindent
Humphreys, N.~A., ed. (1885).
{\it Vital Statistics: A Memorial Volume of Selections from
the Reports and Writings of William Farr\/}.
London: Edward Stanford.

\smallskip\noindent
Humphreys, P. (1997).
A critical appraisal of causal discovery algorithms.
In V. McKim and S. Turner, eds.
{\it Causality in Crisis:  Statistical Methods and the Search for Causal Knowledge in the Social Sciences\/}.
Notre Dame, IN: University of Notre Dame Press, pp.~249--63.

\smallskip\noindent
Humphreys, P. and Freedman, D.~A. (1996).
The grand leap.
{\it British Journal for the Philosophy of Science\/} 47: 113--23.

\smallskip\noindent
International Agency for Research on Cancer (1986).
{\it Tobacco Smoking\/}.\break
Monographs on the Evaluation of the Carcinogenic Risk of Chemicals to Humans,
Vol.~38. IARC, Lyon, France.

\smallskip\noindent
Intersalt Cooperative Research Group (1986).
Intersalt study. An international co-operative study on the relation of blood pressure
to electrolyte excretion in populations. Design and methods.
{\it Journal of Hypertension\/} 4: 781--87.

\smallskip\noindent
Intersalt Cooperative Research Group (1988).
Intersalt: An international study of electrolyte excretion and blood pressure.
Results for 24 hour urinary sodium and potassium excretion.
{\it British Journal of Medicine\/} 297: 319--28.

\smallskip\noindent
Iyengar, S.~I. and Greenhouse, J.~B. (1988).
Selection models and the file drawer problem.
{\it Statistical Science\/} 3: 109--17.

\smallskip\noindent
Jacobs, D. and Carmichael, J.~T. (2002).
The political sociology of the death penalty.
{\it American Sociological Review\/} 67: 109--31.

\smallskip\noindent
Janssen, A. (2000).
Global power functions of goodness-of-fit tests.
{\it Annals of Statistics\/} 28: 239--53.

\smallskip\noindent
Jeffrey, R.~C. (1983).
{\it  The Logic of Decision\/}. 2nd edn.
Chicago: University of Chicago Press.

\smallskip\noindent
Jenner, E. (1798).
{\it An Inquiry into the Causes and Effects of the Variolae Vaccinae,
a Disease Discovered in Some of the Western Counties of England,
Particularly Gloucestershire, and Known by the Name of the Cow Pox\/}.
London: printed for the author by Sampson Low.
Reprinted in Eliot (1910), pp.~151--80.

\smallskip\noindent
Jenner, E. (1801).
{\it The Origin of the Vaccine Inoculation\/}.
London: D.~N.~Shury.
Reprinted in Fenner et al.~(1988), pp.~258--61.

\smallskip\noindent
Jewell, N.~P. (2003).
{\it Statistics for Epidemiology\/}.
Boca Raton, FL: Chapman \& Hall/CRC.

\smallskip\noindent
Johnston, J. (1984).
{\it Econometric Methods\/}.
New York: McGraw-Hill.

\smallskip\noindent
Kahneman, D., Slovic, P., and Tversky, A., eds. (1982).
{\it Judgment Under Uncertainty:  Heuristics and Biases\/}.
Cambridge: Cambridge University Press.

\smallskip\noindent
Kahneman, D. and Tversky, A. (1974).
Judgment under uncertainty:  Heuristics and bias.
{\it Science\/} 185: 1124--31.

\smallskip\noindent
Kahneman, D. and Tversky, A. (1996).
On the reality of cognitive illusions.
{\it Psychological Review\/} 103: 582--91.

\smallskip\noindent
Kahneman, D. and Tversky, A., eds. (2000).
{\it Choices, Values, and Frames\/}.
Cambridge: Cambridge University Press.

\smallskip\noindent
Kalbfleisch, J.~D. and Prentice, R.~L. (1973).
Marginal likelihoods based on Cox's regression and life model.
{\it Biometrika\/} 60: 267--78.

\smallskip\noindent
Kalbfleisch, J.~D. and Prentice, R.~L. (2002).
{\it The Statistical Analysis of Failure Time Data\/}. 2nd edn.
New York: Wiley.

\smallskip\noindent
Kanarek, M.~S., Conforti, P.~M., Jackson, L.~A., Cooper, R.~C., and Murchio, J.~C. (1980).
Asbestos in drinking water and cancer incidence in the San Francisco Bay Area.
{\it American Journal of Epidemiology\/} 112: 54--72.

\smallskip\noindent
Kang, J.~D.~Y. and Schafer, J.~L. (2007).
Demystifying double robustness: A comparison of alternative strategies
for estimating a population mean from incomplete data.
{\it Statistical Science\/} 22: 523--39.

\smallskip\noindent
Kaplan, E.~L. and Meier, P. (1958).
Nonparametric estimation from incomplete observations.
{\it Journal of American Statistical Association\/} 53: 457--81.

\smallskip\noindent
Kaprio, J. and Koskenvuo, M. (1989).
Twins, smoking and mortality: A 12-year prospective study of smoking-discordant twin pairs.
{\it Social Science and Medicine} 29: 1083--89.

\smallskip\noindent
Kaye, D.~H. and Freedman, D.~A. (2000).
Reference Guide on Statistics.
In {\it Reference Manual on Scientific Evidence\/}. 2nd edn.
Washington, D.C.: Federal Judicial Center, pp.~83--178.

\smallskip\noindent
Kempthorne, O. (1952).
{\it The Design and Analysis of Experiments\/}.
New York: Wiley.

\smallskip\noindent
Kennedy, P. (2003).
{\it A Guide to Econometrics\/}. 5th edn.
Cambridge, MA: MIT Press.

\smallskip\noindent
Keynes, J.~M. (1939).
Professor Tinbergen's method.
{\it The Economic Journal\/} 49: 558--70.

\smallskip\noindent
Keynes, J.~M. (1940).
Comment on Tinbergen's response.
{\it The Economic Journal\/} 50: 154--56.

\smallskip\noindent
Kiiveri, H. and Speed, T. (1982).
Structural analysis of multivariate data: A review.
In S. Leinhardt, ed. {\it Sociological Methodology 1982\/}.
San Francisco: Jossey Bass, pp.~209--89.

\smallskip\noindent
King, G. (1997).
{\it A Solution to the Ecological Inference Problem\/}.
Princeton, NJ: Princeton University Press.

\smallskip\noindent
King, G. (1999).
A reply to Freedman et al.
{\it Journal of the American Statistical Association\/} 94: 352--55.

\smallskip\noindent
Kirk, D. (1996).
Demographic transition theory.
{\it Population Studies\/} 50: 361--87.

\smallskip\noindent
Klein, S.~P. and Freedman, D.~A. (1993).
Ecological regression in voting rights cases.
{\it Chance\/} 6: 38--43.

\smallskip\noindent
Klein, S.~P., Sacks, J., and Freedman, D.~A. (1991).
Ecological regression {\it versus} the secret ballot.
{\it Jurimetrics\/} 31: 393--413.

\smallskip\noindent
Koch, C.~G. and Gillings, D.~B. (2005).
Inference, design-based vs.~model-based.
In S. Kotz, C.~B. Read, N. Balakrishnan, and B. Vidakovic, eds.
{\it Encyclopedia of Statistical Sciences\/}. 2nd edn.
Hoboken, NJ: Wiley.

\smallskip\noindent
Koenker, R. (2005).
Maximum likelihood asymptotics under nonstandard conditions:
A heuristic introduction to sandwiches.

\noindent\hskip 10pt{
www.econ.uiuc.edu/{\lower1pt\hbox{\char126}}roger/courses/476/lectures/L10.pdf
}

%\smallskip\noindent
%Kolmogorov, A. N. (1933). Grundbegriffe der
%Wahrscheinlichkeitstheorie. {\it Ergebnisse Mathematische\/} 2 no. 3.

\smallskip\noindent
Kolmogorov, A.~N. (1933).
{\it Foundations of the Theory of Probability\/}.
Original in German, {\it Ergebnisse Mathematische\/} 2: no.~3.
English translation reprinted in 1956 by Chelsea, New York.

\smallskip\noindent
Kolmogorov, A.~N. (1956).
{\it Foundations of the Theory of Probability\/}. 2nd edn.
New York: Chelsea.

\smallskip\noindent
Kreps, D. (1988).
{\it  Notes on the Theory of Choice\/}.
Boulder: Westview Press.

\smallskip\noindent
Kruskal, W. (1988).
Miracles and statistics, the casual assumption of independence.
{\it Journal of the American Statistical Association} 83: 929--40.

\smallskip\noindent
Kumanyika, S.~K. and Cutler, J.~A. (1997).
Dietary sodium reduction: Is there cause for concern?
{\it Journal of the American College of Nutrition\/} 16: 192--203.

\smallskip\noindent
LaLonde, R.~J. (1986).
Evaluating the econometric evaluations of training programs with experimental data.
{\it American Economic Review\/} 76: 604--20.

\smallskip\noindent
Lancaster, P.~A.~L. (1996).
Gregg, Sir Norman McAlister (1892--1966), Ophthalmologist.
In J. Ritchie, ed. {\it\/ Australian Dictionary of Biography\/}.
Melbourne: Melbourne University Press, vol.~14, pp.~325--27.

\noindent\hskip 10pt
http://www.adb.online.anu.edu.au/biogs/A140370b.htm

\smallskip\noindent
Lane, P.~W. and Nelder, J.~A. (1982).
Analysis of covariance and standardization as instances of prediction.
{\it Biometrics\/} {38}: 613--21.

\smallskip\noindent
Laplace, P.~S. (1774).
Memoire sur la probabilit\'e des causes par les \'ev\'ene\-ments.
{\it Memoires de math\'ematique et de physique present\'es \`a l'acad\'emie
royale des sciences,
par divers savants, et lus dans ses assembl\'ees\/} 6.
Re\-printed in Laplace's {\it Oeuvres Compl\`etes\/} 8: 27--65.
English translation by S.~Stigler (1986).
{\it Statistical Science\/} 1: 359--78.

\smallskip\noindent
Lassen, D.~D. (2005).
The effect of information on voter turnout: Evidence from a natural experiment.
{\it American Journal of Political Science\/} 49: 103--18.

\smallskip\noindent
Lauritzen, S.~L. (1996).
{\it Graphical Models\/}.
Oxford: Oxford University Press.

\smallskip\noindent
Lauritzen, S. (2001).
Causal inference in graphical models.
In O.~E. Barndorff-Nielsen, D.~R. Cox, and C. Kl\"uppelberg, eds.
{\it Complex Stochastic Systems\/}.
Boca Raton, FL: Chapman \& Hall/CRC, pp.~63--108.

\smallskip\noindent
Law, M. (1996).
Commentary: Evidence on salt is consistent.
{\it British Journal of Medicine\/} 312: 1284--85.

\smallskip\noindent
Lawless, J.~F. (2003). %(Jerald F.), 1944-
{\it Statistical Models and Methods for Lifetime Data\/}. 2nd edn.
New York: Wiley.

\smallskip\noindent
Leamer, E. (1978).
{\it Specification Searches\/}.
New York: Wiley.

\smallskip\noindent
Le Cam, L.~M. (1977).
A note on metastatistics or ``An essay toward stating a problem in the doctrine of chances.''
{\it Synthese\/} 36: 133--60.

\smallskip\noindent
Le Cam, L.~M. (1986).
{\it Asymptotic Methods in Statistical Decision Theory\/}.
New York: Springer-Verlag.

\smallskip\noindent
Le Cam, L.~M. and Yang, G.~L. (1990).
{\it Asymptotics in Statistics:  Some Basic Concepts\/}.
New York: Springer-Verlag.

\smallskip\noindent
Lee, E.~T. and Wang, J.~W. (2003).
{\it Statistical Methods for Survival Data Analysis\/}. 3rd edn.
New York: Wiley.

\smallskip\noindent
Lee, L.~F. (1981).
Simultaneous equation models with discrete and censored dependent variables.
In C. Manski and D. McFadden, eds.
{\it Structural Analysis of Discrete Data with Economic Applications\/}.
Cambridge, MA: MIT Press, pp.~346--64.

\smallskip\noindent
Legendre, A.~M. (1805).
{\it Nouvelles m\'ethodes pour la d\'etermination des orbites des com\`etes.\/}
Paris: Courcier.
Reprinted in 1959 by Dover, New York.

\smallskip\noindent
Lehmann, E.~L. (1986).
{\it Testing Statistical Hypotheses\/}. 2nd edn.
New York: Wiley.

\smallskip\noindent
Lehmann, E.~L.~(1998).
{\it Elements of Large-Sample Theory\/}.
New York: Sprin\-ger-Verlag.

\smallskip\noindent
Lehmann, E.~L. and Casella, G. (2003).
{\it Theory of Point Estimation\/}. 2nd edn.
New York: Springer-Verlag.

\smallskip\noindent
Lehmann, E. and Romano, J. (2005).
{\it Testing Statistical Hypotheses\/}. 3rd edn.
New York: Springer-Verlag.

\smallskip\noindent
Lehrer, E. (2001).
Any inspection rule is manipulable.
{\it Econometrica\/} 69: 1333--47.

\smallskip\noindent
Leslie, S. and Theibaud, P. (2007).
Using propensity scores to adjust for treatment selection bias.
SAS Global Forum 2007: Statistics and Data Analysis, paper 184-2007.

\smallskip\noindent
Lichtman, A. (1991).
Passing the test.
{\it Evaluation Review\/} 15: 770--99.

\smallskip\noindent
Lieberson, S. (1985).
{\it Making It Count: The Improvement of Social Theory and Research\/}.
Berkeley: University of California Press.

\smallskip\noindent
Lieberson, S. (1988).
Asking too much, expecting too little.
{\it Sociological Perspectives\/} 31: 379--97.

\smallskip\noindent
Lieberson, S. and Lynn, F.~B. (2002).
Barking up the wrong branch:
Alternatives to the current model of sociological science.
{\it Annual Review of Sociology\/} 28: 1--19.

\smallskip\noindent
Lieberson, S. and Waters, M. (1988).
{\it From Many Strands: Ethnic and Racial Groups in Contemporary America\/}.
New York: Russell Sage Foundation.

\smallskip\noindent
Lim, W. (1999).
Estimating impacts on binary outcomes under random assignment.
Technical report, MDRC, New York.

\smallskip\noindent
Lipsey, M.~W. (1992).
Juvenile delinquency treatment: A meta-analysis inquiry into the variability of effects.
In T.~C. Cook, D.~S. Cooper, H. Hartmann et al., eds.
{\it Meta-Analysis for Explanation\/}.
New York: Russell Sage, pp.~83--127.

\smallskip\noindent
Lipsey, M.~W. (1997).
What can you build with thousands of bricks?
Musings on the cumulation of knowledge in program evaluation.
{\it New Directions for Evaluation\/} 76 (Winter): 7--24.

\smallskip\noindent
Lipsey, M.~W. and Wilson, D. (2001).
{\it Practical Meta-Analysis\/}.
Newbury Park, CA: Sage Publications.

\smallskip\noindent
Littlewood, J. (1953).
{\it A Mathematician's Miscellany\/}.
London: Methuen \& Co. Ltd.

\smallskip\noindent
Liu, T.~C. (1960).
Under-identification, structural estimation, and forecasting.
{\it Econometrica\/} 28: 855--65.

\smallskip\noindent
Lombard, H.~L. and Doering, C.~R. (1928).
Cancer studies in Massachusetts, 2.
Habits, characteristics and environment of individuals with and without lung cancer.
{\it New England Journal of Medicine\/} 198: 481--87.

\smallskip\noindent
Lorentz, G.~G. (1986).
{\it Bernstein Polynomials\/}. 2nd edn.
New York: Chelsea.

\smallskip\noindent
Loudon, I. (2000).
{\it The Tragedy of Childbed Fever\/}.
Oxford: Oxford University Press.

\smallskip\noindent
Louis, P. (1986 [1835]).
{\it Researches on the Effects of Bloodletting in Some Inflammatory Diseases,
and the Influence of Emetics and Vesication in Pneumonitis\/}.
Translated and reprinted.
Birmingham, AL: Classics of Medicine Library.

\smallskip\noindent
Lucas, R.~E., Jr. (1976).
Econometric policy evaluation: A critique.
In K.~Brunner and A.~Meltzer, eds.
{\it The Phillips Curve and Labor Markets\/}.
The Carnegie-Rochester Conferences on Public Policy,
supplementary series to the {\it Journal of Monetary Economics\/}.
Amsterdam: North-Holland, vol.~1, pp.~19--64 (with discussion).

\smallskip\noindent
Lunceford, J.~K. and Davidian, M. (2004).
Stratification and weighting via the propensity score in estimation of causal treatment effects:
A comparative study.
{\it Statistics in Medicine\/} 23: 2937--60.

\smallskip\noindent
MacGregor, G.~A. and Sever, P.~S. (1996).
Salt---overwhelming evidence but still no action:
Can a consensus be reached with the food industry?
{\it British Journal of Medicine\/} 312: 1287--89.

\smallskip\noindent
MacKenzie, D.~L. (1991).
The parole performance of offenders released from shock incarceration (boot camp prisons):
A survival time analysis.
{\it Journal of Quantitative Criminology} 7(3): 213--36.

\smallskip\noindent
Mahoney, J. and Goertz, G. (2004).
The possibility principle: Choosing negative cases in comparative research.
{\it The American Political Science Review\/} 98: 653--69.

\smallskip\noindent
Mahoney, J. and Rueschemeyer, D. (2003).
{\it Comparative Historical Analysis in the Social Sciences\/}.
Cambridge: Cambridge University Press.

\smallskip\noindent
Manski, C.~F. (1995).
{\it Identification Problems in the Social Sciences\/}.
Cambridge, MA: Harvard University Press.

\smallskip\noindent
Marini, M.~M., and Singer, B. (1988).
Causality in the social sciences.
In C. Clogg, ed. {\it Sociological Methodology 1988\/}.
Washington, D.C.: American Sociological Association, pp.~347--409.

\smallskip\noindent
Massey, D.~S. (1981).
Dimensions of the new immigration to the United States and the prospects for assimilation.
{\it Annual Review of Sociology\/} 7: 57--85.

\smallskip\noindent
Massey, D.~S. and Denton, N.~A. (1985).
Spatial assimilation as a socioeconomic outcome.
{\it American Sociological Review\/} 50: 94--105.

\smallskip\noindent
McCarron, D.~A. and Reusser, M.~E. (1999).
Finding consensus in the dietary calcium-blood pressure debate.
{\it Journal of the American College of Nutrition\/} 18, Supplement: S398--405.

\smallskip\noindent
McClure, F.~J. (1970).
{\it Water Fluoridation\/}.
Bethesda, MD: National Institute of Dental Research.

\smallskip\noindent
McCue, K.~F. (1998).
Deconstructing King: Statistical Problems in {\it A Solution to the Ecological Inference Problem\/}.
Technical Report, California Institute of Technology, Pasadena, CA.

\smallskip\noindent
McKay, F.~S. (1928).
Relation of mottled enamel to caries.
{\it Journal of the American Dental Association\/} 15: 1429--37.

\smallskip\noindent
McKay, F.~S. and Black, G.~V. (1916).
An investigation of mottled teeth: An endemic developmental imperfection of the enamel of the
teeth, heretofore unknown in the literature of dentistry.
{\it Dental Cosmos\/} 58: 477--84, 627--44, 781--92, 894--904.

\smallskip\noindent
McNiel, D.~E. and Binder, R.~L. (2007).
Effectiveness of mental health court in reducing recidivism and violence.
{\it American Journal of Psychiatry\/} 164: 1395--1403.

\smallskip\noindent
Meehl, P.~E. (1954).
{\it Clinical Versus Statistical Prediction:
A Theoretical Analysis and a Review of the Evidence\/}.
Minneapolis, MN: University of Minnesota Press.

\smallskip\noindent
Meehl, P.~E. (1978).
Theoretical risks and tabular asterisks:
Sir Karl, Sir Ronald, and the slow progress of soft psychology.
{\it Journal of Consulting and Clinical Psychology\/} 46: 806--34.

\smallskip\noindent
Meehl, P.~E. and N.~G. Waller. (2002).
The path analysis controversy: A new statistical approach to strong appraisal of verisimilitude.
{\it Psychological Methods\/} 7: 283--337 (with discussion).

\smallskip\noindent
Middleton, J. (2007).
Even for randomized experiments, logistic regression is not generally consistent.
Technical report, Political Science Department, Yale University.

\smallskip\noindent
Midgley, J.~P., Matthew, A.~G., Greenwood, C.~M., and Logan, A.~G. (1996).
Effect of reduced dietary sodium on blood pressure.
{\it Journal of the American Medical Association\/} 275: 1590--97.

\smallskip\noindent
Miller, D.~P., Neuberg, D., De Vivo, I.~et al.~(2003).
Smoking and the risk of lung cancer:
Susceptibility with GSTP1 polymorphisms.
{\it Epidemiology\/} 14: 545--51.

\smallskip\noindent
Miller, J.~F., Mekalanos, J.~J., and Falkow, S. (1989).
Coordinate regulation and sensory transduction in the control of bacterial virulence.
{\it Science\/} 243: 916--22.

\smallskip\noindent
Miller, R.~G., Jr. (1998).
{\it Survival Analysis\/}.
New York: Wiley.

\smallskip\noindent
Mills, J.~P. (1926).
Table of the ratio: Area to boundary ordinate, for any portion of the normal curve.
{\it Biometrika\/} 18: 395--400.

\smallskip\noindent
Moore, T.~J., Vollmer, W.~M., Appel, L.~J.~et al.~(1999).
Effect of dietary patterns on ambulatory blood pressure:
Results from the Dietary Approaches to Stop Hypertension (DASH) Trial.
DASH Collaborative Research Group.
{\it Hypertension\/} 34: 472--77.

\smallskip\noindent
Mueller, F.~H. (1939).
Tabakmissbrauch und Lungcarcinom (Tobacco abuse and lunch cancer).
{\it Zeitschrift fur Krebsforsuch\/} 49: 57--84.

\smallskip\noindent
Muthen, B. (1979).
A structural probit model with latent variables.
{\it Journal of the American Statistical Association\/} 74: 807--11.

\smallskip\noindent
Nagin, D.~S. and Paternoster, R. (1993).
Enduring individual differences and rational choice theories of crime.
{\it Law \& Society Review\/} 27(3): 467--96.

\smallskip\noindent
Nakachi, K., Ima, K., Hayashi, S.-I., and Kawajiri, K. (1993).
Polymorphisms of the CYP1A1 and glutathione S-transferase genes associated with susceptibility
to lung cancer in relation to cigarette dose in a Japanese population.
{\it Cancer Research} 53: 2994--99.

\smallskip\noindent
National Research Council (1997).
{\it Possible Health Effects of Exposure to Residential Electric and Magnetic Fields\/}.
Washington, D.C.: National Academy of Science.

\smallskip\noindent
Netto, E. (1927).
{\it Lehrbuch der Combinatorik\/}.
Leipzig: B.~G.~Teubner.

\smallskip\noindent
Neyman, J. (1923).
Sur les applications de la th\'eorie des probabilit\'es aux experiences agricoles:
Essai des principes.
{\it Roczniki Nauk Rolniczych\/} 10: 1--51, in Polish.
English translation by Dabrowska and Speed (1990).

\smallskip\noindent
Neyman, J., Kolodziejczyk, S., and Iwaszkiewicz, K. (1935).
Statistical problems in agricultural experimentation.
{\it Journal of the Royal Statistical Society\/} {2}, Supplement: 107--54.

\smallskip\noindent
N\'\i\ Bhrolch\'ain, M. (2001).
Divorce effects and causality in the social sciences.
{\it European Sociological Review\/} 17: 33--57.

\smallskip\noindent
Nicod, J. (1930).
{\it Foundations of Geometry and Induction\/}.
Translated from the French by P.~P. Wiener.
New York: Harcourt Brace. %p219

\smallskip\noindent
Nuland, S. (1979).
The enigma of Semmelweis---An interpretation.
{\it Journal of the History of Medicine and Allied Sciences\/} 34: 255--72.

\smallskip\noindent
Nutton, V., ed. (2008).
{\it Pestilential Complexities\/}.
London: Wellcome Trust.

\smallskip\noindent
Oakes, M. (1990).
{\it Statistical Inference\/}.
Chestnut Hill, MA: Epidemiology Resources Inc.

\smallskip\noindent
Olszewski, W. and Sandroni, A. (2008).
Manipulability of future-indepen\-dent tests.
{\it Econometrica\/} 76(6): 1437--66.

\smallskip\noindent
Ono, H. (2007).
Careers in foreign-owned firms in Japan.
{\it American Sociological Review\/} 72: 267--90.

\smallskip\noindent
Pargament, K.~I., Koenig, H.~G., Tarakeshwar, N., and Hahn, J. (2001).
Religious struggle as a predictor of mortality among medically ill patients.
{\it Archives of Internal Medicine\/} 161: 1881--85.

\smallskip\noindent
Pasteur, L. (1878).
{\it La th\'eorie des germes et ses applications \`a la m\'edecine et \`a la chirurgie,\/}
lecture faite \`a l'Academie de M\'edecine le 30 avril 1878,
par M. Pasteur en son nom et au nom de MM. Joubert et Chamberland.
Paris: G. Masson.

\smallskip\noindent
Pate, A.~M. and Hamilton, E.~E. (1992).
Formal and informal deterrents to domestic violence:
The Dade county spouse assault experiment.
{\it American Sociological Review\/} {57}: 691--97.

\smallskip\noindent
Patz, E.~F., Jr., Goodman, P.~C., and Bepler, G. (2000).
Screening for lung cancer.
{\it New England Journal of Medicine\/} 343: 1627--33.

\smallskip\noindent
Pearl, J. (1988).
{\it Probabilistic Reasoning in Intelligent Systems\/}.
San Mateo, CA: Morgan Kaufmann Publishers, Inc.

\smallskip\noindent
Pearl, J. (1995).
Causal diagrams for empirical research.
{\it Biometrika} 82: 669--710 (with discussion).

\smallskip\noindent
Pearl, J. (2000).
{\it Causality: Models, Reasoning, and Inference\/}.
Cambridge: Cambridge University Press.

\smallskip\noindent
Peikes, D.~N., Moreno, L., and Orzol, S.~M. (2008).
Propensity score matching: A note of caution for evaluators of social programs.
{\it The American Statistician\/} 62: 222--31.

\smallskip\noindent
Petitti, D.~B. (1994).
Coronary heart disease and estrogen replacement therapy:
Can compliance bias explain the results of observational studies?
{\it Annals of Epidemiology\/} 4: 115--18.

\smallskip\noindent
Petitti, D.~B. (1998).
Hormone replacement therapy and heart disease prevention: Experimentation trumps observation.
{\it Journal of the American Medical Association\/} 280: 650--52.

\smallskip\noindent
Petitti, D.~B. (1999).
{\it Meta-Analysis, Decision Analysis, and Cost-Effect\-ive\-ness Analysis\/}. 2nd edn.
New York: Oxford University Press.

\smallskip\noindent
Petitti, D.~B. (2002).
Hormone replacement therapy for prevention.
{\it Journal of the American Medical Association\/} 288: 99--101.

\smallskip\noindent
Petitti, D.~B. and Chen, W. (2008).
Statistical adjustment for a measure of healthy lifestyle
doesn't yield the truth about hormone therapy.
In D. Nolan and T. Speed, eds.
{\it Probability and Statistics: Essays in Honor of David A.~Freedman\/}.
Institute of Mathematical Statistics, pp.~142--52.

\smallskip\noindent
Petitti, D.~B. and Freedman, D.~A. (2005).
Invited commentary: How far can epidemiologists get with statistical adjustment?
{\it American Journal of Epidemiology\/} 162: 415--18.

\smallskip\noindent
Phillips, S. and Grattet, R. (2000).
Judicial rhetoric, meaning-making, and the institutionalization of hate crime law.
{\it Law \& Society Review} 34(3): 567--606.

\smallskip\noindent
Port, S., Demer, L., Jennrich, R., Walter, D., and Garfinkel, A. (2000).
Systolic blood pressure and mortality.
{\it Lancet\/} 355: 175--80.

\smallskip\noindent
Porter, R. (1997).
{\it The Greatest Benefit to Mankind\/}.
New York: Norton. %pp. 258-77.

\smallskip\noindent
Prakasa Rao, B.~L.~S. (1987).
{\it  Asymptotic Theory of Statistical Inference\/}.
New York: Wiley.

\smallskip\noindent
Pratt, J.~W. (1981).
Concavity of the log likelihood.
{\it Journal of the American Statistical Association\/} {76}: 103--06.

\smallskip\noindent
Pratt, J.~W. and Schlaifer, R. (1984).
On the nature and discovery of structure.
{\it Journal of the American Statistical Association\/} 79: 9--33 (with discussion).

\smallskip\noindent
Pratt, J.~W. and Schlaifer, R. (1988).
On the interpretation and observation of laws.
{\it Journal of Econometrics\/} 39: 23--52.

\smallskip\noindent
Prewitt, K. (2000).
Accuracy and coverage evaluation:
Statement on the feasibility of using statistical methods to improve the accuracy of Census 2000.
{\it Federal Register\/} 65: 38, 373--38, 398.

\smallskip\noindent
Psaty, B.~M., Weiss, N.~S., Furberg, C.~D.~et al.~(1999).
Surrogate end points, health outcomes, and the drug-approval process
for the treatment of risk factors for cardiovascular disease.
{\it Journal of the American Medical Association\/} 282: 786--90.

\smallskip\noindent
Ramsey, F.~P. (1926), in R.~B. Braithwaite (1931).
{\it  The Foundations of Mathematics and other Logical Essays\/}.
London: Routledge and Kegan Paul.

\smallskip\noindent
Rao, C.~R. (1973).
{\it Linear Statistical Inference and its Applications}. 2nd edn.
New York: Wiley.

\smallskip\noindent
Redfern, P. (2004).
An alternative view of the 2001 census and future census taking.
{\it Journal of the Royal Statistical Society\/},
Series A, 167: 209--48 (with discussion).

\smallskip\noindent
Reif, F. (1965).
{\it Fundamentals of Statistical and Thermal Physics\/}.
New York: McGraw-Hill.

\smallskip\noindent
Resnick, L.~M. (1999).
The role of dietary calcium in hypertension: A hierarchical overview.
{\it American Journal of Hypertension\/} 12: 99--112.

\smallskip\noindent
Ridgeway, G., McCaffrey, D., and Morral, A. (2006).
Toolkit for weighting and analysis of nonequivalent groups: A tutorial for the TWANG package.
RAND Corporation, Santa Monica, CA.

\smallskip\noindent
Rindfuss, R.~R., Bumpass, L., and St. John, C. (1980).
Education and fertility: Implications for the roles women occupy.
{\it \ American Sociological Review\/} 45: 431--47.

\smallskip\noindent
Rivers, D. and Vuong, Q.~H. (1988).
Limited information estimators and exogeneity tests for simultaneous probit models.
{\it Journal of Econometrics\/} 39: 347--66.

\smallskip\noindent
Robins, J.~M. (1986).
A new approach to causal inference in mortality studies with a sustained exposure period---
application to control of the healthy worker survivor effect.
{\it Mathematical Modelling\/} 7: 1393--1512.

\smallskip\noindent
Robins, J.~M. (1987a).
A graphical approach to the identification and estimation of causal parameters
in mortality studies with sustained exposure periods.
{\it Journal of Chronic Diseases\/} 40, Supplement 2: 139S--61.

\smallskip\noindent
Robins, J.~M. (1987b).
Addendum to  ``A new approach to causal inference in mortality studies with a sustained
exposure period---application to control of the healthy worker survivor effect.''
{\it Computers and Mathematics with Applications\/} 14: 923--45.

\smallskip\noindent
Robins, J.~M. (1995).
Discussion.
{\it Biometrika\/} 82: 695--98.

\smallskip\noindent
Robins, J.~M. (1999).
Association, causation, and marginal structural models.
{\it Synthese\/} 121: 151--79.

\smallskip\noindent
Robins, J.~M. and Rotnitzky, A. (1992).
Recovery of information and adjustment for dependent censoring using surrogate markers.
In N. Jewell, K. Dietz, and V. Farewell, eds.
{\it AIDS Epidemiology---Methodological Issues\/}.
Boston, MA: Birkh\"auser, pp.~297--331.

\smallskip\noindent
Robins, J.~M. and Rotnitzky, A. (1995).
Semiparametric efficiency in multivariate regression models with missing data.
{\it Journal of the American Statistical Association\/} 90: 122--29.

\smallskip\noindent
Robins, J.~M., Rotnitzky, A., and Zhao, L.~P. (1994).
Estimation of regression coefficients when some regressors are not always observed.
{\it Journal of the American Statistical Association\/} 89: 846--66.

\smallskip\noindent
Robins, J.~M., Sued, M., Lei-Gomez, Q., and Rotnitzky, A. (2007).
Performance of double-robust estimators when ``inverse probability'' weights are highly variable.
{\it Statistical Science\/} 22: 544--59.

\smallskip\noindent
Robinson, W.~S. (1950).
Ecological correlations and the behavior of individuals.
{\it American Sociological Review\/} 15: 351--57.

\smallskip\noindent
Robinson, L.~D. and Jewell, N.~P. (1991).
Some surprising results about covariate adjustment in logistic regression models.
{\it International Statistical Review\/} {58}: 227--40.

\smallskip\noindent
Roe, D.~A. (1973).
{\it A Plague of Corn\/}.
Ithaca, NY: Cornell University Press.

\smallskip\noindent
Rogowski, R. (2004).
How inference in the social (but not the physical) sciences neglects theoretical anomaly.
In Brady and Collier (2004), pp.~75--82.

\smallskip\noindent
Rosenbaum, P.~R. (2002).
Covariance adjustment in randomized experiments and observational studies.
{\it Statistical Science\/} {17}: 286--327 (with discussion).

\smallskip\noindent
Rosenberg, C.~E. (1962).
{\it The Cholera Years\/}.
Chicago: University of Chicago Press.

\vfill\eject
\smallskip\noindent
Rosenblum, M. and van der Laan, M.~J. (2008).
Using regression models to analyze randomized trials:
Asymptotically valid hypothesis tests despite incorrectly specified models.

\noindent \hskip 10pt
http://www.bepress.com/ucbbiostat/paper219/

\smallskip\noindent
Rosenthal, R. (1979).
The ``file drawer'' and tolerance for null results.
{\it Psychological Bulletin\/} 86: 638--41.

\smallskip\noindent
Rossouw, J.~E., Anderson, G.~L., Prentice, R.~L.~et al.~(2002).
Risks and benefits of estrogen plus progestin in healthy postmenopausal women:
Principal results from the Women's Health Initiative randomized controlled trial.
{\it Journal of the American Medical Association\/} 288: 321--33.

\smallskip\noindent
Rotnitzky, A., Robins, J.~M., and Scharfstein, D.~O. (1998).
Semiparametric regression for repeated outcomes with nonignorable nonresponse.
{\it Journal of the American Statistical Association\/} 93: 1321--39.

\smallskip\noindent
Rubin, D. (1974).
Estimating causal effects of treatments in randomized and nonrandomized studies.
{\it Journal of Educational Psychology\/} 66: 688--701.

\smallskip\noindent
Rudin, W. (1976).
{\it Principles of Mathematical Analysis\/}. 3rd. edn.
New York: McGraw-Hill.

\smallskip\noindent
Sacks, F.~M., Svetkey, L.~P., Vollmer, W.~M.~et al.~(2001).
Effects on blood pressure of reduced dietary sodium and the
dietary approaches to stop hypertension (DASH) diet.
{\it New England Journal of Medicine\/} 344: 3--10.

\smallskip\noindent
Sampson, R.~J., Laub, J.~H., and Wimer, C. (2006).
Does marriage reduce crime? A counterfactual approach to within-individual causal effects.
{\it Criminology} 44: 465--508.

\smallskip\noindent
Savage, L.~J. (1972 [1954]).
{\it  The Foundations of Statistics\/}. 2nd rev.~edn.
New York: Dover Publications.

\smallskip\noindent
Scharfstein, D.~O., Rotnitzky, A., and Robins, J.~M. (1999).
Adjusting for non-ignorable drop-out using semiparametric non-response models.
{\it Journal of the American Statistical Association\/} 94: 1096--1146.

\smallskip\noindent
Scheff\'e, H. (1956).
Alternative models for the analysis of variance.
{\it Annals of Mathematical Statistics\/} 27: 251--71.

{\overfullrule0pt
\smallskip\noindent
Schonlau, M. (2006).
Charging decisions in death-eligible federal cases (1995--2005):
Arbitrariness, capriciousness, and regional variation.
In S.~P. Klein, R.~A. Berk, and L.~J. Hickman, eds.
{\it Race and the Decision to Seek the Death Penalty in Federal Cases\/}.
Technical report \#TR-389-NIJ, RAND Corportation, Santa Monica, CA, pp.~95--124.
\par}

\smallskip\noindent
Semmelweis, I. (1941 [1861]).
{\it Die Aetiologie, der Begriff und die Prophylaxis des Kindbettfiebers\/}.
Pest, Wien und Leipzig, C. A. Hartleben's Ver\-lags-Expe\-di\-tion.
English translation by F.~P. Murphy (1981).
{\it The Etiology, Concept, and Prophylaxis of Child\-bed Fever\/}.
The Classics of Medicine Library, Birmingham, Alabama, vol.~5, pp.~338--775.
Page cites are to the Murphy translation.

\smallskip\noindent
Sen, A.~K. (2002).
{\it Rationality and Freedom\/}.
Cambridge, MA: Harvard University Press.

\smallskip\noindent
Shadish, W.~R., Cook, T.~D., and Campbell, D.~T. (2002).
{\it Experimental and Quasi-Experimental Designs for Generalized Causal Inference.\/}
Boston:\break Houghton Mifflin.

\smallskip\noindent
Shaffer, J.~P. (1995).
Multiple hypothesis testing.
{\it Annual Review of Psychology\/} 46: 561--84.

\smallskip\noindent
Shapiro, S. (1994).
Meta-analysis, shmeta-analysis.
{\it American Journal of Epidemiology\/} 140: 771--91 (with discussion).

\smallskip\noindent
Shapiro, S., Venet, W., Strax, P., and Venet L. (1988).
{\it Periodic Screening for Breast Cancer: The Health Insurance Plan Project and its Sequelae, 1963--1986\/}.
Baltimore, MD: Johns Hopkins University Press.

\smallskip\noindent
Sherman, L.~W., Gottfredson, D., MacKenzie, D.~et al.~(1997).
{\it Preventing Crime: What Works, What Doesn't, What's Promising?}
Washington, D.C.: U.S. Department of Justice.

\smallskip\noindent
Shields, P.~G., Caporaso, N.~E., Falk, K.~T., Sugimura, H.~et al.~(1993).
Lung cancer, race and a CYP1A1 genetic polymorphism.
{\it Cancer Epidemiology, Biomarkers and Prevention\/} 2: 481--85.

\smallskip\noindent
Simon, H. (1957).
{\it Models of Man\/}.
New York: Wiley.

\smallskip\noindent
Sims, C.~A. (1980).
Macroeconomics and reality.
{\it Econometrica\/} 48: 1--47.

\smallskip\noindent
Singer, B. and Marini, M.~M. (1987).
Advancing social research: An essay based on Stanley Lieberson's
{\it Making It Count: The Improvement of Social Theory and Research\/}.
In C. Clogg, ed. {\it Sociological Methodology 1987\/}.
Washington, D.C.: American Sociological Association, pp.~373--91.

\smallskip\noindent
Skerry, P. (1995).
{\it Mexican Americans: The Ambivalent Minority}.
Cambridge, MA: Harvard University Press.

\smallskip\noindent
Skerry, P. (2000).
{\it Counting on the Census? Race, Group Identity, and the Evasion of Politics\/}.
Washington, D.C.: Brookings Institution Press.

\smallskip\noindent
Smith, G.~D. and Phillips, A.~N. (1996).
Inflation in epidemiology: ``The proof and measurement between two things'' revisited.
{\it British Journal of Medicine\/} 312: 1659--63.

\smallskip\noindent
Smith, W.~C., Crombie, I.~K., Tavendale, R.~T.~et al.~(1988).
Urinary electrolyte excretion, alcohol consumption, and blood pressure in the Scottish heart health study.
{\it British Journal of Medicine\/} 297: 329--30.

\smallskip\noindent
Snow, J. (1965 [1855]).
{\it On the Mode of Communication of Cholera\/}. 2nd edn.
London: Churchill.
Reprinted as part of {\it Snow on Cholera\/} in 1965 by Hafner, New York.
Page cites are to the 1965 edition.

\smallskip\noindent
Sobel, M.~E. (1998).
Causal inference in statistical models of the process of socioeconomic achievement---A case study.
{\it Sociological Methods \& Research\/} 27: 318--48.

\smallskip\noindent
Sobel, M.~E. (2000).
Causal inference in the social sciences.
{\it Journal of the American Statistical Association\/} 95: 647--51.

\smallskip\noindent
Spirtes, P., Glymour, C., and Scheines, R. (1993).
{\it Causation, Prediction, and Search\/}.
New York: Springer. 2nd edn., Cambridge, MA: MIT Press (2000).

\smallskip\noindent
Spirtes, P., Scheines, R., Glymour, C., and Meek, C. (1993).
TETRAD II. Documentation for Version 2.2.
Technical report, Department of Philosophy, Carnegie Mellon University,
Pittsburgh, PA.

\smallskip\noindent
Stamler, J. (1997).
The Intersalt study: Background, methods, findings, and implications.
{\it American Journal of Clinical Nutrition\/} 65, Supplement: S626--42.

\smallskip\noindent
Stamler, J., Elliott, P., Dyer, A.~R.~et al.~(1996).
Commentary: Sodium and blood pressure in the Intersalt study and other studies---in reply to the Salt Institute.
{\it British Journal of Medicine\/} 312: 1285--87.

\smallskip\noindent
Stark, P.~B. (2001).
Review of {\it Who Counts?\/}
{\it Journal of Economic Literature\/} 39:~592--95.

\smallskip
\noindent
Stata (2005).
{\it Stata Base Reference Manual\/}.
Stata Statistical Software. Release~9. Vol.~1.
College Station, TX: StataCorp LP.

\smallskip\noindent
Steiger, J.~H. (2001).
Driving fast in reverse.
{\it Journal of the American Statistical Association\/} 96: 331--38.

\smallskip\noindent
Stigler, S.~M. (1986).
{\it The History of Statistics\/}.
Cambridge, MA: Harvard University Press.

\smallskip\noindent
Stolzenberg, R.~M. and Relles, D.~A. (1990).
Theory testing in a world of constrained research design.
{\it Sociological Methods and Research\/} 18: 395--415.

\smallskip\noindent
Stone, R. (1993).
The assumptions on which causal inferences rest.
{\it Journal of the Royal Statistical Society\/}, Series B, 55: 455--66.

\smallskip\noindent
Stoto, M.~A. (1998).
A solution to the ecological inference problem: Reconstructing individual behavior from aggregate data.
{\it Public Health Reports\/} 113: 182--83.

\smallskip\noindent
Svetkey, L.~P., Sacks, F.~M., Obarzanek, E.~et al.~(1999).
The DASH diet, sodium intake and blood pressure trial (DASH-sodium): Rationale and design.
{\it Journal of the American Dietetic Association\/}
99, Supplement: 96--104.

\smallskip\noindent
Swales J. (2000).
Population advice on salt restriction: The social issues.
{\it American Journal of Hypertension\/} 13: 2--7.

\smallskip\noindent
Taleb, N.~T. (2007).
{\it The Black Swan\/}.
Random House.

\smallskip\noindent
Tauber, S. (1963).
On multinomial coefficients.
{\it American Mathematical\break Monthly\/} {70}: 1058--63.

\smallskip\noindent
Taubes, G. (1998).
The (political) science of salt.
{\it Science\/} 281(5379): 898--907.

\smallskip\noindent
Taubes, G. (2000).
A DASH of data in the salt debate.
{\it Science\/} 288: 1319.

\smallskip\noindent
Temple, R. (1999).
Are surrogate markers adequate to assess cardiovascular disease drugs?
{\it Journal of the American Medical Association\/} 282: 790--95.

\smallskip\noindent
Terris, M., ed. (1964).
{\it Goldberger on Pellagra\/}.
Baton Rouge, LA: Louisiana State University Press.

\smallskip\noindent
Thi\'ebaut, A.~C.~M. and B\'enichou, J. (2004).
Choice of time-scale in Cox's model analysis of epidemiologic cohort data: A simulation study.
{\it Statistics in Medicine\/} 23: 3803--20.

\smallskip\noindent
Timberlake, M. and  Williams, K. (1984).
Dependence, political exclusion and government repression: Some cross national evidence.
{\it American Sociological Review\/} 49: 141--46.

\smallskip\noindent
Timonius, E. and Woodward, J. (1714).
An account, or history, of the procuring the small pox by incision, or inoculation;
as it has for some time been practised at Constantinople.
{\it Philosophical Transactions\/} 29: 72--82.

\smallskip\noindent
Tinbergen, J. (1940).
Reply to Keynes.
{\it The Economic Journal\/} 50: 141--54.

\smallskip\noindent
Tita, G. and Ridgeway, G. (2007).
The impact of gang formation on local pattern of crime.
{\it Journal of Research on Crime and Delinquency\/} 44: 208--37.

\smallskip\noindent
Tong, Y.~L. (1980).
{\it Probability Inequalities in Multivariate Distributions\/}.
New York: Academic Press.

\smallskip\noindent
Tropfke, J. (1903).
{\it Geschichte der Elementar-mathematik in systematischer Dar\-stellung\/}.
Leipzig: Verlag Von Veit \& Comp.

\smallskip\noindent
Truett, J., Cornfield, J., and Kannel, W. (1967).
A multivariate analysis of the risk of coronary heart disease in Framingham.
{\it Journal of Chronic Diseases\/} {20}: 511--24.

\smallskip\noindent
Tsiatis, A. (1975).
A nonidentifiability aspect of the problem of competing risks.
{\it Proceedings of the National Academy of Sciences\/}, {\it USA\/} 72: 20--22.

\smallskip\noindent
U.S.~Census Bureau (2001a).
{\it Report of the Executive Steering Committee for Accuracy and Coverage Evaluation Policy\/}.
With supporting documentation, Reports B1--24. Washington, D.C.

\noindent\hskip 10pt
http://www.census.gov/dmd/www/EscapRep.html

\smallskip\noindent
U.S.~Census Bureau (2001b).
{\it Report of the Executive Steering Committee for Accuracy and Coverage Evaluation
Policy on Adjustment for Non-Redistrict\-ing Uses\/}.
With supporting documentation, Reports 1--24. Washington, D.C.

\noindent\hskip 10pt
http://www.census.gov/dmd/www/EscapRep2.html

\smallskip\noindent
U.S.~Census Bureau (2003).
{\it Technical Assessment of A.C.E.~Revision II\/}.

\noindent\hskip 10pt
http://www.census.gov/dmd/www/ace2.html

\smallskip\noindent
U.S.~Department of Commerce (1991).
Office of the Secretary.
{\it Decision on Whether or Not a Statistical Adjustment of the 1990 Decennial Census of Population
Should Be Made for Coverage Deficiencies Resulting in an Overcount or Undercount of the Population, Explanation\/}.
Three volumes, Washington, D.C.
Reprinted in part in {\it Federal Register\/} 56: 33, 582--33, 642 (July 22).

\smallskip\noindent
U.S.~Department of Health and Human Services (1990).
{\it The Health Benefits of Smoking Cessation: A Report of the Surgeon General\/}.
Washington, D.C.

\smallskip\noindent
U.S.~Geological Survey. (1999).
Working group on California earthquake probabilities.
Earthquake probabilities in the San Francisco Bay Region: 2000-2030--A Summary of findings.
Technical Report Open-File Report 99-517, USGS, Menlo Park, CA.

\smallskip\noindent
U.S.~Preventive Services Task Force (1996).
{\it Guide to Clinical Preventive Services.\/} 2nd edn.
Baltimore, MD: Williams \& Wilkins.

\smallskip\noindent
U.S.~Public Health Service. (1964).
{\it Smoking and Health. Report of the Advisory Committee to the Surgeon General\/}.
Washington, D.C.: U.S.~Government Printing Office.

\smallskip\noindent
van der Vaart, A. (1998).
{\it Asymptotic Statistics\/}.
Cambridge: Cambridge University Press.

\smallskip\noindent
Van de Ven, W.~P.~M.~M. and Van Praag, B.~M.~S. (1981).
The demand for deductibles in private health insurance:
A probit model with sample selection.
{\it Journal of Econometrics\/} 17: 229--52.

\smallskip\noindent
Verhulst, P.~F. (1845).  %Pierre Francois
Recherches math\'ematiques sur la loi d'accroisse\-ment de la population.
{\it Nouveaux m\'emoires de l'Acad\'emie Royale des Sciences et Belles-Lettres de Bruxelles\/} {18}: 1--38.

\smallskip\noindent
Verma, T. and Pearl, J. (1990).
Causal networks:  Semantics and expressiveness.
In R. Shachter, T.~S. Levitt, and L.~N. Kanal, eds.
{\it Uncertainty in AI 4\/}.
Elsevier Science Publishers, pp.~69--76.

\smallskip\noindent
Victora, C.~G., Habicht, J.~P., and Bryce, J. (2004).
Evidence-based public health: Moving beyond randomized trials.
{\it American Journal of Public Health\/} 94: 400--405.

\smallskip\noindent
Vinten-Johansen, P., Brody, H., Paneth, N., and Rachman, S. (2003).
{\it Chol\-era, Chloroform, and the Science of Medicine\/}.
New York: Oxford University Press.

\smallskip\noindent
von Mises, R. (1964).
{\it  Mathematical Theory of Probability and Statistics\/}.
H.~Geiringer, ed.
New York: Academic Press.

\smallskip\noindent
von Neumann, J. and Morgenstern, O. (1944).
{\it Theory of Games and Economic Behavior\/}.
Princeton, NJ: Princeton University Press.

\smallskip\noindent
Wachter, K.~W. and Freedman, D.~A. (2000).
The fifth cell.
{\it Evaluation Review\/} 24: 191--211.

\smallskip\noindent
Wainer, H. (1989).
Eelworms, bullet holes, and Geraldine Ferraro:
Some problems with statistical adjustment and some solutions.
{\it Journal of Educational Statistics\/} 14: 121--40 (with discussion).
Reprinted in J. Shaffer, ed. (1992).
{\it The Role of Models in Nonexperimental Social Science\/}.
Washington, D.C.: AERA/ASA, pp.~129--207.

\smallskip
\noindent
Wald, A. (1940).
The fitting of straight lines if both variables are subject to error.
{\it The Annals of Mathematical Statistics\/} 11: 284--300.

\smallskip
\noindent
Wald, A. and Wolfowitz, J. (1950).
Bayes solutions of sequential decision problems.
{\it The Annals of Mathematical Statistics\/} 21: 82--99.

\smallskip\noindent
Walsh, C. (2003).
{\it Antibiotics: Actions, Origins, Resistance\/}.
Washington, D.C.: ASM Press.

\smallskip\noindent
Webster, W.~S. (1998).
Teratogen update: Congenital rubella.
{\it Teratology\/} 58: 13--23.


\smallskip\noindent
Weisberg, S. (1985)
{\it Applied Linear Regression\/}.
New York: Wiley.

\smallskip\noindent
Welch, H.~G., Woloshin, S., Schwartz, L.~M.~et al.~(2007).
Overstating the evidence for lung cancer screening:
The International Early Lung Cancer Action Program (I-ELCAP) study.
{\it Archives of Internal Medicine\/} 167: 2289--95.

\smallskip\noindent
White, H. (1980).
A heteroskedasticity-consistent covariance matrix estimator and a direct test for heteroskedasticity.
{\it Econometrica\/} 48: 817--38.

\smallskip\noindent
White, H. (1994).
{\it Estimation, Inference, and Specification Analysis\/}.
Cambridge: Cambridge University Press.

\smallskip\noindent
White, M.~D. (2000).
Assessing the impact of administrative policy on the use of deadly force
by on- and off-duty police.
{\it Evaluation Review} 24(3): 295--318.

\smallskip\noindent
Wilde, E.~T. and Hollister, R. (2007).
How close is close enough? Evaluating propensity score matching using
data from a class size reduction experiment.
{\it Journal of Policy Analysis and Management\/} 26: 455--77.

\smallskip\noindent
Winship, C. and Mare, R.~D. (1992).
Models for sample selection bias.
{\it Annual Review of Sociology\/} 18: 327--50.

\smallskip\noindent
Woodward, J. (1997).
Causal models, probabilities, and invariance.
In V. McKim and S. Turner, eds. {\it Causality in Crisis?\/}
Notre Dame, IN: University of Notre Dame Press, pp.~265--315.

\smallskip\noindent
Woodward, J. (1999).
Causal interpretation in systems of equations.
{\it Synthese\/} 121: 199--247.

\smallskip\noindent
Wright, P.~G. (1928).
{\it The Tariff on Animal and Vegetable Oils\/}.
New York: MacMillan.

\smallskip\noindent
Wright, S. (1921).
Correlation and causation.
{\it Journal of
Agricultural Research\/} 20: 557--85.

\smallskip\noindent
Yee, T.~W. (2007).
The VGAM Package.

\noindent\hskip 10pt
http://www.stat.auckland.ac.nz/\string~yee/VGAM

\smallskip\noindent
Ylvisaker, D. (2001).
Review of {\it Who Counts?\/}
{\it Journal of the American Statistical Association\/} 96: 340--41.

\smallskip\noindent
Yule, G.~U. (1899).
An investigation into the causes of changes in pauperism in England,
chiefly during the last two intercensal decades.
{\it Journal of the Royal Statistical Society\/} 62: 249--95.

\smallskip\noindent
Yule, G.~U. (1925).
The growth of population and the factors which control it.
{\it Journal of the Royal Statistical Society\/} {88}: 1--62 (with discussion).

\smallskip\noindent
Zaslavsky, A.~M. (1993).
Combining census, dual system, and evaluation study data to estimate population shares.
{\it Journal of the American Statistical Association\/} 88: 1092--1105.



\bye

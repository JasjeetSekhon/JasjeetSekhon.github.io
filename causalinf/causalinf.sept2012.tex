% -*-Mode: LaTeX;-*-
% %W%   %G%
% 
%

\documentclass[fleqn,11pt]{article}
\usepackage{amsmath,amsfonts,amssymb,graphicx,psfrag,rotating}
\usepackage{hyperref,setspace} \usepackage{threeparttable}
\usepackage{dcolumn} \usepackage[longnamesfirst]{natbib}
\usepackage{comment}
\usepackage{epsfig,rotate}
\usepackage[mdyyyy]{datetime}

\bibliographystyle{info} 
\bibpunct{(}{)}{;}{author-year}{}{,}

\newcommand{\MyPerp}{\perp \! \! \! \perp}

\setlength{\oddsidemargin}{0in}
\setlength{\evensidemargin}{0in}
\setlength{\topmargin}{-.26in}%{0in}
%\setlength{\topmargin}{0.05in}
\setlength{\headheight}{0in}
\setlength{\headsep}{0in}
\setlength{\textwidth}{6.5in}
%\setlength{\textheight}{8.987in}%{8.9in}
\setlength{\textheight}{9in}

\hypersetup{
%  colorlinks=true,
  bookmarksopen, pdftitle={}, pdfsubject={}, pdfauthor={Jasjeet S. Sekhon},
%  linkcolor=cyan,               
%  citecolor=cyan,               
%  pagecolor=cyan,
%  urlcolor=cyan
}

%\doublespacing

% change section to large and subsection to normalsize font
\makeatletter
\renewcommand{\section}{\@startsection{section}{1}{0em}{\baselineskip}{0.5\baselineskip}{\large\bfseries\large}}
\renewcommand{\subsection}{\@startsection{subsection}{0}{0em}{\baselineskip}{0.5\baselineskip}{\normalfont\bfseries\normalsize}}
\makeatletter
\newcommand{\vs}{\vspace{-\baselineskip}}
\newcommand{\vvs}{\vspace{-.4cm}}

\newcommand{\mr}{\mathrm}

%dcolumn column types
\newcolumntype{.}{D{.}{.}{-1}}
\newcolumntype{d}[1]{D{.}{.}{#1}}

\newcommand{\Rsymb}{{\bf{\textsf{R}}}}

\newcommand{\hlink}{\htmladdnormallink}

\title{The Statistics of Causal Inference in the Social Sciences\\
  Political Science C236A\\
  Statistics C239A}
\author{Professor Jasjeet Singh Sekhon}
\date{Class: 2-4pm Tuesday\\220 Wheeler}

\begin{document}
\maketitle

\begin{quote}
  \textbf{Professor Jasjeet Singh Sekhon} \\
  \hlink{\texttt{sekhon@berkeley.edu}}%
  {mailto:sekhon@berkeley.edu} \\
  \hlink{\texttt{HTTP://sekhon.berkeley.edu}}%
  {http://sekhon.berkeley.edu} \\
  Office: Barrows Hall 750C\\
\end{quote}

\begin{quote}
  \textbf{John Henderson, GSI}\\
  \hlink{\texttt{jahenderson@berkeley.edu}}{jahenderson@berkeley.edu} \\
  Section: Thursday 6--8 \\
\end{quote} 

\subsection*{Description}
\pdfbookmark[1]{Description}{sec:description}

Approaches to causal inference using the potential outcomes framework.
Covers observational studies with and without ignorable treatment
assignment, randomized experiments with and without noncompliance,
instrumental variables, regression discontinuity, sensitivity analysis
and randomization inference. Applications are drawn from a variety of
fields including political science, economics, sociology, public
health and medicine.  \\ 

This course can be used to meet the Department's Methodology course-out
option. 

\subsection*{Prerequisites}

At least one graduate matrix based multivariate regression course in
addition to introductory statistics and probability.  If you need to
review material, please consult David Freedman's excellent
\textit{Statistical Models: Theory and Practice} or John Fox's
\textit{Applied Regression Analysis, Linear Models, and Related
  Methods}, which includes examples using the
\hlink{\textit{R}}{http://www.r-project.org/} programming language.

\begin{comment}
You may also want to review my lecture notes for an introductory
course (Government 1000 at Harvard).  Link to \hlink{[presentation
  version of
  notes]}{http://sekhon.berkeley.edu/gov1000/g1000_presentation.pdf},
\hlink{[printer version of
  notes]}{http://sekhon.berkeley.edu/gov1000/g1000_printing.pdf}.
\end{comment}

\section*{Evaluation}
\pdfbookmark[1]{Evaluation}{sec:evaluation}

Final grades will be based on a series of homework assignments (30\%
of final grade), a midterm (30\%), a term paper or final exam (30\%),
and class and section participation (10\%).  Student have the choice
between a term paper and a final exam.

It is recommended that students write the term paper jointly with one
or at most two other students.  Experience has shown that this greatly
facilitates learning as well as increases the likelihood that the
paper will eventually become a published article.

Weekly readings and homework assignments are the norm.  It is highly
recommended that students form study groups in order to complete the
homework assignments.  Although it is recommended that people work
together in order to complete the assignments, students must hand in
their own individual answers.  Photocopies and other reproductions of
someone else's answers are not acceptable.  Students should hand in
the answers to the problem sets, and all computer code written to find
those answers.

During exams, students are not allowed to communicate or cooperate
with anyone in any way about exam. Any questions should be asked
directly to me and the GSI. To repeat: for exams, one is not allowed
to use study groups, online help forms, the writing center, or any
other form of help aside from those that are explicitly allowed on the
exam instructions. If in doubt, ask.

Incompletes: all course material must be handed in by the first day of
class of the spring semester unless an exemption is explicitly
granted.

\section*{Course Software and Books}
\pdfbookmark[1]{Course Books and Software}{sec:books} 

The programming language for this course is the
\hlink{\textit{R}}{http://www.r-project.org/} variant of the
\textit{S} statistical programming language.  It is available for
download from:
\hlink{\texttt{http://www.r-project.org/}}{http://www.r-project.org/}.
\hlink{\textit{R}}{http://www.r-project.org/} is open source software
(released under the \hlink{GNU public license}{http://www.gnu.org/})
and is available at no charge.  We will also be making extensive use
of an R package called
\hlink{``Matching''}{http://sekhon.berkeley.edu/matching}
\citep{sekhonJSS}.

The books listed below are required and available at various online
bookstores and at the University Book Store.

\begin{itemize}
\item Rubin, Donald.  2006.  \textit{Matched Sampling for Causal
    Effects}.  Cambridge University Press.  ISBN 0521674360.

\item Rosenbaum, Paul R.  2002.  \textit{Observational Studies}.
  Springer-Verlag.  2nd edition.  ISBN 0387989676.
  
\item Morgan, Stephen L.\ and Christopher Winship. 2007.
  \textit{Counterfactuals and Causal Inference: Methods and Principles
    for Social Research}.  Cambridge University Press. ISBN-10:
  0521671930.

\item Krause, Andreas and Melvin Olson. 2005. \textit{The Basics of
    S-PLUS}. Springer. ISBN-10: 0387261095.
\end{itemize}

In addition to the required books, you may wish to obtain a copy of:
\begin{itemize}
\item Freedman, David A. 2010. \textit{Statistical Models and Causal
    Inference: A Dialogue with the Social Sciences}. David Collier,
  Jasjeet S.\ Sekhon, and Philip B.\ Stark, Editors. Cambridge
  University Press. ISBN-10 0521123909.\\ [1.25ex]
  This volume contains a collection of Freedman's articles of special
  interest to social scientists.

\item Angrist, Joshua D. and J\"{o}rn-Steffen
  Pischke. 2008. \textit{Mostly Harmless Econometrics: An Empiricist's
    Companion}. Princeton University Press.

\item Rosenbaum, Paul R.  2009.  \textit{Design of Observational
    Studies}.  Springer-Verlag.  ISBN-10 1441912126. \\ [1.25ex]
  This book is less technical than Rosenbaum's \textit{Observational
    Studies}.  The book is also available online via Cal's contract
  with Springer-Verlag: \url{http://www.springerlink.com/content/978-1-4419-1212-1/#section=627451&page=1}

\item Venables, W.N and Brian D.~Ripley.  2003.  \textit{Modern
    Applied Statistics with S}.  New York: Springer-Verlag. 4th
  edition.  ISBN: 0387954570 \\ [1.25ex]
  A reference manual on the implementation of many statistical
  techniques in \textit{R} and \textit{S}.
\end{itemize}

\begin{comment}
  Rosenbaum (2002) is a useful book, and provides a treatment of the
  concepts with a uniform notation.  The 2nd edition is available at
  \href{http://www.amazon.com/gp/product/0387989676/qid=1136502162/sr=2-1/ref=pd_bbs_b_2_1/002-5192299-7212830?s=books&v=glance&n=283155}{[
    amazon.com ] }\\
\end{comment}


\subsection*{Course outline}
\begin{enumerate}

\item \textsc{Causality} 

  {\em The potential outcomes framework for causal inference.
    }

\begin{itemize}
  \item \cite{holland1986}
  \href{http://links.jstor.org/sici?sici=0162-1459%28198612%2981%3A396%3C%3E1.0.CO%3B2-9}{``Statistics and Causal Inference''}

  \item \cite{LittleRubin2000}
  \href{http://arjournals.annualreviews.org/doi/abs/10.1146%2Fannurev.publhealth.21.1.121}{``Causal Effects in Clinical and Epidemiological Studies via Potential Outcomes''}

  \item \cite{Sekhon2004}:
    \hlink{``Quality Meets Quantity: Case Studies, Conditional Probability and Counterfactuals''}{http://sekhon.berkeley.edu/papers/QualityQuantity.pdf}

%   \item \cite{cox:jrss1992}
%   \href{http://links.jstor.org/sici?sici=0964-1998%281992%29155%3A2%3C291%3ACSSA%3E2.0.CO%3B2-O}{[ JSTOR ]}

%   \item \cite{freedman:1991}
%   \href{http://links.jstor.org/sici?sici=0081-1750%281991%2921%3C%3E1.0.CO%3B2-H}{[ JSTOR ToC ]}
% 
%   {\em With discussion by Berk, Blalock, and Mason, and rejoinder pp. 315-358 }

\end{itemize}

Extra reading:
\begin{itemize}
  \item \cite{WinshipMorgan1999}
  \href{http://arjournals.annualreviews.org/doi/abs/10.1146%2Fannurev.soc.25.1.659}{``The Estimation of Causal Effects from Observational Data''}
\end{itemize}

\item \textsc{Statistical Modeling: Foundations and Limitations} 
  \begin{itemize}
  \item Freedman Chapter 1 \citep{freedman_foundations}: \href{http://www.stat.berkeley.edu/~census/fos.pdf}{``Some Issues in the Foundations of Statistics: Probability and Model Validation.''}

 \item Freedman Chapter 2 \citep{freedmanCh2}: \href{http://www.stat.berkeley.edu/~census/berk2.pdf}{``Statistical
   Assumptions as Empirical Commitments.''}

 \item Freedman Chapter 3 \citep{freedman_shoe}: ``Statistical Models
   and Shoe Leather.''
 \end{itemize}
 For extra readings see the rest of the Freedman volume, especially:
\begin{itemize}
\item Freedman Chapter 20 \citep{freedmanCh20}: ``On Types of Scientific Inquiry: The Role of Qualitative Reasoning.''
\item Freedman Chapter 14 \citep{freedmanCh14}: ``The Grand Leap'' (of graphical models).
\item Freedman Chapter 15 \citep{freedman_graph}: ``On Specifying Graphical Models for
  Causation, and the Identification Problem.'' 
\end{itemize}



\item \textsc{Randomized Experiments and Controlling Bias in Observational Studies} 

  {\em Properties of experiments, basic implementations, and illustrations of observational studies based on     approximate experimental design. }

 \begin{itemize}

% \item Fisher

 \item \cite{neyman1923}: ``On the Application of Probability Theory to
   Agricultural Experiments. Essay on Principles. Section 9.''
   \textit{Statistical Science} 5, 465--472.

 \item \cite{rubin1990} ``Comment: Neyman (1923) and Causal Inference in Experiments
   and Observational Studies,'' \textit{Statistical Science} 5, 472-480.
   
 \item \citet{rubin2006} Chapters 1 and 2:\\
   ``William G. Cochran's Contributions to the Design, Analysis and
   Evaluation of Observational Studies'' \\    
   \cite{CochranRubin1973}: ``Controlling Bias in Observational Studies: A Review''

 \item \cite{rosenbaum2002} Chapter 2
 \end{itemize}

 Extra readings:

 \begin{itemize}
 \item \cite{przeworski:science} \href{http://www.nyu.edu/gsas/dept/politics/faculty/przeworski/papers/isthescience.pdf#search=%22is%20the%20science%20of%20comparative%20politics%20possible%22}{``Is the Science of Comparative Politics Possible?''
}

 \item \cite{cox1958}: {\em Planning of Experiments}. Chapters 1 and 2.
   % \href{http://www.stanford.edu/class/polisci355/classonly/cox.pdf}{[pdf]}

 \item \cite{cochran1965}: ``The Planning of Observational Studies of
   Human Populations''
   
 \item \cite{cochran1983}: Chapters 1 and 7
   %\href{http://www.stanford.edu/class/polisci355/classonly/cochran.pdf}{ [ pdf ] }
   
   % \cite{cox.reid:2000}. Chapters 1 and 2
 \end{itemize}

\item \textsc{Randomization Inference} 

  {\em Fisherian and permutation Inference, and the Lady Tasting Tea}

  \begin{itemize}
  \item \citet[][ch 1--2]{fisher1935}:
    \href{http://tinyurl.com/c9tj2hy}{\textit{Design of
        Experiments}}. \url{http://tinyurl.com/c9tj2hy}

  \item \citet[][ch 2]{rosenbaum2002}: \textit{Observational Studies}
  \item \cite{rosenbaum.ri}: ``Covariance adjustment in randomized
    experiments and observational studies.'' \textit{Statistical
      Science} 17 286--327 (with discussion).

\end{itemize}

 Extra reading:
  \begin{itemize}
 \item Freedman Chapter 8: ``What is the Chance of an Earthquake?''

 \item \citet{BowersPanagopoulos2011}:
   \href{http://www.jakebowers.org/PAPERS/BowPan-Fisher.pdf}{``Fisher's
     Randomization Mode of Statistical Inference, Then and Now.''}

  \item Attributable effects: \citet[][188--194]{rosenbaum2002}.

  \item \citet{ptiman1937a}: ``Significance Tests Which May be Applied
    to Samples From any Populations''

  \item \citet{pitman1937b}: ``Significance Tests Which May be Applied
    to Samples from any Populations. II. The Correlation Coefficient
    Test''

  \item \citet{pitman1938}: ``Significance Tests which can be Applied
    to Samples from any Populations. III. The Analysis of Variance
    Test''

  \item 
  \end{itemize}


\item \textsc{Univariate Matching Methods for Controlling Bias in Observational Studies} 
  
  {\em Experimental and observational studies where
    assignment to treatment is done on observables. Stratification and
    matching.}
  
  \begin{itemize}
  \item \citet[]{rubin2006} Chapters 3 to 5:\\
    ``Matching to Remove Bias in Observational Studies'' \cite{rubin1973a}\\
    ``The Use of Matched Sampling and Regression Adjustment to Remove Bias in Observational
    Studies'' \cite{rubin1973b}\\
  ``Assignment to a Treatment Group on the Basis of a Covariate''
    
  \item \citet{rosenbaum2002} Chapter 3.1--3.3
  \end{itemize}
  
  %Extra readings:
  %\begin{itemize}
    % \item  \cite{rubin:1991}
    %   \href{http://links.jstor.org/sici?sici=0006-341X%28199112%2947%3A4%3C1213%3APIOMOS%3E2.0.CO%3B2-H}{ [ JSTOR ] }
    
  %\item Cox, D.~R.. 1958. {\em Planning of Experiments}. New York: Wiley. Chapters 3 and 4.
  %  \href{http://www.stanford.edu/class/polisci355/classonly/cox3-4.pdf}{ [pdf]}
  %\end{itemize}

\item \textsc{The Propensity Score} 

 {\em Logistic regression and the fundamentals of propensity score matching}
  \begin{itemize}
    \item Handout on general linear models
    \item \citet{rubin2006} Chapters 10, 11 and 14 all with Paul R. Rosenbaum:\\
      ``The Central Role of the Propensity Score in Observational Studies'' \cite{RosenbaumRubin1983} \\
      ``Assessing Sensitivity to an Unobserved binary Covariate in an Observational Study with Binary Outcome'' \\
      ``The Bias Due to Incomplete Matching''
    \item \citet{SekhonInformation}: \hlink{The Varying Role of Voter Information Across Democratic Societies}{http://sekhon.berkeley.edu/papers/SekhonInformation.pdf}
    \item \citet{MorganHarding2006}: ``\hlink{Matching Estimators of
        Causal Effects: Prospects and Pitfalls in Theory and
        Practice}{http://smr.sagepub.com/cgi/content/abstract/35/1/3}''
  \end{itemize}
  Also see \cite{RosenbaumRubin1984,RubinThomas2000}.

\item \textsc{Regression Discontinuity Design} 
  \begin{itemize}
  \item \citet{ThistlethwaiteCampbell1960}: ``Regression-Discontinuity Analysis: An alternative to the ex post facto experiment''

  \item \citet{leeRD}: ``Randomized Experiments from Non-random Selection in U.S. House Elections''

\item \citet{CaugheySekhon2010}: ``Elections and the   Regression-Discontinuity Design: Lessons from Close U.S. House   Races, 1942--2008''

  \item \citet{HahnToddvanderKlaauw}: ``Identification and Estimation of Treatment Effects with a Regression-Discontinuity Design''
  \end{itemize}


  Extra reading:
  \begin{itemize}
  \item \cite{dunning2008}: ``Improving Causal Inference: Strengths
    and Limitations of Natural Experiments.'' \textit{Political Science Quarterly} 61(2):282--293 2008.
  \end{itemize}

\item \textsc{Multivariate Matching} 

 {\em Mahalanobis distance, Genetic Matching and Equal Percent Bias Reduction}

  \begin{itemize}
  \item \cite{rubin2006} Chapters 8 and 9:\\
    ``Bias Reduction Using Mahalanobis-Metric Matching'' \cite{rubin1980}\\
    ``Using Multivariate Matched Sampling and Regression Adjustment to Control Bias in Observational Studies'' \cite{rubin1979}
  \item \citet{DiamondSekhon2005}: \hlink{Genetic Matching for Estimating Causal Effects: A General Multivariate Matching Method for Achieving Balance in Observational Studies}{http://sekhon.berkeley.edu/papers/GenMatch.pdf}
  \end{itemize}

\item \textsc{Genetic Matching}

 {\em Automatic balance optimization, evaluating balance and the LaLonde controversy}
  \begin{itemize}
 \item \citet{DiamondSekhon2005}: \hlink{``Genetic Matching for
     Estimating Causal Effects}{http://sekhon.berkeley.edu/papers/GenMatch.pdf''}

 \item \citet{SekhonGrieve_pac}: ``A Matching Method for Improving
   Covariate Balance in Cost-Effectiveness Analyses.''

\item \citet{lalonde1986}
  \href{http://links.jstor.org/sici?sici=0002-8282%28198609%2976%3A4%3C604%3AETEEOT%3E2.0.CO%3B2-P}{ [ JSTOR ] }

  \item \citet{DehejiaWahba1999}
    \href{http://links.jstor.org/sici?sici=0162-1459%28199912%2994%3A448%3C1053%3ACEINSR%3E2.0.CO%3B2-K}{ [ JSTOR ] }

  \item \citet{SmithTodd2001}
\end{itemize}

\item \textsc{Natural Experiments} 
  \begin{itemize}
  \item \citet{SekhonTitiunik_APSR}: ``When Natural Experiments Are
    Neither Natural Nor Experiments''
  \end{itemize}

\item \textsc{Matching Examples Using Observational Data} \\
  Please read the first three of the Political Science examples listed
  here and any of the others you find of interest.  An effort has been
  made to
  obtain examples across fields which are pedagogically interesting.\\
  \textit{Political Science}
  \begin{itemize}
  \item \citeauthor*{GordonHuber2007}: \href{http://sekhon.berkeley.edu/causalinf/papers/faithful5.0s.pdf}{``The Effect of Electoral Competitiveness on Incumbent Behavior''}
  \item \citeauthor*{GilliganSergenti2006}:
    \href{http://sekhon.berkeley.edu/causalinf/papers/gilligan_sergenti_06.pdf}{``Evaluating
      UN Peacekeeping with Matching to Improve Causal Inference''}
  \item \citeauthor*{LenzLadd2006}:
    \href{http://sekhon.berkeley.edu/causalinf/papers/LaddLenzBritish.pdf}{``Exploiting a Rare Shift in Communication Flows: Media Effects in the 1997 British Election''}
  \item \cite{SimmonsHopkins2005}: \href{http://sekhon.berkeley.edu/causalinf/papers/treaties.pdf}{``The Constraining Power of International Treaties: Theory and Methods"}
  \end{itemize}

  \textit{Economics}
  \begin{itemize}
  \item \cite{GalianiGertlerSchargrodsky2005}: \href{http://www.journals.uchicago.edu/JPE/journal/issues/v113n1/113106/113106.web.pdf}{``Water for Life: The
    Impact of the Privatization of Water Services on Child Mortality''}
  \item \cite{ImbensRubinSacerdote2001}: \href{http://links.jstor.org/sici?sici=0002-8282%28200109%2991%3A4%3C778%3AETEOUI%3E2.0.CO%3B2-7}{``Estimating the Effect of
    Unearned Income on Labor Earnings, Savings, and Consumption:
    Evidence from a Survey of Lottery Players''}
  \item \cite{angrist1998}: \href{http://links.jstor.org/sici?sici=0012-9682%28199803%2966%3A2%3C249%3AETLMIO%3E2.0.CO%3B2-A}{``Estimating the Labor Market Impact of
    Voluntary Military Service Using Social Security Data on Military}
    Applicants.''
  \end{itemize}

  \textit{Other}
  \begin{itemize}
  \item \cite{ChristakisIwashyna2003}: ``The Health Impact of Health
    Care on Families: A matched cohort study of hospice use by
    decedents and mortality outcomes in surviving, widowed spouses''
  \item \cite{rubin2001}: ``Using Propensity Scores to Help Design Observational Studies: Application to the Tobacco Litigation''
  \end{itemize}


%Using Matching, but not GenMatch
%  \textit{Additional Applications}
%  \begin{itemize}
%  \item \citeauthor*{Eggers2006}:
%    \href{http://sekhon.berkeley.edu/causalinf/papers/rtaffects5.pdf}{``How Much Do Trade Agreements Increase Trade''}
%  \end{itemize}

\item \textsc{Instrumental Variables (IV)} 
  \begin{itemize}

  \item \cite{AngristKrueger2001}: \hlink{``Instrumental Variables and the Search for Identification: From Supply and Demand to Natural Experiments''}{http://www.irs.princeton.edu/pubs/pdfs/455jep.pdf}

  \item \cite{air1996}
  \href{http://links.jstor.org/sici?sici=0162-1459%28199606%2991%3A434%3C%3E1.0.CO%3B2-D }{``Identification of Causal Effects Using Instrumental Variables''}

% Comment on James J. Heckman, "Instrumental Variables: A Study of Implicit Behavioral Assumptions Used in Making Program Evaluations"
% Joshua D. Angrist; Guido W. Imbens
% The Journal of Human Resources > Vol. 34, No. 4 (Autumn, 1999), pp. 823-827
% Stable URL: http://links.jstor.org/sici?sici=0022-166X%28199923%2934%3A4%3C823%3ACOJJH%22%3E2.0.CO%3B2-B 
% 
%     Instrumental Variables: Response to Angrist and Imbens
%         James J. Heckman
%         The Journal of Human Resources, Vol. 34, No. 4. (Autumn, 1999), pp. 828-837.
%         Stable URL: http://links.jstor.org/sici?sici=0022-166X%28199923%2934%3A4%3C828%3AIVRTAA%3E2.0.CO%3B2-Q
% 
  \item \cite{heckman1997}
  \href{http://links.jstor.org/sici?sici=0022-166X%28199722%2932%3A3%3C441%3AIVASOI%3E2.0.CO%3B2-P}{``Instrumental Variables: A Study of Implicit Behavioral Assumptions Used in Making Program Evaluations''}

%  \item Rosenbaum comment (1996 and 1999)

%   \item J.J. Heckman and Edward Vytlacil. (2003).  Structural Equations,
%     Treatment Effects, and Econometric Policy Evaluation

%    \item Pedro Carneiro James J. Heckman, and Edward Vytlacil. (2000)
%      "Understanding What Instrumental Variables Estimate: Estimating
%      Marginal and Average Returns to Education

%   \item Bartels
% 

%
%
%\item Imbens, G.W., Rubin, D.B. and Sacerdote, B. (2001), "Estimating the Effect of
%    Unearned Income on Labor Supply, Earnings, Savings and Consumption: Evidence from
%    a Survey of Lottery Players", American Economic Review, 91, 778�794.
  \end{itemize}

  Application, and use of randomization inference to correct an issue:
  \begin{itemize}
  \item \citet{ImbensRosenbaum2005}: ``Robust, Accurate Confidence
    Intervals with a Weak Instrument: Quarter of Birth and
    Education,'' \textit{Journal of the Royal Statistical Society},
    Series A, vol 168(1), 109--126.

  \item \citet{AngristKrueger1991}: ``Does compulsory school attendance
    affect earnings?'' \textit{Quarterly Journal of Economics} 1991; 106: 979--1019.

  \item \citet{BoundJaegerBaker1995}: ``Problems with Instrumental
    Variables Estimation when the Correlation Between the Instruments
    and the Endogenous Regressors is Weak,'' \textit{JASA} 90, June
    1995, 443--450.
  \end{itemize}

\item {(Regression) Adjustment to Experimental Data}
  \begin{itemize}
  \item \citet{lin_winston_adjustment}:
    \href{http://tinyurl.com/9378kmk}{``Agnostic Notes on Regression
      Adjustments to Experimental Data: Reexamining Freedman's
      Critique.''} \url{http://tinyurl.com/9378kmk}

  \item
    \citet{sekhon_poststrat}:\href{http://sekhon.berkeley.edu/papers/postadjustment.pdf}{Adjusting
      Treatment Effect Estimates by Post-Stratification in Randomized
      Experiments} 

 \end{itemize}

  Extra readings:

  \begin{itemize}
  \item \citet{freedmanOLSa}: ``On regression adjustments to
    experimental data.''
  \item \citet{freedmanLogit}: ``Randomization does not justify
    logistic regression''
  \item \citet{freedmanOLSb}: ``On regression adjustments in
    experiments with several treatments''
  \end{itemize}
 
\item \textsc{Application: Fixing Broken Experiments and a Controversy} 
  \begin{itemize}
  \item Gerber, Alan S. and Donald P. Green. 2000. "The Effects of
    Canvassing, Telephone Calls, and Direct Mail on Voter Turnout: A
    Field Experiment." American Political Science Review 94(3): 653
    663.
  \item Imai, Kosuke. "Do Get-Out-The-Vote Calls Reduce Turnout? The
    Importance of Statistical Methods for Field Experiments." American
    Political Science Review
  \item Green and Gerber Reply
    
  \item Bowers, Jake and Ben Hansen. 2005.
    \href{http://sekhon.berkeley.edu/causalinf/papers/bowershansen2006-10TechReport.pdf}{``Attributing Effects to A Cluster Randomized Get-Out-The-Vote Campaign.''}

%  \item \cite{efron.feldman:1991}
%   \href{http://links.jstor.org/sici?sici=0162-1459%28199103%2986%3A413%3C%3E1.0.CO%3B2-5}{[ JSTOR ToC ]}
  \end{itemize}

\item \textsc{Synthetic Cohorts} \\
  {\em When good matches cannot be found: create a new unit}
  \begin{itemize}
  \item \citet{AbadieGardeazabal2003}:
    \href{http://ksghome.harvard.edu/~.aabadie.academic.ksg/ecc.pdf}{``The
      Economic Costs of Conflict: a Case-Control Study for the Basque
      Country''}
\end{itemize}

\item \textsc{Full and Optimal Matching} 
  \begin{itemize}
    \item \citet{Rosenbaum1991,Rosenbaum1989}
    \item \citet{Hansen2004}
  \end{itemize}

\item \textsc{Sensitivity Analysis for Hidden Bias and other Helpful Suggestions}

\item \textsc{Application: Voting Irregularities} 
  \begin{itemize}
    \item \citet{WSSMHB}: \hlink{The Butterfly Did It: The Aberrant Vote for Buchanan in Palm Beach County, Florida}{http://elections.berkeley.edu/election2000/butterfly.pdf}
    \item \citet{MebaneSekhon2004}: \hlink{Robust Estimation and Outlier Detection for Overdispersed Multinomial Models of Count Data}{http://sekhon.berkeley.edu/elections/election2000/MebaneSekhon.multinom.pdf}
    \item \citet{HerronWand2006}: \hlink{Assessing Partisan Bias in Voting Technology: The Case of the 2004 New Hampshire Recount}{http://www.dartmouth.edu/~herron/nh.pdf}
    \item \citet{Sekhon_florida}: \hlink{The 2004 Florida Optical Voting Machine Controversy: A Causal Analysis Using Matching}{http://sekhon.berkeley.edu/papers/SekhonOpticalMatch.pdf}
  \end{itemize}

\item \textsc{Pre-Test Problems}
\begin{itemize}
\item \citet{diaconis1985}:
  \href{http://www-stat.stanford.edu/~cgates/PERSI/papers/magicalthinking.pdf}{``Theories
    of Data Analysis: From Magical Thinking Through Classical
    Statistics''}

\item \cite{Freedman_screening}: ``A Note on Screening Regression
  Equations''
\end{itemize}

\end{enumerate}


\pdfbookmark[1]{References}{sec:references}
\bibliography{causalinf}

\end{document}


% LocalWords:  Verlag nd Rocio Titiunik GSI Dwinelle sekhon online Venables th
% LocalWords:  Webpage Neyman rubin rosenbaum DiamondSekhon unconfoundedness pm
% LocalWords:  pdf Counterfactuals mimeo DOI Imai Kosuke causalinf TBA Tu sec
% LocalWords:  CA ISBN com ACSSA JSTOR ToC Berk Blalock pp Cochran's Cochran
% LocalWords:  APIOMOS LaLonde Outlier Overdispersed AbadieGardeazabal Winship
% LocalWords:  freedmanLogit freedmanOLSb Krause florida Jasjeet Singh Channing
% LocalWords:  else's Univariate Covariate SekhonInformation MorganHarding SLS
% LocalWords:  lalonde DehejiaWahba SmithTodd GordonHuber GilliganSergenti QJE
% LocalWords:  LenzLadd freedmanOLSa ThistlethwaiteCampbell facto DavidLee JASA
% LocalWords:  HahnToddvanderKlaauw Hidalgo Incompletes WSSMHB MebaneSekhon
% LocalWords:  Multinomial HerronWand sekhonJSS freedmanCh Calnet's Fisherian
% LocalWords:  ImbensRosenbaum leeRD AngristKrueger BoundJaegerBaker Regressors
% LocalWords:  Angrist rn-Steffen Pischke hlink texttt textbf pdfbookmark
% LocalWords:  Ripley textsc jrss1992 przeworski ImbensRosenbaum2005 3AETEEOT
% LocalWords:  ThistlethwaiteCampbell1960 CaugheySekhon2010 Mahalanobis Heckman
% LocalWords:  DehejiaWahba1999 3ACEINSR citeauthor GilliganSergenti2006 Imbens
% LocalWords:  LenzLadd2006 SimmonsHopkins2005 GalianiGertlerSchargrodsky2005
% LocalWords:  ImbensRubinSacerdote2001 3AETEOUI 3AETLMIO AngristKrueger2001
% LocalWords:  ChristakisIwashyna2003 3AIVRTAA 3AIVASOI  maketitle Vytlacil% LocalWords:  AngristKrueger1991
% LocalWords:  AbadieGardeazabal2003 MebaneSekhon2004 HerronWand2006
% LocalWords:  diaconis1985

\documentclass{article}
\usepackage{amsmath}
\usepackage{amsthm}
\usepackage{color}
\usepackage{setspace}
\usepackage{fullpage}
\usepackage[round]{natbib}
\usepackage[utf8]{inputenc}
 
\usepackage{fullpage}
\usepackage{boxedminipage}

\usepackage{listings}

\usepackage{minitoc}

\usepackage{ifpdf}
\usepackage{amsmath}
\usepackage{latexsym}
\usepackage{amssymb}

\usepackage{natbib} 
\usepackage{times} 
\usepackage{setspace}
\usepackage{subfigure}

\usepackage{hyperref} 
\newcommand\independent{\protect\mathpalette{\protect\independenT}{\perp}} 
\def\independenT#1#2{\mathrel{\rlap{$#1#2$}\mkern2mu{#1#2}}} 
\newcommand{\var}[0]{\text{var}}
\newcommand{\cov}[0]{\text{cov}}
\def\E{{\mathbb E}}   
\ifpdf 
\usepackage[pdftex]{graphicx} \else 
\usepackage{graphicx} \fi 

\begin{document}
\begin{itemize}
  \item[1)]
   We first show that 
   \begin{equation}
     r_0 \independent Z | e(X_{Z=1})
     \label{condindep}
   \end{equation}
   where $e(X_{Z=1})$ is the distribution of the propensity score for the treated units.
   
   Note that 
   $P(Z=1|X) = E(Z|X)$.
   Following Rosenbaum and Rubin, we have:
   \begin{eqnarray*}
     \E(Z  | r_0, e(X_{Z=1})) & = & \E[\E[Z | r_0, X_{Z=1}] | r_0, e(X_{Z=1}) ]
     = \E[\E[Z | X_{Z=1}] | r_0, e(X_{Z=1}) ]\\
     &=& \E[e(X_{Z=1})|r_0, e(X_{Z=1})] = e(X_{Z=1})
   \end{eqnarray*}
   and
   \begin{eqnarray*}
     \E(Z | e(X_{Z=1})) = \E[\E(Z | X_{Z=1})| e(X_{Z=1})] = \E(e(X_{Z=1})| e(X_{Z=1})) = e(X_{Z=1})
   \end{eqnarray*}
   Thus, $\E(Z  | r_0, e(X_{Z=1})) = \E(Z  |  e(X_{Z=1}))$, and so,~\eqref{condindep} must hold.
   
   Now, by~\eqref{condindep} and the law of iterated expectations,
   \begin{eqnarray*}
     ATT &=& \E(r_1 - r_0 | Z= 1) = \E[\E(r_1|Z = 1, e(X_{Z=1})] - \E[\E(r_0|Z = 1, e(X_{Z=1})] \\
       &=& \E[\E(r_1|Z = 1, e(X_{Z=1})] - \E[\E(r_0|Z = 0, e(X_{Z=1})] 
   \end{eqnarray*}
   We can compute this last expression using actual data.
   By the assumption that $e(X_{Z=1}) < 1$, the
   expectation $\E[\E(r_0|Z = 0, e(X_{Z=1})]$ is well defined.
   
   To estimate the ATE unbiasedly, we need to strengthen the conditions to
   \begin{eqnarray*}
     r_0, r_1 \independent Z | X \\
     0 < e(X) < 1
   \end{eqnarray*}
   These are the conditions outlined in Rosenbaum and Rubin (1983).
   \end{itemize}     
\end{document}
\documentclass{article}
\usepackage{latexsym}
\usepackage{amsmath}
\usepackage{amssymb}
\usepackage{epsfig}
\usepackage{fullpage}
\usepackage{latexsym}
\usepackage{setspace}
\usepackage{natbib}

\newcommand{\indep}{\perp\!\!\!\perp}
\newcommand{\cov}[0]{\text{cov}}
\newcommand{\var}[0]{\text{var}}
\newcommand{\E}[0]{\mathbb{E}}
\begin{document}
  \begin{itemize}
    \item[1)]
      A scientist runs an experiment, and assigns people to 
      treatment groups and control groups randomly.
      The scientist estimates the ATE two ways: 
      \begin{enumerate}
        \item The scientist computes
          $$
            \widehat{ATE}_1 = \sum_{i=1}^N \frac{Y_iT_i}{\#Trt} -\sum_{i=1}^N \frac{Y_i(1-T_i)}{\#Con} 
          $$
           where $T_i$ are treatment indicators and 
           $\#Trt$ and $\#Con$ denote how many people were assigned to 
           treatment and control respectively.
           This estimate is unbiased for the ATE.
         \item The scientist assumes that responses were generated by the model
           $$
             Y_i = \alpha + \beta T_i + \epsilon_i
           $$
           where $\epsilon_i$ are independent
           and identically distributed with $\E(\epsilon_i) = 0$.
           The scientist obtains $\widehat{ATE}_2 = \hat\beta $ through OLS.
           Since OLS estimates coefficients unbiasedly, this 
           method obtains an unbiased estimate for the ATE.
      \end{enumerate}
      Both methods estimate that the average treatment effect is large and positive.\\[1ex]
      True or False:  
      Both methods obtaining large and positive estimates of the ATE gives 
      more evidence that the ATE is positive than if
      only one of these methods were used.
    \item[2)]
      Consider a large medical trial for a new weight loss drug.  
      Before the trial, 
      each patient has their weight, height, and body fat percentage measured.
      A goodness-of-health score 
      is calculated for each patient based on those characteristics
      (higher scores are a proxy for worse health).
      Assume that patients do not have time to manipulate their weight or body fat
      once selected to participate in the trial.
      Historically, a histogram of patient goodness-of-health 
      scores closely follows a normal
      distribution with mean $c$.
      It is thought that the effect of the drug varies with the value of this score.
      \begin{itemize}
        \item[a)]  Consider the following mechanism for treatment assignment:
          Before being assigned to treatment or control, each patient rolls a 6-sided die.
          For a patient with a score above $c$, 
          If the die comes up as 1, 2, 3, or 4 and the patient has a score of $c$ or above,
          that patient takes the weight loss drug, otherwise they receive a placebo.
          If the die comes up as a 5 or 6 and the patient has a score below $c$,
          that patient takes the weight loss drug, otherwise they receive a placebo.
          Suppose that the die roll is known to the experimenter.
          Under this setup,
          what inferences could you make about the effect of the drug on weight loss? 
          Discuss the parameter of interest and the methods used to estimate this parameter.
          What assumptions are required to estimate this parameter?
        \item[b)]
          Suppose the same set up in part a), except the result of the die roll is unknown to the
          experimenter.  
          Can the parameter of interest in part a) still be estimated?
          If so, how?  If not, why not?
        \item[c)]
          Suppose that the effect of treatment is thought to be the same
          for all patients with goodness-of-health scores within the interval
          $(c-5,c+5)$.
          Suppose that patients with scores below $c$ 
          are ineligible to receive the weight loss drug.
          Patients with scores of $c$ or above are given an appointment to receive 
          the the new drug.
          The drug is administered only once during the trial, and only at this appointment.
          Some patients fail to arrive at their appointment.
          Under this setup,
          discuss at least two types of inference 
          possible for measuring the effect of the drug on weight loss? 
          Discuss the parameters of interest and the methods used to estimate this parameter.
          What assumptions are required to estimate these parameters?
          Which estimate will be larger (in absolute value)?          
      \end{itemize}
    \item[3)]  
      Suppose that there is a study with 
      a total of $2n$ subjects.
      Exactly $n$ of these people are smokers.
      A height and weight are measured for
      each subject.
      Suppose that there are enough people so that
      the joint distribution of the heights and weights is extremely close to a
      multivariate normal distribution.
      The researcher wants to test whether smoking affects 40-yard dash times.
      \begin{itemize}
        \item[a)]
          A statistician notices some imbalance in the average weight and height 
          between the smokers and the non-smokers.         
          To fix the imbalance, the statistician matches smokers
          to non-smokers by matching 
          on the Mahalanobis distance with height and weight covariates          
          (with replacement, nearest neighbor).
          Will the differences in average height and average
          weight between the smokers
          and matched non-smokers            
          be as small or smaller as they were before matching?
        \item[b)]
          Suppose instead that all subjects in the study are twins.
          For each set of twins, one twin is a smoker and one twin is a non-smoker,
          and both twins in each set have the same height and weight.
          In his analysis, the statistician believes that 
          the smoking sibling in a twin pair 
          is essentially random, though he concedes
          that some unobserved trait may help explain a twin's 
          propensity for being a smoker.
           
          For each set of twins $s$, let $(1,s)$ denote the twin that smokes,
          and let $(2,s)$ denote the non-smoking twin.
          Let $T_{is}$ denote random smoking indicators;
          $T_{is} = 1$ if the $i$th unit in the $s$th twin pair smokes, $i = 1,2$.
          For this study, for each pair $s$, 
          the smoking indicators are observed to be $T_{1s} = 1$
          and $T_{2s} = 0$.
          The statistician models the probability that a subject smokes in the following way:
          \begin{equation}
            \log\left( 
              \frac{P(T_{is} = 1)}{1 - P(T_{is} = 1)}
            \right) = \alpha + \kappa_1 h_{is} + \kappa_2 w_{is} + \gamma u_{is}
            \label{probassign}
          \end{equation}
          where $h_{is}$ and $w_{is}$ are the height and weight of twin $(i,s)$,
          and $u_{is}$ is the value of an unobserved covariate for that twin.
          The statistician also assumes that any subject cannot influence
          any other subject to smoke or not smoke
          (smoking is independent across all subjects).
           
          Show that, under this model, the probability
          that subject $(1,s)$ is a smoker is:
%          of the obtaining
%          the observed smoking assignment is:
          \begin{equation}
            P(T_{1s} = 1| T_{1s} + T_{2s} = 1) 
            = \frac{e^{\gamma u_{1s}}}{e^{\gamma u_{1s}} + e^{\gamma u_{2s}}}
            \label{aprobques}
          \end{equation}
          Hint: Use $P(A | B) = P(A\cap B)/P(B)$, and find an expression for
          $$
            \frac{P(T_{1s} = 1 \cap T_{2s} = 0)}{P(T_{1s} = 0 \cap T_{2s} = 1)} = 
            \left(\frac{P(T_{1s} = 1)}{1 - P(T_{1s} = 1)}\right)
            \left(\frac{P(T_{2s} = 1)}{1 - P(T_{2s} = 0)}\right)
          $$
        \item[c)]
          Suppose that $0 \leq u_{is} \leq 1$ and that $\gamma > 0$.
          Find an upper and lower bound (sharper than just 1 and 0) for the probability
          $P(T_{1s} = 1| T_{1s} + T_{2s} = 1).$
          Denote these bounds by $p^+_s$ and $p^-_s$ respectively.
          Do the same for $P(T_{1s} = 0| T_{1s} + T_{2s} = 1).$
          Comment, in one sentence, on how these bounds change if $\gamma < 0$.
        \item[d)]
          Let $Y_{is}$ denote the 40-yard dash time of subject $(i,s)$ in milliseconds.
          Let $Z_s$ denote an indicator variable for the smoker having the faster 40-yard dash time:
          $Z_s = 1$ if and only if twin $(1,s)$ had a faster 40-yard dash time than twin $(2,s)$.
          Let $d_s$ denote the rank of $|Y_{1s} - Y_{2s}|$; higher ranks denote larger absolute values.
          Assume there are no ties between $Y_{1s}$ and $Y_{2s}$ within any 
          twin pair $s$,
          and that $|Y_{1s} - Y_{2s}| \neq |Y_{1t} - Y_{2t}|$ for all distinct twin pairs $s,t$.
          
          The Wilcoxon signed rank statistic is:
          $$
            W = \sum_{s=1}^n d_sZ_s.
          $$
          Let $Z^+_s$ and $Z^-$ be independent and identically distributed 
          bernoulli random variables (or indicator variables) with $P(Z^+_s = 1) = p^+_s$
          and $P(Z^-_s = 1) = p^-_s$.
          Consider the following statistics:
          \begin{eqnarray*}
            W^+ &=& \sum_{s=1}^n d_s Z^+_s \\
            W^- & = & \sum_{s=1}^n d_sZ^-_s
          \end{eqnarray*}
          Show that, under the null hypothesis that smoking does not effect 40-yard dash times, 
          the following property holds:
          $$
           \E(W^-)\leq  \E(W| T_{1s} + T_{2s} =1) \leq \E(W^+) 
          $$
        \item[e)]
          In fact, it can be shown that under this null hypothesis, for any $a$:
          \begin{equation}
            P(T^- \geq a) \leq P(T \geq a | Z_{1s} + Z_{2s} = 1) \leq P(T^+ \geq a)
            \label{probabs}
          \end{equation}
          Discuss, in about 3 -5 sentences or so, how property~\eqref{probabs} can be
          exploited to test the exact null of no treatment effect. 
        \item[Bonus:]
          Prove property~\eqref{probabs}. 
      \end{itemize}
  \end{itemize}  
\end{document}
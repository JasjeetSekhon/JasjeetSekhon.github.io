
\documentclass{article}
\usepackage{amsmath}
\usepackage{amsthm}
\usepackage{color}
\usepackage{setspace}
\usepackage{fullpage}
\usepackage[round]{natbib}
\usepackage[utf8]{inputenc}
 
% Setup for fullpage use
\usepackage{fullpage}

% Uncomment some of the following if you use the features
%
% Running Headers and footers
%\usepackage{fancyhdr}
% Multipart figures
%\usepackage{subfigure}
% More symbols
%\usepackage{amsmath}
%\usepackage{amssymb}
%\usepackage{latexsym}
% Surround parts of graphics with box
\usepackage{boxedminipage}

% Package for including code in the document
\usepackage{listings}

% If you want to generate a toc for each chapter (use with book)
\usepackage{minitoc}

% This is now the recommended way for checking for PDFLaTeX:
\usepackage{ifpdf}

%\newif\ifpdf
%\ifx\pdfoutput\undefined
%\pdffalse % we are not running PDFLaTeX
%\else
%\pdfoutput=1 % we are running PDFLaTeX
%\pdftrue
%\fi
\usepackage{natbib} 
\usepackage{times} 
\usepackage{setspace}
\usepackage{subfigure}

\usepackage{hyperref} 

\newcommand\independent{\protect\mathpalette{\protect\independenT}{\perp}} 
\def\independenT#1#2{\mathrel{\rlap{$#1#2$}\mkern2mu{#1#2}}} 

\ifpdf 
\usepackage[pdftex]{graphicx} \else 
\usepackage{graphicx} \fi 

\title{PS C236A / Stat C239A \\ Problem Set 1 \\ Due: Sept. 21, 2012}
\date{}

\begin{document}

\maketitle

\paragraph{Problem 1:}
Suppose you are in a simplified world, and you wish to determine the returns to education for a group of N workers you have data for.  In this simplified version of the world, there are two factors that influence a worker's income, level of education and intelligence.  The correct model would, therefore, be:
\begin{equation} y_{i} = \alpha_{1} + \gamma_{1}*\text{education level}_{i} + \gamma_{2}*\text{intelligence}_{i} + \epsilon_{1i} \end{equation}
Where $Y_{i}$ is individual $i$'s income.  However, you naively assume that the only factor that influences income is education level, and you run a regression using the following model:
\begin{equation} y_{i} = \alpha_{2} + \beta_{1}*\text{education level}_{i} + \epsilon_{2i}  \end{equation}
\begin{itemize}

\item[a.] Write down or describe the design matrix for the correct model of the world (model 1) as well as the naive model (model 2).
\item[b.] Show that $\frac{1}{N}\sum_{i = 1}^N y_{i} = \frac{1}{N}\sum_{i = 1}^N \hat{y}_{i}$
\item[c.] Which, if any, assumptions and conditions are necessary for part b to be true?
\item[d.] Assume that education level and intelligence are positively correlated.  By using the naive model instead of the true model, what happens to your estimate of $\beta_{1}$?  How would it relate to your estimate of $\gamma_{1}$ if you ran a regression using the true model?  Prove it.
\end{itemize}

% freedman page 66 question 7
\paragraph{Problem 2:} True or False, and explain: as long as the design matrix has full rank, the computer can find the OLS estimator $\hat{\beta}$.  If so, what are the assumptions good for? Discuss briefly.



\paragraph{Problem 3:}
\renewcommand{\theenumi}{\alph{enumi}}
Suppose $x_1, \ldots, x_n$ and $y_1,\ldots,y_n$ have means $\bar x, \bar y$, the standard deviations are $s_x>0, s_y>0$; and the correlation is $r$. Let
$$
\textrm{cov}(x,y) = \frac{1}{n}\sum_{i=1}^n (x_i - \bar x) (y_i - \bar y).
$$
``cov'' is shorthand for covariance. Show that ---
\begin{enumerate}
\item cov$(x,y) = rs_xs_y$
\item The slope of the regression line for predicting $y$ from $x$ is 
$$
\frac{\textrm{cov}(x,y)}{\textrm{var}(x)}
$$
\item var$(x)=\textrm{cov}(x,x)$
\item cov$(x,y) = \overline{xy} - \bar x \bar y$
\item var$(x) = \overline{x^2} -\bar x^2$
\end{enumerate}

\paragraph{Problem4:} 
\renewcommand{\theenumi}{\alph{enumi}} Suppose
$Yi=au_i+bv_i+\epsilon_i$ for $i=1,...,100$.The $\epsilon_i$ are
independent $N(0,1)$.The $u$’s and $v$’s are fixed not random; these
two data variables have mean 0 and variance 1: the correlation between
them is $r$. If $r = \pm 1$, show that the design matrix has rank
1. Otherwise, let $\hat{a}$ and $\hat{b}$ be the OLS estimator. Find
the variance of $\hat{a}$, find the variance of $\hat{b}$, and find
the variance of $\hat{a} - \hat{b}$. What happens if $r = 0.99$? What
are the implications for collinearity for applied work? For instance,
what sort of inferences about a and b are made easier or harder by
collinearity?

\section*{Application}
In this section, you will use R to calculate descriptive statistics
and treatment effect estimates from a dataset used in: 
\begin{quote}
  Benjamin A. Olken. 2007. ``Monitoring Corruption: Evidence from a
  Field Experiment in Indonesia.'' \textit{Journal of Political
    Economy} 115: 300-249
\end{quote}
Note: You can download the data file on the class website at:
\paragraph{}
\url{http://www.sekhon.berkeley.edu/causalinf/fa2009/hw1data.RData}

The data are contained in an object called \texttt{data}.
\paragraph{}
This objective of this experiment was to evaluate two interventions
thought to reduce corruption in road building projects in Indonesian
villages. The two treatments were audits by engineers and efforts to
encourage communities to monitor the projects
themselves. i.e. ``grassroots participation''.  While the actual
experimental design is somewhat involved, in this exercise we will
focus on the intervention designed to increase community
monitoring. The full paper can be found here:
\begin{quote}
  \url{http://econ-www.mit.edu/files/2913}
\end{quote}


Olken describes the intervention to be analyzed as follows:
\begin{quote}
  ...[T]he experiments sought to enhance participation at
  ``accountability meetings'', the village-level meetings in which
  project officials account for how they spent project
  funds. ...[H]undreds of invitations to these meetings were
  distributed throughout the village, to encourage direct
  participation in the monitoring process and to reduce elite
  dominance of the process. 
\end{quote}
Note that residents in treatment villages were notified about these meetings
\textit{before} construction began, but after the total budget
was decided. While the total budget was allocated before assignment to treatment, decisions about how the budget was to be spent was decided after the intervention. 

The main dependent variable is \texttt{pct.missing}, which is a
measure of the difference between what the villages claimed they spent
on road construction and an independent estimate of what the villages
actually spent. Treatment status is indicated by the dummy variable
\texttt{treat.invite}, which takes a value of 1 if the village
received the intervention and 0 if it did not. 

\begin{table*}[h]
  \caption{Variables \label{vars}}
  \centering
  \begin{tabular}{c|c}
    \hline \textbf{Variable} & \textbf{Definition}\\ \hline
    \texttt{pct.missing} & Percent expenditures missing\\
    \texttt{treat.invite} & Treatment assignment \\ 
    \texttt{head.edu} & Village head education \\
    \texttt{mosques} & Mosques per 1,000 \\
    \texttt{pct.poor} & Percent of households below the poverty line\\
    \texttt{total.budget} & Total budget (Rp. million)\\
    \texttt{share.total.unskilled} & Share of road construction expenses spent on
    unskilled labor
  \end{tabular}
\end{table*}
Other variables in the dataset are listed in Table \ref{vars}. 


\paragraph{Problem 4:}
Check whether the variables in the dataset have missing values, and
report the number of missing values by variable. 

\paragraph{Problem 5:}
Report the minimum, maximum, mean, and standard deviation of the
\textit{pre}-treatment covariates in the data set, separately for treatment and
controls.  Are treatment and control units similar in terms of these
characteristics? Be sure that you only include variables that were
measured before the treatment was applied. 

If you can, use a ``for loop'' or the \texttt{apply}
function to calculate these summary statistics. 

\paragraph{Problem 6:}
\begin{itemize}
\item[a.] Report the average difference in the outcome variable by treatment
assignment status (the ``treatment effect''). What is the standard error of this estimate? 

\item[b.] Now estimate the treatment effect using a regression model with no
covariates. Is this estimate different from the
difference-in-means estimate? Are the standard errors of the two estimates different?

\item[c.] Finally, estimate the treatment effect using a regression model, but
this time include all pre-treatment covariates as additional independent
variables.  What is your estimated treatment effect? What is the
standard error of this estimate? Is this estimate substantively
different from the difference-in-means estimate?

\item[d.] Bonus question: Is there a reason to prefer one of these methods of estimating treatment effects over the others?
\end{itemize}

\paragraph{Problem 7:}
In a couple of sentences, what can you conclude about the
effectiveness of this intervention? 
\end{document}
\documentclass{article}
\usepackage{amsmath}
\usepackage{amsthm}
\usepackage{color}
\usepackage{setspace}
\usepackage{fullpage}
\usepackage[round]{natbib}
\usepackage[utf8]{inputenc}
 
\usepackage{fullpage}
\usepackage{boxedminipage}

\usepackage{listings}

\usepackage{minitoc}

\usepackage{ifpdf}
\usepackage{amsmath}
\usepackage{latexsym}
\usepackage{amssymb}

\usepackage{natbib} 
\usepackage{times} 
\usepackage{setspace}
\usepackage{subfigure}

\usepackage{hyperref} 
\newcommand\independent{\protect\mathpalette{\protect\independenT}{\perp}} 
\def\independenT#1#2{\mathrel{\rlap{$#1#2$}\mkern2mu{#1#2}}} 
\newcommand{\var}[0]{\text{var}}
\newcommand{\cov}[0]{\text{cov}}
\def\E{{\mathbb E}}   
\ifpdf 
\usepackage[pdftex]{graphicx} \else 
\usepackage{graphicx} \fi 

\begin{document}

\title{PS C236A / Stat C239A \\ Problem Set 5 - Solutions}
\date{}
\maketitle
\begin{itemize}
\item[1)] 
%An alternative distance metric to the propensity score is  the Mahalanobis distance. This metric also reduces a  multi-dimensional problem into a unidimensional problem. The  Mahalanobis distance was originally developed for use with  multivariate Normal data, however, we often use covariates that are  not normally distributed in our matching methods. This problem will  explore what the implications of these non-normal variables are for  this distance metric.
    \begin{itemize}
    \item[a)] 
      % When including a binary variable in a Mahalanobis distance
      % metric, will a binary variable with $p = \frac{1}{2}$ or a
      % binary variable with $p$ near zero be given greater weight by
      % this distance metric? Prove why this is true mathematically.
\vspace{1em} 
The Mahalanobis distance is defined as:
$$D_m(X_i,X_j)=\left\{ (X_i-X_j)^TS^{-1}(X_i- X_j)
\right\}^{\frac{1}{2}}$$
\vspace{1em}
Where $S^{-1}$ is the inverse of the sample covariance matrix of $X$.  

A binary variable with probability of success $p$ has variance $p(1 -
p)$.  A variable with $p = \frac{1}{2}$ therefore has variance of $p(1
- p) = \frac{1}{4}$, whereas a variable with $p$ near 0 would have
variance near 0. Since we take the inverse of the sample covariance
matrix, therefore dividing by the variance, a variable with $p =
\frac{1}{2}$ will be given less weight than a variable with $p$ near
0 (or, similarly a variable with $p$ near 1). By FOC, we can show that
the variance of a binary variable is greatest when $p = \frac{1}{2}$, so a binary
variable with $p = \frac{1}{2}$ will be given less weight than any binary variable
with $p \neq \frac{1}{2}$

   
\item[b)] Variables with long tails or extreme outliers tend to have
  inflated variances, and by the same logic as above, any variable
  with larger variance will be given relatively less weight.
       
\item[c)] We should be concerned. Outliers and long tails do not make
  a covariate unimportant, so we may not wish to downweight it
  relative to other covariates. Binary variables that are very rare
  may not be of overriding importance, so it may not be wise to give
  them significantly higher weight than binary variables with $p$
  closer to $\frac{1}{2}$. However, if it is a rare binary event, then
  we might want to treat a difference in outcome as worse than
  a difference in outcome for a covariate where $p$ is closer to
  $\frac{1}{2}$. Overall, we should be concerned that Mahalanobis
  distance exhibits these behaviors for variables for which the theory
  was not designed.
      
    \end{itemize}
  \item[2)]
    \begin{itemize}
    \item[a)] The identification strategy in ``Fox News Effect'' is to
      tap the variation in Fox news' cable market penetration across
      localities in order to estimate the effect of access to
      right-leaning television media on turnout and presidential vote
      choice.  The authors argue that Fox news is introduced by 2000
      as-if randomly, {\em after} conditioning on turnout and
      Republican presidential vote choice variables in 1996, as well
      as a number of town-level census demographic variables from 2000
      and 1990, and cable system controls.  The authors check this
      selection assumption by regressing Fox media introduction on
      these controls, in a series of models, some of which include
      cluster-corrected standard errors and fixed-effect terms for
      congressional district membership at the town-level.
      \vspace{1em}

The model with controls, plus fixed-effects and cluster standard
errors reports zero coefficients for the effect of prior presidential
vote turnout and Republican choice on Fox entrance in a locality's
market.  The authors interpret this as recovering conditional
independence between treatment and key (i.e. two) selection confounders.
\vspace{1em}

The authors then regress change in Republican presidential vote choice
and turnout on Fox news entrance in a regression, including the same
controls and fixed-effects specifications that elicited zero
coefficients in the above `selection' regression, interpreting this as
the selection on observations assumptions.  They find that Fox news
had a positive effect on Republican presidential vote.
\vspace{1em}
\newpage

{\em Weaknesses}
\vspace{1em}

The key identifying assumpton is that the included demographic and
cable controls are sufficient to recover conditional independence of
the potential outcomes of Fox exposure and Fox media presence.  The
authors provide no extended discussion as to why any or all of these
are sufficient or necessary to satisfy this assumption, i.e. there is
no theory here.  Also, recovering a zero association with the prior
outcome is not the same as recovering exchangeability on the potential
outcomes conditional on the controls and the model, even if we grant
the selection assumption.  This itself requires an additional
untestable assumption: namely that only the assignment of Fox news,
and no other additional factors are relevant in explaining variation
in the presidential vote at $t+1$, beyond those used to model the vote
at $t$. 


\vspace{1em}

Let's grant however that the selection assumption holds in general,
i.e. these are the correct selection controls for studying the
potential outcomes.  Unlike in a matching framework where we have an
explicit check as to whether conditioning in estimation ensures
similarity across Fox and non-Fox markets, the regression evidence
suggests there is no linear dependence between the prior outcome and
treatment, given the controls and the model.  This requires a number
of strong modeling assumptions to hold, i.e. correct model for cluster
covariance, the fixed-effects model is capturing remaining unit
differences (but why not random effects?), stability and unbiasedness
in the regression estimates, etc, for the placebo estimates to be
informative.  This implies that the selection assumption in the
regression framework can be quite strong, (and perhaps much stronger
than in non-parametric matching), especially since we lose the ability
to check for whether we recover similarities on the covariates after
conditioning in an separate testing stage. 
 \vspace{1em}

 Finally on a similar point, since a placebo regression was used to
 identify covariates during a specification search to construct the
 included model, this changes the interpretation of this as an
 independent test of a placebo.  Given that the search is aimed at
 specifying a model that recovers a zero here, this test provides no
 new information that included covariates satisfy the conditional
 exchageability assumption.  \vspace{1em}

% Finally, on this point, existing replications of the study using
% non-parametric matching approaches recover different results than
% that in the paper, suggesting there is such model sensitivity to
% assumptions that are troubling in interpreting the result here as
% causal.  \vspace{1em}

{\em Convincing}
\vspace{1em}

The potentially haphazard introduction of Fox news may provide for an
opportunity to estimate the causal effect of biased media on political
choices and attitudes. However, it is not convincing that the
demographic and cable controls contain the comprehensive information
used by the Fox organization when deciding where to expand, and thus
are sufficient to recover exchangeability.  One possible improvement
would be to get this covariate information from Fox, and then include
that in the selection stage of the design.  Having this type of
information would seriously improve the plausibility of the result,
since it would permit (at least some of) the actual controls used to
determine market penetration to be included in the study.  \vspace{1em}

The most convincing piece of evidence is that the authors recover zero
effects for Fox market present on the pre-introduction election
outcomes.  This evidence is suggestive that the model and controls are
reducing bias, but not dispositive since these zeros could also be
sensitive to model choices and assumptions.  Also, see above about how
best to interpret this information.  

     \item[b)] This section is very important to establishing the
       argument that Fox entrance is conditionally random based on the
       controls and model choices.  As mentioned above, my approach
       would be, at least, to match first and obtain balance on
       relevent covariates.  And then do the regression analysis,
       after matching, to ensure that the model assumptions do not
       drive the interpretation that conditional exchangeability holds.

     \item[c)]  The existing data I would want is the same data used
       by Fox to determine where and when to enter a cable market, as
       well as additional data that Fox possessed on cable pricing and
       competition. It would also be great to have similar data for
       the local cable providers who marketed and made decisions about
       whether or not to carry Fox above other programming.  I would
       then matching on this data, using GenMatch until balance was
       obtained before estimating treatment effects.  With data
       limitations, I would still match and obtain balance on whatever
       controls were available that seem to be plausible predictors of
       Fox market entrance.    
     \item[d)] Overall, I think the finding is plausible, but not
       particularly convincing.  It's unclear whether the regression
       model is removing confounding, and the empirical tests of this
       claim require strong model assumptions.  Relaxing these
       assumptions, and getting more detailed information about the
       selection process would shore up the strength of the findings.






   \end{itemize}
\item[3)] See \texttt{HW5\_Answers.R} for solutions
\end{itemize}     
\end{document}
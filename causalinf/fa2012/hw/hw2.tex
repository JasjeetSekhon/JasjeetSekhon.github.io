
\documentclass{article}
\usepackage{amsmath}
\usepackage{amsthm}
\usepackage{color}
\usepackage{setspace}
\usepackage{fullpage}
\usepackage[round]{natbib}
\usepackage[utf8]{inputenc}
\usepackage{amssymb} 

% Setup for fullpage use
\usepackage{fullpage}

% Uncomment some of the following if you use the features
%
% Running Headers and footers
%\usepackage{fancyhdr}
% Multipart figures
%\usepackage{subfigure}
% More symbols
%\usepackage{amsmath}
%\usepackage{amssymb}
%\usepackage{latexsym}
% Surround parts of graphics with box
\usepackage{boxedminipage}

% Package for including code in the document
\usepackage{listings}

% If you want to generate a toc for each chapter (use with book)
\usepackage{minitoc}

% This is now the recommended way for checking for PDFLaTeX:
\usepackage{ifpdf}

%\newif\ifpdf
%\ifx\pdfoutput\undefined
%\pdffalse % we are not running PDFLaTeX
%\else
%\pdfoutput=1 % we are running PDFLaTeX
%\pdftrue
%\fi
\usepackage{natbib} 
\usepackage{times} 
\usepackage{setspace}
\usepackage{subfigure}

\usepackage{hyperref} 

\newcommand\independent{\protect\mathpalette{\protect\independenT}{\perp}} 
\def\independenT#1#2{\mathrel{\rlap{$#1#2$}\mkern2mu{#1#2}}} 

\ifpdf 
\usepackage[pdftex]{graphicx} \else 
\usepackage{graphicx} \fi 

\title{PS C236A / Stat C239A \\ Problem Set 2 \\ Due: Sept. 28, 2012}
\date{}

\begin{document}

\maketitle
\vspace{-4em}
\section*{Instructions}
This assignment is due {\bf 4 pm Friday, Sept. 28.}  You may submit
your analytical work either electronically or in paper form.
Electronic versions must be sent as a .pdf to
$<$\texttt{jahenderson[at]berkeley.edu}$>$. Paper copies should be
placed in my mailbox in 210 Barrows.  For the computing portion of the
assignment, you \underline{must} submit a fully executable version of
all .R code, along with any data used in the code (excepting that
provided through the course webpage) to the email above.  All files
for each assignment sent electronically should be included in one
omnibus email, with the subject line containing the course and
homework number, and your last name (e.g., PS239A/STAT236A: HW2 - Obama).\\







\noindent You are encouraged to work together in groups to complete
the assignments. However, you must hand in your own individual
answers. Photocopies and other reproductions of someone else’s answers
are not acceptable. Please also list the names of everyone with
whom you have collaborated on this assignment.


%test statistic question; maybe in R

%permutation inference in R;

%univariate matching 

%\section*{Permutation Inference}



\paragraph{Problem 1: The Lady Tasting Tea }
Consider the following variation of the Lady Tasting Tea example that we discussed in class. The Lady tastes eight cups of tea, four of which have milk added first and four of which have tea added first. The cups are organized into matched pairs and for each pair, a fair coin is flipped to determine which gets milk first. The Lady knows the design, meaning that she knows there is one milk-first cup and one tea-first cup in each matched pair.

\begin{itemize}


\item[a.]  In the case where the Lady makes one mistake (classifies
  one milk-first cup as a tea-first cup), what is the $p$-value for a
  test under the null hypothesis that the Lady has no ability to
  discriminate the order in which milk is added to tea?

\item[b.] Pretend that you mistakenly thought that assignment of
  milk-first or tea-first was completely randomized, i.e. that there
  was no randomization within matched pairs, but rather across all
  cups. If the Lady makes one mistake, what $p$-value would you
  calculate for a test under the null hypothesis that the Lady has no
  ability to discriminate the order in which milk is added to tea?  Is
  this $p$-value different from the one calculated in part (a)? Why or
  why not?


  % \item[c.] Discuss why the $p$-value calculated in part (a) is
  %   different than the one calculated in part (b). Discuss the
  %   implications of
  %   these differences for when permutation tests provide a test with
  %   the
  %   correct level. How does this relate to the one-margin versus
  %   two-margin fixed issue we discussed in class? More generally,
  %   discuss the implications for the the analysis of experiments.

  % In the case where the Lady makes one mistake (classifies one
  % milk-first cup as a tea-first cup), what is the $p$-value for a
  % test
  % under the null hypothesis that the Lady has no ability to
  % discriminate the order in which milk is added to tea?
\end{itemize}

Now, instead of using fixed margins, let’s imagine that we conduct the
Lady Tasting Tea experiment under binomial randomization {\em without}
a fair coin.  There are {\em six} cups, $C_c = \left\{ C_1, C_2,...,C_6 \right\}$,
(indexed by $c$) with the following vector of probabilities of having
milk first, $p_c= \left\{ p_1, p_2,...,p_6 \right\}$, and $1-p_c=
\left\{1-p_1, 1-p_2,...,1-p_6 \right\}$ probabilities of having tea
added first. The Lady does not know the values of $p_c$, but does know
that the cups are assigned randomly under binomial randomization

% fixed margins, let's imagine that we conduct the Lady Tasting Tea
% experiment under binomial randomization.  There are six cups, and
% each has a probability $p = 2/3$ of having milk first and a $1-p$
% probability of having tea added first.  The Lady does not know the
% value of $p$, but does know that the cups are assigned randomly
% under binomial randomization.


\begin{itemize}

\item[c.] First fix $p_c=2/3$, for all $c$ cups.  Which null hypothesis would we prefer: The Lady has no
  ability to identify milk-first cups or The Lady has no ability to
  identify tea-first cups. Why?


%In the case where the Lady
%  makes no mistakes, what is the $p$-value for a test under the null
%  hypothesis that the Lady has no ability to discriminate the order in
 % which milk is added to tea?

\item[d.] Now fix $p_c=\left\{ 0.45,0.5,0.55,0.8,0.85,0.9 \right\} $.
  If the Lady makes one mistake, now what is the $p$-value for a test
  under the null hypothesis that the Lady has no ability to
  discriminate the order of milk first?  What if a researcher, years
  later, came across this data and assumed $p_c=0.5$, $\forall c$?
  How much would this bias the inferences the researcher draws from
  the experiment?  What is the expected number of cups with milk first
  under the true assignment mechanism (rounded to the nearest cup)?
  How probable is it to realize this (rounded) expected number of
  milk-first cups with $p_c = .5$?  Is this rare?



\item[e.] Continuing with this example, {\em fix the margins} so that
  the number of cups with milk first is held at {\em three}, but the
  probability of each cup is determined by the above non-fair coin in
  (d) (i.e., the orderings are a function of these individual
  probabilities).  Assume the experimental draw is $C_c = \left\{ 0,
    1, 1, 0, 1, 0 \right\}$, and the Lady selects two milk-first cups
  correctly, with an associated $p$-value, $q_1$, under the null of no
  ability.  If we repeat the experiment and randomly assign cup order
  as $C_c = \left\{ 0, 0, 1, 0, 1, 1\right\}$ and again the Lady gets
  two correct with $p$-value $q_2$, is $q_1=q_2$?  Why or why not?

  % What does this tell you about If not, what changed between the two
  % experiments?

  % \item[e.] Which null hypothesis would we prefer: The Lady has no
  %   ability to identify milk-first cups or The Lady has no ability
  %   to
  %   identify tea-first cups. Why?
\end{itemize}

%\section*{Observational Studies}



\paragraph{Problem 2: Catholic School -- I}  In an observational study of the effects of
attending a Catholic school, the central dependent variable of
interest is a binary variable, $Y_i$, which indicates whether or not
student $i$ graduated from high school. The treatment variable, $T_i$,
indicates Catholic school attendance. In a very large sample of
students, half attended Catholic school and half did not. You observe
that the treated students have a graduation rate of .7 and the control
students have a graduation rate of .5. You wish to estimate the
average treatment effect of attending Catholic school. Assume that
your sample is large enough to make sampling variability negligible.

\begin{itemize}
\item[a.] Without making any assumptions about the relationship between the students’ potential outcomes and treatment assignment, what is the largest possible value of the ATE? What is the smallest possible value of the ATE? What is the difference between these two values? Will this difference between the maximum and minimum possible ATE always be the same, irregardless of the specific observed values of the outcome variable?
\item[b.] Again making no assumptions about treatment assignment, assume that Catholic school does not prevent any student from graduating. What is the largest possible value of the ATE? What is the smallest possible value of the ATE?
\end{itemize}

\paragraph{Problem 3: Catholic School -- II}

To address this question more precisely, researchers randomly sample
$n$ students, collecting the variable $X$ for students who attend
Catholic (treateds) and non-Catholic schools (controls).  The
researchers then exactly match 6 controls to 6 treated students on
$X$, producing the data in the Table below.  Assume {\em
  unconfoundedness} conditional on $\{X,U\}$, so that units are
exchangeable across Catholic and non-Catholic attendance given $X$ and
$U$.  Also assume that the conditionality in $T_i$ follows a logit
distribution, $\pi_i/(1-\pi_i) = \exp(X_i\beta+\gamma U_i)$, where
$\pi_i$ is student $i$'s probability of attending Catholic school,
$U_i \in\{0,1\}$ is a binary variable, and $\gamma$ and $\beta$ are an
additive parameters.
% (Note: this of course means $\pi_i = 1/(1+\exp(-f(X_i) - \gamma
% U_i))$)


\begin{table}[!h]
		\caption{Catholic School Graduation Data}
	\begin{center}
		\begin{tabular}{ccccc}
			Strata (S) & $Y$ & $T$ & $X$ & $U$ \\ \hline  
			1    & 1     & 1   & \ .89   & 1     \\ 
			2    & 1     & 1   & -.25   & 1       \\ 
			3    & 1    & 1    & \ .67   & 1     \\ 
			4    & 1    & 1    & -.11   & 1     \\ 
			5   &  0    &  1    &\ .13   & 1     \\ 
			 6  & 1     & 1     &\ .73   & 1    \\ 
                         1    & 0   & 0    &\ .89   & 1       \\ 
			2    & 0   & 0    & -.25   & 0       \\ 
			3    & 0    & 0    &\ .67   & 0     \\ 
			4    & 0    & 0    & -.11   & 0     \\ 
			5 &  1     &  0    &\ .13   & 0      \\ 
			 6  & 0    &  0   &\ .73    & 1    \\     
		\end{tabular}
		\label{}
	\end{center}
\end{table}

\begin{itemize}
\item[a.] Ignore the strata for a moment.  What is Fisher's test
  statistic for this data?  Assume $\beta=0$ and $\gamma=0$.  What is
  the permutation $p$-value for this statistic, under the sharp null
  of no difference in graduation outcomes for Catholic and
  non-Catholic schools? Now assume $\beta=1.45$ and $\gamma=0$.  What
  is the $p$-value under Fisher's sharp null?
\item[b.] Turning to the stratification analysis, what is the McNemar
  test statistic for the matched-pair data?  Again assume $\beta=1.45$
  and $\gamma=0$. What is the permutation $p$-value for this
  statistic, under the sharp null of no difference in graduation
  outcomes in the matched pairs?
\item[c.] Assume that matching only eliminated the bias in $X$, and
  thus $\gamma>0$. Assuming $\beta=1.45$, at what minimum (positive)
  value of $\gamma$ would we {\em fail to reject} the sharp null of no
  effect given the matched-pair design at the $p\geq.05$ level?

\end{itemize}


\paragraph{Problem 4}
Consider an observational study, where $Z_i=1$ if unit $i$ is in the treatment group and
$Z_i=0$ if unit $i$ is in  the control group. Let $X$ be
a vector of observed pretreatment covariates. Write
$X_{Z=1}$ for the observed covariates of the units in the
treatment group. Similarly, let $X_{Z=0}$ be the observed
covariates in the control group.  Let $r_{1}$ be outcome under treatment
and $r_{0}$ be the outcome under control.  Assume the following:
$$r_0 \independent Z|X_{Z=1}$$
$$ P(Z=1|X_{Z=1})<1$$

\noindent Suppose you know the propensity score $e(X)=P(Z=1)$ for all
units $i$.  With these assumptions, can conditioning on the propensity
score estimate the ATT without bias? Prove it mathematically and
describe your logic in words.  What additional assumption would we
need in order to estimate the ATE without bias? 

% First show that conditioning on the propensity score is equivalent
% to conditioning on $X_{Z=1}$. Then show that conditioning on the
% propensity score can produce unbiased ATT estimates under the
% assumptions above.



\paragraph{Problem 5:}
In this problem, you will analyze a famous experiment conducted by Leonard Wantchekon in Benin in 2001. Wantchekon wanted to examine the effectiveness of different types of campaign messages on voting behavior in a presidential election.
For details, see:
\begin{quote}
  \url{http://www.princeton.edu/~lwantche/Clientelism_and_Voting_Behavior_Wantchekon.pdf}
\end{quote}

 Wantchekon convinced the campaigns of the major presidential candidates to randomize the messages they employed in 24 villages. The three treatment conditions were as follows: 
\begin{enumerate}
\item \textit{Public Policy:} Wantchekon describes this treatment condition as: ``It was decided that any public policy platform would raise issues pertaining to national unity and peace, eradicating corruption, alleviating poverty, developing agriculture and industry, protecting the rights of women and children, developing rural credit, providing access to the judicial system, protecting the environment, and/or fostering educational reforms.''
\item \textit{Clientelist}: Wantchekon describes this treatment as: ``A clientelist message, by contrast, would take the form of a specific promise to the village, for example, for government patronage jobs or local public goods, such as establishing a new local university or providing financial support for local fishermen or cotton producers.''
%\item \textit{Both}: These villages received both types of messages. 
\end{enumerate}

The data has been modified for the assignment, but the basic structure
of the experiment was \textit{block} randomization. For the purposes
of the assignment, villages were divided into groups of 2 based on
geography and treatment status was randomized within the 8 groups of
2. The outcome variable is the vote share of the candidate
participating in the experiment. The only covariate is the number of
registered voters. In the dataset, \texttt{block} indicates block
group, \texttt{reg.voters} is the registered voters
covariate,\texttt{vote.pop}is the outcome variable, \texttt{treat} is
a variable indicating treatment status.\\

\noindent In this problem, we are interested in the difference between
the clientelist and public policy conditions.  
\begin{itemize}
\item[a.] Estimate the effect the clientelist message compared to the
  public policy message, using the ITT estimator and the regression
  estimator. For the regression estimate, include block level dummy
  variables in your regression equation.  
\item[a.] Now test the sharp null
  of no treatment effect using randomization inference. Use two test
  statistics: Wilcoxon’s signed rank test (Rosenbaum 2002, pg. 32) and
  the difference in means. What are the two sided $p$-values under these
  two tests?  
\item[a.] Under the assumption of a constant, additive, treatment
  effect, use randomization inference to find a 95\% confidence
  interval of the treatment effect. Use the signed rank as your test
  statistic. See pages 44-46 in Rosenbaum (2002).
\item[d.] What can you conclude about the effectiveness of clientelistic
  appeals in Benin?
\item[e.] Bonus: Perform randomization inference with covariance adjustment. How does this effect your results? For a very good article on  covariance adjustment with randomization inference, see: 
\begin{quote}
Rosenbaum, Paul. 2002. “Covariance Adjustment in Randomized Experiments and Observational Studies.” \textit{Statistical Science} 17(3): 286-327. 
\end{quote}
\end{itemize}



\end{document}
% LocalWords:  texttt head.edu rnorm noindent Olken

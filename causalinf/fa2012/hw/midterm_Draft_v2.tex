
\documentclass{article}
\usepackage{amsmath}
\usepackage{amsthm}
\usepackage{color}
\usepackage{setspace}
\usepackage{fullpage}
\usepackage[round]{natbib}
\usepackage[utf8]{inputenc}
\usepackage{amssymb} 

% Setup for fullpage use
\usepackage{fullpage}

% Uncomment some of the following if you use the features
%
% Running Headers and footers
%\usepackage{fancyhdr}
% Multipart figures
%\usepackage{subfigure}
% More symbols
%\usepackage{amsmath}
%\usepackage{amssymb}
%\usepackage{latexsym}
% Surround parts of graphics with box
\usepackage{boxedminipage}

% Package for including code in the document
\usepackage{listings}

% If you want to generate a toc for each chapter (use with book)
\usepackage{minitoc}

% This is now the recommended way for checking for PDFLaTeX:
\usepackage{ifpdf}

%\newif\ifpdf
%\ifx\pdfoutput\undefined
%\pdffalse % we are not running PDFLaTeX
%\else
%\pdfoutput=1 % we are running PDFLaTeX
%\pdftrue
%\fi
\usepackage{natbib} 
\usepackage{times} 
\usepackage{setspace}
\usepackage{subfigure}

\usepackage{hyperref} 

\newcommand\independent{\protect\mathpalette{\protect\independenT}{\perp}} 
\def\independenT#1#2{\mathrel{\rlap{$#1#2$}\mkern2mu{#1#2}}} 



\ifpdf 
\usepackage[pdftex]{graphicx} \else 
\usepackage{graphicx} \fi 

\title{PS C236A / Stat C239A \\  Midterm Exam \\ Due: December 4, 2012}
\date{}

\begin{document}

\maketitle
\vspace{-4em}
\section*{Instructions}




%This is an ungraded practice exam.  The following instructions outline
%the expectations for the upcoming midterm. \vspace{1em}

\noindent This exam is due at the beginning of class (2:10 pm) on
Tuesday, December 4. The questions below will be graded as follows:
True/False (I) 10\%, analytical section (II, III, IV) 40\%, and the
empirical section (V) 50\%.  You \underline{\bf must} submit your
midterm answers in paper form to class.  This material should include
all .R output, figures, and tables needed to answer the computing
portion of the exam.  We will not read computer code to find your
answer, however, you \underline{\bf must} submit a fully executable
version of all .R code to
$<$\texttt{jahenderson[at]berkeley.edu}$>$. If you do not send an
electronic version of your .R code, that portion of the midterm
\underline{\bf will not} be graded.  All files sent electronically
should be included
in one omnibus email, with the subject line containing the course number and your last name (e.g., PS239A/STAT236A: Midterm - Rice).\\


\noindent Note: This exam is open book.  However, during the exam,
you are not allowed to communicate or cooperate with anyone in
any way about the exam. Any questions should be asked directly to the
Professor or the GSI. To repeat: you may not use study groups, online
help forms, the writing center, or any other form of external help.
If in doubt, ask.







\paragraph{\Large I. True or False}
Answer {\em True} or {\em False}.  Explain your answer in a sentence or two.
\begin{itemize}
\item[1.]   A scientist runs an experiment, and assigns people to 
      treatment groups and control groups randomly.
      The scientist estimates the ATE two ways: 
      \begin{enumerate}
        \item The scientist computes
          $$
            \widehat{ATE}_1 = \sum_{i=1}^N \frac{Y_iT_i}{\#Trt} -\sum_{i=1}^N \frac{Y_i(1-T_i)}{\#Con} 
          $$
           where $T_i$ are treatment indicators and 
           $\#Trt$ and $\#Con$ denote how many people were assigned to 
           treatment and control respectively.
           This estimate is unbiased for the ATE.
         \item The scientist assumes that responses were generated by the model
           $$
             Y_i = \alpha + \beta T_i + \epsilon_i
           $$
           where $\epsilon_i$ are independent
           and identically distributed with $\E(\epsilon_i) = 0$.
           The scientist obtains $\widehat{ATE}_2 = \hat\beta $ through OLS.
           Since OLS estimates coefficients unbiasedly, this 
           method obtains an unbiased estimate for the ATE.
      \end{enumerate}
      Both methods estimate that the average treatment effect is large and positive.\\[1ex]
      True or False:  
      Both methods obtaining large and positive estimates of the ATE gives 
      more evidence that the ATE is positive than if
      only one of these methods were used.
\item[2.]  A researcher is analyzing the effect of a treatment in a
  randomized experiment and uses a two-sample $t$-test (with unequal
  variances) to reject the null hypothesis of a 0 average treatment
  effect. The researcher could have tested the same null hypothesis
  with a randomization (permutation) test and his inference would not
  depend on any large sample approximations.

\item[3.] A group of researchers begin a drug trial to study the
  effectiveness of a particular psychostimulant in treating Attention
  Deficit Hyperactivity Disorder (ADHD).  At time $t$, people in the
  study are randomly assigned either to treatment and receive the
  drug, or to control and receive a sugar pill placebo. Since
  compliance is a common problem in drug trials, the researchers
  included an additional intervention at $t+1$ aimed to increase
  compliance rates.  In this second part, half the subjects were
  randomly assigned to an `encouragement' condition, where they were
  counseled on the importance of taking their assigned pill dosage --
  the other half of the subjects received no such encouragement.
  Since both the drug and encouragement interventions are randomly
  assigned, it is generally valid (without additional assumptions) to
  estimate the Intention-to-Treat (ITT) effect of the drug on changes
  in behavior at $t+2$, by pooling all people in the drug arm to
  measure the average outcome for the treated, and pooling people in
  the placebo to measure the average outcome for the controls, with
  the estimate of the ITT being the difference between these two
  averages.

\end{itemize}


\paragraph{\Large II. Sample Selection \\ \\}
\vspace{1em}

\noindent A political scientist wants to estimate the personal incumbency
advantage. Let $Y_i(1)$ be the vote share in the next election of
candidate $i$ if he or she wins in the present election, and $Y_i(0)$
is the vote share of candidate $i$ in the next election if he or she
loses in the present election.  Define incumbency advantage as:
$$E[\delta] = E[Y_i(1) - Y_i(0)]$$

Let $D_i$ be an indicator (treatment) variable for whether or not
candidate $i$ wins election. If all candidates re-run in the subsequent
election, then the political scientist would observe $Y_i(1)|D_i = 1$ and
$Y_i(0)|D_i = 0$.

Unfortunately, not all candidates re-run in subsequent elections. Let
$R_i$ be an indicator variable for whether or not candidate $i$ runs for
office in the subsequent election. Thus, the political scientist can
observe $Y_i(1)|(D_i = 1, R_i = 1)$ and $Y_i(0)|(D_i = 0, R_i = 1)$, but not
$Y_i(1)|(D_i = 1, R_i = 0)$ and $Y_i(0)|(D_i = 0, R_i = 0)$.

Denote the proportion of treated units in the population as $\pi$ and the
proportion of candidates running for office in the next election as
$\lambda$. Assume a very large sample so that sampling error is negligible.

\begin{itemize}
\item[a.]  The political scientist naively estimates the incumbency
  advantage using the following estimator: 
$$\hat{\delta}_{naive} = E[Y_i|(D_i = 1, R_i
= 1)] - E[Y_i|(D_i = 0, R_i = 1)]$$

This is simply the average difference in vote shares between the
winners and losers {\em who run again} in the following
election. Without making any assumptions, if the estimand is the
average treatment effect ($E[\delta]$) what is the bias in this
estimator? Be sure to account for the bias resulting from
``fundamental missingness'' (unobservability of the counterfactual
conditions) as well as the bias resulting from candidates not always
rerunning.


\noindent {\em Hint}: Decompose $E[Y_i(1)]$ as the weighted average of
four causal types, based on their potential outcomes under treatment
and control and whether or not they re-run. Do the same for
$E[Y_i(0)]$.

\item[b.] Assume $(Y_i(1),Y_i(0)) \perp D_i$. With this assumption,
  what is the bias in the naive estimator? Under what conditions would
  the bias be 0?

\item[c.]  Assume that incumbency advantage is bigger among candidates
  who re-run than those who do not re-run. Under this assumption, as
  well as the independence assumption made in part (b), what is the
  largest possible value of $E[\delta]$? What is the smallest?
\end{itemize}

\paragraph{\Large III. Regression Discontinuity \\ \\}

Imagine a study where a treatment group of unemployed workers in San
Francisco are given the opportunity to participate in a worker
training program. Six months after the treatment is administered, the
workers' employment status and yearly income is measured. You are
asked to evaluate the effect of this program.

Please be explicit about your assumptions, how you would make your
inferences, and the workers to which your inferences would apply. In
your answers below, use mathematical notation where appropriate.

\begin{itemize}
\item[a.] Suppose that the randomization to treatment occurred as
  follows: a randomly generated number $X$ is drawn from a uniform
  distribution with the range [0, 4]. Units with $X \geq 2$ are given
  the treatment while units with $X < 2$ are denied treatment. All
  workers assigned to treatment are forced to attend the training
  program. Under this setup, what inferences could you make about the
  effect of the training program on the workers' employment status and
  income? What assumptions are required?
\item[b.] Now imagine that for ethical reasons, workers are
  compensated for having received a ``bad draw'' by being awarded
  monetary compensation inversely proportional to the random number
  $X$. So workers with a $X \approx 0$ receive a large sum of money
  and those with $X \approx 4$ receive very little. The workers are
  enrolled in the worker trainer program if $X \geq 2$. Under this
  setup what inferences could you make about the workers training
  program? How would you make these inferences?

\item[c.] Suppose the same set-up as in part (b) (including the
  compensation), except workers with $X\geq2$ flip a coin -- those who
  flip heads are enrolled in the worker trainer program, and those who
  flip tails are not.  The value of $X$ is observed, but it is not
  possible to know whether or not a worker actually participated in
  the program.  What assumptions need to be made in order to bound the
  estimand in (b)?
\end{itemize}

\paragraph{\Large IV. Media Bias \\ \\}

For this problem, you will compare the research design from three
papers studying the effects of media bias on political attitudes and
choices.  The first paper is ``The Fox News Effect'', by S.
DellaVigna and E. Kaplan (DVK), and can be found here
\url{http://sekhon.berkeley.edu/causalinf/papers/DellaVignaFoxNews.pdf}. The
second paper is ``Exploiting a Rare Shift in Communication Flows to
Document News Media Persuasion'', by J. Ladd and G. Lenz (LL),
and can be found here
\url{http://sekhon.berkeley.edu/causalinf/papers/LaddLenzBritish.pdf}. And
the third paper is ``Does the Media Matter? A Field Experiment
Measuring the Effect of Newspapers on Voting Behavior and Political
Opinions'', by A. Gerber, D. Karlan, and D. Bergan (GKB), and
can be found here
\url{http://sekhon.berkeley.edu/causalinf/papers/GerberNewspapers.pdf}.

\vspace{1em}

Please write a page or two addressing the following questions:


   \begin{itemize}
   \item[a.]  Compare the identification strategies of the
     three papers.  Which strategy do you find the most convincing?
     The least?  Why?

   \item[b.] Given the different types of interventions being studied
     (e.g., biased media exposure v. change in media bias, television v. newspaper media,
     etc), in what sense are the findings across these three studies
     `comparable'?  Do these studies give us useful information to
     test the same theoretical claim or different theoretical claims?  

   \item[c.] Imagine at a future point in time, Fox News expanded to
    every major cable and media market in the US. Would we 
     expect to see a similar media effect as measured by DVK as a
     result of this national expansion?  Why or why
    not?  What would be an analogous type of issue in the studies
    conducted by LL and GKB?  Do any of the three studies seem more
    robust to this issue than the others?

   \item[d.]  Which paper do you find the most 
     interesting, weighting both the scope and significance of the
     effect being estimated, as well as the {\em external} and {\em internal} validity
     of the respective estimates?  Generally speaking, which study is
     more informative about the substantive impact of media bias on 
public opinion or vote choice?  


\end{itemize}




\paragraph{\Large V. Data and Matching \\ \\}

For this problem, you will perform several matching exercises using
Ladd and Lenz's ``Exploiting a Rare Shift'' data.  The unit of
observation is the individual respondent in a UK election survey, and
the treatment under study is whether an individual is a reader of a
newspaper that switched its party endorsement from Tory to Labour in
the run-up to the 1997 election.  The main outcome is change in Labour
party vote support between 1992 and 1997.  To control for confounding,
the authors condition on a number of covariates (listed in Table 3
and Table 1A of their paper) that may predict both readership and party voting
behavior.
\\

\noindent The Ladd and Lenz data is available here:
\url{http://sekhon.berkeley.edu/causalinf/data/midterm.dta}.
The variables are described in the following file:
\url{http://sekhon.berkeley.edu/causalinf/data/midterm_codebook.xsls}


\vspace{1em}
\noindent For parts (a) - (e) below, be sure to explicitly set seeds
to ensure that GenMatch recovers reproducible results,
i.e. \texttt{set.seed} in general, and in GenMatch \texttt{unif.seed}, \texttt{int.seed}.

\begin{itemize}
\item[a.]  Estimate the causal effect of being a typical reader of a
  newspaper that switched party endorsement (from Tory to Labour) on
  the {\em change} in Labour party vote support between 1992 and 1997.
  In doing so, select a set of relevant covariates to condition on.
  In matching, first use a custom loss function and then use
  GenMatch’s default loss function. Provide some justification for
  your custom loss function. Choose the matched dataset with the best
  balance on the relevant covariates.  Are the media effects on voting
  you estimate significantly different from zero? What are the mean
  differences in change in party vote suport you recover after
  matching?  What are
  the three {\em worst} balanced covariates in this best-matched
  dataset?  What are the standardized mean differences across matched
  treated and control on these three covariates?


\item[b.] Using the best-matched dataset from part (a), stratify the
  matched-pairs to include only those treated individuals who also are
  also habitual readers of a newspaper that switched its party
  endorsement.  Check balance on this `stratified' dataset using
  \texttt{MatchBalance}.  Does balance change considerably in this
  dataset, compared with that recovered in (a)?  Now, use GenMatch to
  match on the same covariates used in (a), utilizing habitual readers
  as the treatment indicator.  Does balance in this matched dataset
  improve compared to the `stratified' matched data?  What media
  effects do you recover in these two matched datasets?  Are these
  different from that found in (a)?

\item[c.] Choose one matched dataset from (a) or (b) that you think is
  the most convincing in recovering conditional exchangeability (for
  either habitual or typical readers), and conduct two robustness
  tests of the conditional exchangeability assumption.  The first
  robustness check should be a Rosenbaum sensitivity test using the
  \texttt{rbounds} package in {R}.  The second robustness check either
  should be a post-matching parametric bias adjustment on the matched
  data (e.g., a probit regression including covariates and treatment
  to model the outcome on the matched data), or a placebo test of the
  effect of treatment on a prior party vote outcome before and after
  matching.  What is the $\Gamma$ magnitude of confounding due to an
  unobserved covariate in the Rosenbaum sensitivity test at which the
  estimated treatment effect is indistinguishable from zero?  How does
  this $\Gamma$ compare to the imbalance recovered in the
  best-balanced dataset in (a)?  Are these robustness tests convincing
  that conditional exchangeability holds?


\item[d.] Repeat the analysis in part (a), this time using the {\em
    level} of Labour party vote support in 1997 (rather than change in
  vote support). Is this estimate consistent with the one recovered in
  (a)?  Is this causal estimate more or less persuasive than the
  difference-in-difference estimate you recovered in (a)?   Overall, what do we learn about the effects of media from this
  analysis?  





%\item[e.] In this study, does SUTVA limit what we can say? If so, how?
%  If not, why not?  

\item[e.] [BONUS QUESTION] Fully replicate the regression analysis in
  Table 1A (excluding the 1992 instrument column), on both the {\em
    level} and {\em change} in party vote support.  That is, do the
  bivariate analysis, the exact matching on the same covariates used
  by Ladd and Lenz, and the GenMatch analysis on the same coveriates,
  and also perform linear adjustment on each matched data set.  Can
  you replicate the table exactly? If not, which parts can you
  replicate exactly? How confident are you that this analysis is
  recovering an unbiased estimate of the persuasive effect of media on
  vote choice behavior? Does replicating the analysis change your
  assessment?







\end{itemize}

\end{document}


\item[a.] Create a loss function that ensures that GenMatch will not return a matched data set with worse balance on any variable in BalanceMatrix than the balance obtained by your matching method used in part a—as judged by eQQ-plots and difference of means. Do this so that this property holds by design—i.e., it holds regardless of the dataset used. In order to make this happen, you will have to both write a custom loss function (you may alter the one created in question “c” or write a new one), and provide GenMatch with starting.values so that it starts with the pscore matched dataset. Report your balance statistics after using this loss function.

\item[a.] Pick the matching method that produced, in your judgement, the best
balance. Estimate treatment effects and report them. How do they
differ from the reported estimates?

\item[a.] Select a set of covariates to condition on. Be sure to consider if any higher order terms and interactions are appropriate. Using these variables, perform Mahnolobis distance matching on a propensity score and “orthogo- nalized” covariates, with ATT as your estimand. Report your balance statistics, preferably using a plot.

\item[a.] Now using the same set of covariates (propensity score and “orthogonalized” covariates), use GenMatch to generate weights that optimize balance. Use the default setting for the loss function, but feel free to adjust other parameters of the function. Present balance before and after matching.

\item[a.] Now find balance with GenMatch using your own loss function. Explain the logic behind your choice of loss function. You may want to prioritize important selection variables, such as age. Present balance statistics after matching.
1

\item[a.] Create a loss function that ensures that GenMatch will not return a matched data set with worse balance on any variable in BalanceMatrix than the balance obtained by pscore matching—as judged by eQQ-plots and difference of means. Do this so that this property holds by design—i.e., it holds regardless of the dataset used. In order to make this happen, you will have to both write a custom loss function (you may alter the one created in question “c” or write a new one), and provide GenMatch with starting.values so that it starts with the pscore matched dataset. Report your balance statistics after using this loss function.

\item[a.] Pick the matching method that produced, in your judgement, the best
balance. Estimate treatment effects and report them. How do they
differ from the reported estimates?
\end{itemize}




\end{document}


For this problem, you will perform several matching exercises using
the ``Fox News Effect'' data.  The unit of observation are towns in
the US, and the treatment under study is the availability of Fox News
during the 2000 election season. The outcome (\texttt{reppresfv2p00m96}) is the
change in the Republican presidential vote share between 1996 and
2000. The dataset for this assignment only includes those towns with
pre-treatment outcome data, i.e. the change in the Republican
presidential vote share between 1988 and 1992 (\texttt{reppresfv2p92m88}). The
treatment indicator (\texttt{foxnews2000}) has been defined as equal to one if
the town’s cable system carried the Fox News network before the 2000
election. The dataset includes a set of demographic covariates from
the 2000 and 1990 census

\begin{itemize}
\item[a.] Select a set of covariates to condition on. Be sure to
  consider if any higher order terms and interactions that are
  appropriate. Using these variables, perform Mahnolobis distance
  matching on a propensity score and ``orthogonalized'' covariates
  (simultaneously),
  with ATT as your estimand. Report your treatment effect estimates
  and your balance statistics, either in a table or a plot. Is it
  similar to the estimate reported in the original paper, in
  particular the estimate reported in columns 6 and 7 in Table IV?
   \item[b.] Now with the same set of covariates, use GenMatch to
     generate weights that optimize balance. Use the default setting
     for the loss function, but feel free to adjust other parameters
     of the function. Present balance before and after matching, as
     well as your effect estimates. How is your estimate different
     from the results reported in the paper and your findings in part
     a?
   \item[c.] Now match using either the method from part (a) or the
     method from part (b), using only demographic covariates. Estimate
     the ``treatment effect'' of the introduction of Fox news prior to
     the 2000 election on pre-treatment outcome of the change in
     Republican presidential vote share between 1988 and 1992. This is
     known as a ``placebo test''. Can you recover a 0 ATT estimate using
     only demographic covariates as the conditioning set?
\end{itemize}






\begin{itemize}
\item[a.] Select a set of covariates to condition on. Be sure to consider if
any higher order terms and interactions are appropriate. Using these
variables, perform Mahalanobis distance matching, with ATT as your
estimand. Report your balance statistics.

\item[a.] Now using the same set of covariates, use GenMatch to generate weights that optimize balance. Use the default setting for the loss function, but feel free to adjust other parameters of the function. Present balance before and after matching.

\item[a.] Now find balance with GenMatch using your own loss function. Explain the logic behind your choice of loss function. You may want to prioritize important selection variables, such as age. Present balance statistics after matching.

\item[a.] Create a loss function that ensures that GenMatch will not return a matched data set with worse balance on any variable in BalanceMatrix than the balance obtained by your matching method used in part a—as judged by eQQ-plots and difference of means. Do this so that this property holds by design—i.e., it holds regardless of the dataset used. In order to make this happen, you will have to both write a custom loss function (you may alter the one created in question “c” or write a new one), and provide GenMatch with starting.values so that it starts with the pscore matched dataset. Report your balance statistics after using this loss function.

\item[a.] Pick the matching method that produced, in your judgement, the best
balance. Estimate treatment effects and report them. How do they
differ from the reported estimates?

\item[a.] Select a set of covariates to condition on. Be sure to consider if any higher order terms and interactions are appropriate. Using these variables, perform Mahnolobis distance matching on a propensity score and “orthogo- nalized” covariates, with ATT as your estimand. Report your balance statistics, preferably using a plot.

\item[a.] Now using the same set of covariates (propensity score and “orthogonalized” covariates), use GenMatch to generate weights that optimize balance. Use the default setting for the loss function, but feel free to adjust other parameters of the function. Present balance before and after matching.

\item[a.] Now find balance with GenMatch using your own loss function. Explain the logic behind your choice of loss function. You may want to prioritize important selection variables, such as age. Present balance statistics after matching.
1

\item[a.] Create a loss function that ensures that GenMatch will not return a matched data set with worse balance on any variable in BalanceMatrix than the balance obtained by pscore matching—as judged by eQQ-plots and difference of means. Do this so that this property holds by design—i.e., it holds regardless of the dataset used. In order to make this happen, you will have to both write a custom loss function (you may alter the one created in question “c” or write a new one), and provide GenMatch with starting.values so that it starts with the pscore matched dataset. Report your balance statistics after using this loss function.

\item[a.] Pick the matching method that produced, in your judgement, the best
balance. Estimate treatment effects and report them. How do they
differ from the reported estimates?
\end{itemize}


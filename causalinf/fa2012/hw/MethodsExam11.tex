\documentclass[fleqn,titlepage,12pt]{article}
\usepackage{hyperref,setspace} 
\usepackage{dcolumn} \usepackage[longnamesfirst]{natbib}
\usepackage{comment}
\usepackage{epsfig,rotate}
\usepackage[mdyyyy]{datetime}

%\bibliographystyle{info} 
%\bibpunct{(}{)}{;}{author-year}{}{,}

\newcommand{\MyPerp}{\perp \! \! \! \perp}

\setlength{\oddsidemargin}{0in}
\setlength{\evensidemargin}{0in}
\setlength{\topmargin}{-.26in}%{0in}
%\setlength{\topmargin}{0.05in}
\setlength{\headheight}{0in}
\setlength{\headsep}{0in}
\setlength{\textwidth}{6.5in}
%\setlength{\textheight}{8.987in}%{8.9in}
\setlength{\textheight}{9in}

\hypersetup{
%  colorlinks=true,
  bookmarksopen, pdftitle={}, pdfsubject={}, pdfauthor={Jasjeet S. Sekhon},
%  linkcolor=cyan,               
%  citecolor=cyan,               
%  pagecolor=cyan,
%  urlcolor=cyan
}

\doublespacing

% change section to large and subsection to normalsize font
\makeatletter
\renewcommand{\section}{\@startsection{section}{1}{0em}{\baselineskip}{0.5\baselineskip}{\large\bfseries\large}}
\renewcommand{\subsection}{\@startsection{subsection}{0}{0em}{\baselineskip}{0.5\baselineskip}{\normalfont\bfseries\normalsize}}
\makeatletter
\newcommand{\vs}{\vspace{-\baselineskip}}
\newcommand{\vvs}{\vspace{-.4cm}}

\newcommand{\mr}{\mathrm}

% fonts
%\usepackage{mathtime}
%\usepackage{mathptmx}
%\usepackage{mtpro2}

\usepackage{amsfonts, amsmath, amssymb}
\usepackage{times}



%dcolumn column types
\newcolumntype{.}{D{.}{.}{-1}}
\newcolumntype{d}[1]{D{.}{.}{#1}}

\begin{document}

\begin{center}
  \textbf{\Large{Quantitative Methods Exam}}\\
  UC Berkeley $\cdot$ Travers Department of Political Science \\
  Summer 2011
\end{center}

There are three section to this exam: quantitative Methods, formal
modeling, and Data and Design. Some of the questions are easier than
others.  Please do your best to answer them; partial grades will be
given.  You do not need to answer every question perfectly to do well
on the exam.  If you need to make additional assumptions not stated in
a given question in order to obtain an answer, please clearly state
these assumptions and the reasons for them.

\section{Quantitative Methods}

\begin{enumerate}
\item Let $X$ and $Y$ be random variables with finite means. Prove the law of iterated expectations, $E(E(Y|X)) = E(Y)$. 

\item Consider a simple random sample of $N$ observations of $Y$ and $X$ from some joint distribution. 
\begin{enumerate}
\item For the least squares regression of sample $y$'s on sample $x$'s, $\widehat{y} = \widehat{\alpha} + \widehat{\beta} x$, derive the slope $\widehat{\beta}$ and intercept $\widehat{\alpha}$ by minimizing the mean squared error of the regression.
\item Show that this solution implies that sample $x$'s are uncorrelated with the residuals.
\end{enumerate}

\item Suppose $Y = \alpha + \beta x + \varepsilon$, where $\varepsilon$ is a random variable with $E(\varepsilon) = 0$ and $Var(\varepsilon|x) = \sigma^2$ for all $x$. 
\begin{enumerate}
\item Suppose that $\alpha$ and $\beta$ are estimated by ordinary least squares regression. What is the expected value of $\widehat{\beta}$ in repeated simple random sampling?
\item Identify a condition such that $E(\widehat{\beta}) = \beta$, and interpret its substantive meaning.
\end{enumerate}

\item Prove that if one conditions non-parametrically (e.g., by
  matching) on the true propensity score, asymptotically, the observed
  covariates, $X$, will be balanced between treatment and control
  groups, where the propensity score is a function of $X$.

\item In the potential outcomes framework, the causal effect of
  treatment $T \in \{ 0,1\}$ on unit $i$ is $\tau_{i} =
  Y_{i}(1)-Y_{i}(0)$, where $Y_{i}(1)$ denotes the potential outcome
  of $i$ under treatment and $Y_{i}(0)$ the potential outcome under
  control. In a regression discontinuity (RD) design, if the potential
  outcomes are distributed smoothly at the cut-point, the design
  estimates the average causal effect of treatment at the cut-point,
  $Z_{i} = c$:
\begin{equation}
  \tau_{RD} \equiv \mathbb{E}[Y_{i}(1) - Y_{i}(0) | Z_{i} = c] = 
  \lim_{Z_{i} \downarrow c}\mathbb{E}[Y_{i}(1) | Z_{i} = c] -
  \lim_{Z_{i} \uparrow c}\mathbb{E}[Y_{i}(0) | Z_{i} = c].
  \label{eq:rd}
\end{equation}
This equation is for the sharp RD design. In Lee's (\citeyear{leeRD})
application to U.S. House election it has been shown that baseline
covariates, such as campaign contributions, are not balanced in a
neighborhood of the cutpoint. What RD assumptions does this violate if
any?

Formally define a fuzzy RD design, as done in Equation
\ref{eq:rd}. Could a fuzzy design overcome the problems found in using
RD with U.S. House elections? Why or why not?

\end{enumerate}

\newpage
\section{Formal modeling}

Answer 2 of the following 5 questions. Each question has equal weight.

\begin{enumerate}
\item Discordia is in the throes of what has become known as the
Bougainvillea Revolution. The Plaza of\ Perpetual Passivity in the center of
the capital city is packed with protesters calling for the ouster of the
long-entrenched President. The President has ordered the military to clear
the square. The military's leaders must choose whether to clear the square
by shooting tear gas canisters or by shooting bullets. The next day, the
protest leaders will have to decide whether to escalate the protests to a
nationwide uprising or to back down. If the movement backs down the game
ends. If the protests escalate, the military will then have to decide
whether to support the President in instituting a nationwide crackdown or to
change sides and place the President under arrest. The protest movement is
uncertain whether the military is fundamentally loyal to the President or if
it is sympathetic to the regime change. If the military is loyal, the cost
it incurs from clearing the square with tear gas is $0$ and its cost of
clearing with bullets is $2$. If it is sympathetic these costs are $0$ and $6
$. If the movement is cleared from the square with tear gas, the movement
incurs a loss of $d$, and if it is cleared with bullets it incurs a loss of $%
D>d$. If there is no escalation, the military gets a payoff of $5$.
Supporting the crackdown gives the military a benefit of $2$ if it is loyal
to the President but costs it $2$ if it is sympathetic to the movement. If
it does not support the crackdown following escalation by the movement it
incurs no costs or benefits. At the start of the interaction the protest
movement believes the military is loyal with probability $\alpha =\frac{1}{2}
$. \vspace{0in}The protest movement can be treated as a unitary
decision-maker, as can the military. The President makes no decisions in
this game. The movement's payoff for backing down is $0$. The movement's
payoff for escalation followed by a crackdown is $-100$. The movement's
payoff for escalation that does not result in a crackdown is $50$.

\begin{enumerate}
\item Write out the game tree for this interaction.

\item Let $\hat{\alpha}$ designate the movement's beliefs about the
probability the military is loyal following the initial clearing of the
square. How high must $\hat{\alpha}$ be to ensure that the movement will not
choose to escalate?

\item Is there a separating equilibrium in which only a loyal military will
clear the square with bullets?

\item Is there an equilibrium in which the military clears with bullets
regardless of its loyalty?

\item Is there an equilibrium in which the military clears with tear gas
regardless of its loyalty?

\item In what ways do the values of $d$ and $D$ affect equilibrium play?
\end{enumerate}

\item Continuing with the analysis of the Revolution game, do the following:

\begin{enumerate}
\item Find the mixed-strategy equilibrium of this game in which the
military, if loyal, uses tear gas with probability $\mu $, and the movement
backs down after tear gas is used with probability $\lambda $.

\item Let $\alpha =\frac{1}{4}$. Characterize a pooling Perfect Bayesian
Equilibrium of this game that would violate the Intuitive Criterion
(invented by Cho and Kreps) and demonstrate that it does so.

\item Now suppose that the military can clear the square with any level of
brutality $b\geq 0$, incurring cost $c\left( b\right) =\frac{b}{\theta }$,
where $\theta =1$ if the military is sympathetic and $\theta =2$ if the
military is loyal. Construct a pooling equilibrium in which the military
chooses a strictly positive level of brutality $b^{\ast }>0$ irrespective of
its type. Make sure to specify the opposition's beliefs for all possible
off-path behavior by the military.
\end{enumerate}

\item Two candidates, $A$ and $B$, each propose (and can commit to) a
platform with a level of taxes $\tau $ and a level of public goods $g$. The
government's budget constraint is $\tau \geq g+r$, where $r$ is rents from
holding office. Candidate $j$'s utility is $p_{j}r$, where $p_{j}$ is
candidate $j$'s probability of winning. Losing candidates receive utility of 
$0$. There is a continuum of citizens indexed $i$. A citizen $i$ will vote
for candidate $A$ if 
\begin{equation*}
1-\tau _{B}+h\left( g_{A}\right) >1-\tau _{B}+h\left( g_{B}\right) +\sigma
_{i}
\end{equation*}%
and will vote for $B$ otherwise. $\sigma ^{i}$\vspace{0in} is distributed
uniformly on $\left[ -1,1\right] $. $h\left( g\right) =\ln g$.

\begin{enumerate}
\item Solve for the equilibrium proposed levels of taxes and rents. Hint:
Solve for the optimal proposal $g_{A}$ first.

\item Explain the following in a way that would be understandable to someone
who has never studied game theory:

\begin{enumerate}
\item What does $h\left( g\right) $ represent?

\item What does $\sigma _{i}$ represent?

\item Why do $g$ and $r$ take the values you found, and what does the
equilibrium tell us about politics?
\end{enumerate}

\item What assumption(s) might be problematic or too unrealistic, and how
would you conjecture that the results would differ if these assumption(s)
changed?
\end{enumerate}

\item Consider a game between a dictator $D$ and an opposition $O$. Let
policy be a unidimensional variable $x\in 
%TCIMACRO{\U{211d} }%
%BeginExpansion
x\mathbb{R}
%EndExpansion
$. The dictator's ideal point is $0$ and the opposition's ideal point is $1$%
, which also happens to be the preferences of the median person in this
country. Ruling the country generates rents in the amount $x$, because
ordinary people in the country are more motivated to produce when policy is
closer to their ideal. Both players receive additively separable utility
that derives from the distance of policy to their ideal point and from
whatever share of the rents they receive. \ Thus, with a policy of $x$ and a
share of rents $s\in \left[ 0,1\right] $ granted to the opposition, the
dictator has a payoff of $\left( 1-s\right) x$ $-x^{2}$ and the opposition
has a total payoff of $sx-\left( 1-x\right) ^{2}$. The game has two moves.
First, the dictator offers the opposition a policy proposal $x\in \left[ 0,1%
\right] $ and a share of the rents $s\in \left[ 0,1\right] $. Second, the
opposition either accepts or rejects this offer and attempts to overthrow
the dictator, succeeding with probability $q\in \left( 0,1\right) $. Whoever
wins that conflict can choose any policy and keep all the rents.
Furthermore, if the dictator wins he imposes a punishment of $L$ on the
opposition, where $L=1$ is exogenously fixed. Thus, the utility of the
opposition for a successful overthrow is $\max_{x}x-\left( 1-x\right) ^{2}$,
and the utility of a failed overthrow is $-1$. The utility to the dictator
of a successful overthrow is $0$ and the utility of a failed overthrow is $%
\max_{x}x$ $-x^{2}$.

\begin{enumerate}
\item What is the expected utility to the opposition of an overthrow attempt?

\item Characterize the set of offers $\left( x,s\right) $ that will be
accepted by the opposition.

\item What is the expected utility to the dictator if the opposition
attempts an overthrow?

\item If $q=\frac{1}{4}$ what is the equilibrium proposal $\left( x^{\ast
},s^{\ast }\right) $ from the dictator? \ Will it be accepted?

\item If $q=\frac{3}{4}$, what is the equilibrium proposal $\left( x^{\ast
},s^{\ast }\right) $ from the dictator? \ Will it be accepted?

\item Find the conditions on $q$ necessary for the dictator to make an offer 
$s^{\ast }>0$.
\end{enumerate}

\item \vspace{0in}How important is empirical testing to validate a formal
model? Is a formal model with no testable implications valuable? To what
extent are any formal models testable? How does the endeavor of testing a
formal model differ from other empirical work? Discuss with reference to
specific published models. You may restrict attention to one substantive
area/subfield or range more widely in your answer.
\end{enumerate}

\newpage
\section{Data and Design}

This portion of the exam consists of two different sections, both of
which are equally weighted.  Please make sure you have read all of the
articles in the reference list carefully. Most of these articles were
previously assigned.


\subsection{Fearon and Laitin 2003}

This section is based on Fearon and Laitin (2003). This is a prominent
article which examines if ethnic and religious antagonisms are the
cause of the civil wars which proliferated as the cold war ended.
Please read the article before proceeding.

Table 1 of this article can be replicated using the following R file:
\url{http://sekhon.berkeley.edu/qe/FearonLaitin_replication_qe1.R}. The
output of this R file can be found here:
\url{http://sekhon.berkeley.edu/qe/FearonLaitin_replication_qe1.Rout}. Reading
the R source code and comparing the output with Table 1 suffices to
define the variables in the dataset.

Please answer the following questions:
\begin{enumerate}
\item Many researchers have questioned the findings of Fearon and
  Laitin.  One concern has been that the countries which undergo civil
  wars are very different from those which do not so there is a lack
  of common support. The parametric models which Fearon and Laitin use
  have to rely upon extrapolation.  So, let's use matching to see if
  we recover the same substantive results as Fearon and Laitin.

  \begin{enumerate}
  \item Take Model 1 from Table 1, and estimate the causal effect of     ethnic fractionalization using matching. The estimand is ATT. For     this question, match on all of variables which Fearon and Laitin     include in their Model 1.  In order to use matching, make the     ethnic fractionalization variable discrete.  For this question:     call control all countries with ethnic fractionalization levels     below the median of this variable, and call treated all countries     above the median.  Note that we only need to match     \textit{countries} since their level of ethnic fractionalization     does not change over time.  After matching, consider what to do     with the time dimension of the data.  \textit{Hint:} consider how     matching was used in both \citet{GordonHuber2007} and     \citet{GalianiGertlerSchargrodsky2005}.  Both articles have
    data across units and over time but still use matching.
    
  \item Redo question 1, but make ethnic fractionalization discrete     but not dichotomous---e.g., make it have at least three different     categories.  Discretize the variable with an eye towards     maximizing the chances of finding a significant effect.  Please     redo the matching.
    
  \item Analyze the research design used in this article. What
    identification strategy is used? What are the inferential
    problems?  Is there a way forward?  What is the probative value of
    the evidence for \textit{any} causal question?
  \end{enumerate}

\item Using this dataset, someone has estimated an OLS where the
  dependent variable is per capita income.  In particular, the analyst
  estimates the following model which is based on Model 1 in Table 1
  of \cite{FearonLaitin2003} (but with a different dependent
  variable).  The dependent variable is per capital income at time $t$
  (variable name: $gdpen$).  And this is regressed on: per capital
  income at $t-1$ ($gdpenl$), war at $t-1$ ($warl$), log(population)
  at $t-1$ ($lpopl1$), log(\% mountainous) ($lmtnest$), noncontigous
  state ($ncontig$), oil exporter ($Oil$), new state ($nwstate$),
  instability at $t-1$ ($instab$), democracy at $t-1$ ($polity2l$),
  ethnic fractionalization ($ethfrac$) and religious fractionalization
  ($relfrac$).  Note that the time-series cross-section nature of the
  dataset is maintained.

  The following R code estimates this model:
  \url{http://sekhon.berkeley.edu/qe/FearonLaitin_income_qe1.R}.

  \begin{enumerate}
  \item Please provide bootstrap confidence intervals for every
    parameter of this least squares model in the manner you think is
    most appropriate.  For this question write your own bootstrap
    code---e.g., you cannot use functions in the \texttt{boot} and
    related libraries.  Given the complex structure of the dataset,
    the \texttt{boot} function will not work anyways.
  \item Under what assumptions does your bootstrap provided correct
    coverage?
  \item Please provide bootstrap confidence intervals for this
    regression model using a ``better bootstrap''---e.g., a bootstrap
    based on a pivot statistic like the studentized version of the
    difference of means bootstrap (case 3 in the lecture notes).  If
    you have already done this for (a), ignore this question.
  \item Please provide bootstrap confidence intervals for the logistic
    regression Model 1 in Table 1 of Fearon and Laitin.
  \end{enumerate}
\end{enumerate}

\subsection{Regression Discontinuity and Close House Elections}

This question concerns \cite{leeRD} and \citet{CaugheySekhon2010}. The
following URL provides David Lee's dataset: 
\url{http://sekhon.berkeley.edu/stuff1/LeeRDdata.zip}

Note that Lee's dataset is different from that used by
\citet{CaugheySekhon2010}. For example, it includes far fewer
variables and it contains some errors and missing values that were
imputed. Both the Caughey and Sekhon
\href{http://sekhon.berkeley.edu/papers/CaugheySekhonRD.pdf}{article}
and an
\href{http://sekhon.berkeley.edu/papers/RDappendix.pdf}{appendix} that
includes details about their dataset are available on Sekhon's
webpage.

Questions:
\begin{enumerate}
\item Please use David Lee's replication files to replicate the tables
  and figures in Lee's article.

\item To the extent possible please use Lee's dataset to replicate the
  key tables and figures in \citet{CaugheySekhon2010}. Which key
  findings differ between \citet{leeRD} and \citet{CaugheySekhon2010}
  because of data differences and which findings are consistent even
  if one uses Lee's original dataset?
\end{enumerate}

\newpage
\pdfbookmark[1]{References}{sec:references}
\bibliographystyle{chicago}
\bibliography{causalinf}


\end{document}

%%% Local Variables: 
%%% mode: latex
%%% TeX-master: t
%%% End: 

% LocalWords:  OLS Neyman UC researcher's equilibria cutpoint BNE PBE subgame
% LocalWords:  partitional subintervals subinterval maximizers SPNE textbf cdot
% LocalWords:  Fearon Laitin GalianiGertlerSchargrodsky2005 gdpen gdpenl warl
% LocalWords:  lpopl1 lmtnest noncontigous ncontig nwstate ethfrac relfrac eq
% LocalWords:  texttt studentized Caughey pdfbookmark causalinf widehat widehat
% LocalWords:  widehat varepsilon equiv mathbb newpage vspace Kreps geq ln sx
% LocalWords:  TCIMACRO

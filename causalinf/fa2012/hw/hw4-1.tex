
\documentclass{article}
\usepackage{amsmath}
\usepackage{amsthm}
\usepackage{color}
\usepackage{setspace}
\usepackage{fullpage}
\usepackage[round]{natbib}
\usepackage[utf8]{inputenc}
\usepackage{amssymb} 

% Setup for fullpage use
\usepackage{fullpage}

% Uncomment some of the following if you use the features
%
% Running Headers and footers
%\usepackage{fancyhdr}
% Multipart figures
%\usepackage{subfigure}
% More symbols
%\usepackage{amsmath}
%\usepackage{amssymb}
%\usepackage{latexsym}
% Surround parts of graphics with box
\usepackage{boxedminipage}

% Package for including code in the document
\usepackage{listings}

% If you want to generate a toc for each chapter (use with book)
\usepackage{minitoc}

% This is now the recommended way for checking for PDFLaTeX:
\usepackage{ifpdf}

%\newif\ifpdf
%\ifx\pdfoutput\undefined
%\pdffalse % we are not running PDFLaTeX
%\else
%\pdfoutput=1 % we are running PDFLaTeX
%\pdftrue
%\fi
\usepackage{natbib} 
\usepackage{times} 
\usepackage{setspace}
\usepackage{subfigure}

\usepackage{hyperref} 

\newcommand\independent{\protect\mathpalette{\protect\independenT}{\perp}} 
\def\independenT#1#2{\mathrel{\rlap{$#1#2$}\mkern2mu{#1#2}}} 



\ifpdf 
\usepackage[pdftex]{graphicx} \else 
\usepackage{graphicx} \fi 

\title{PS C236A / Stat C239A \\ Problem Set 4 \\ Due: Oct. 29, 2012}
\date{}

\begin{document}

\maketitle
\vspace{-4em}
\section*{Instructions}
This assignment is due {\bf 4 pm Monday, Oct. 29.}  You may submit your
analytical work either electronically or in paper form.  Electronic
versions must be sent as a .pdf to
$<$\texttt{jahenderson[at]berkeley.edu}$>$. Paper copies should be
placed in my mailbox in 210 Barrows.  For the computing portion of the
assignment, you \underline{must} submit a fully executable version of
all .R code, along with any data used in the code (excepting that
provided through the course webpage) to the email above.  All files
for each assignment sent electronically should be included in one
omnibus email, with the subject line containing the course and
homework number, and your last name (e.g., PS239A/STAT236A: HW4 - McConnell).\\


\noindent You are encouraged to work together in groups to complete
the assignments. However, you must hand in your own individual
answers. Photocopies and other reproductions of someone else’s answers
are not acceptable. Please also list the names of everyone with
whom you have collaborated on this assignment.





\paragraph{Problem 1}
  Eggers and Hainmueller (2009) estimate the 
    LATE of just barely winning (losing) an election contest on wealth at death. 
    Their estimate compares the average wealth at death of
    all winners of contests between 1950-1970
    to the average wealth over all losers over that time.   There were seven general elections over that period. 

\vspace{1em}
\noindent 
 Consider another estimator of the LATE that takes the average difference 
   between the winners and the losers participating in a given general election, 
   and takes the weighted average of these differences---weights are:
   $$
     \frac{\#\text{near-winners and near-losers in that general election}}
      {\text{total number of near-winners and near-losers across all elections}}.
   $$
   Each near-win (near-loss) candidate is only included in this average once, according
   to the contest in which that candidate is ultimately classified as near-win or near-loss.
   
   \begin{itemize}
     \item[a.]  Write out mathematical expressions for both of these
       estimators (i.e., Eggers and Haimueller (2009) and the weighted
       estimator).
     \item[b.]
       Show that, if the number of contests within each general election is the same, 
       and the number of near winners is equal to the 
       number of near losers in every general election, then
       the two estimates are the same.
     \item[c.]  Suppose the same assumptions as part (b), 
       except that the 1950 general election 
       had twice as many contests as all other elections
       and had twice as many near-winning candidates as near-losing candidates.       
       In general, are the two estimators the same?  Or are they different?
   \end{itemize}   



\paragraph{Problem 2}

   Suppose that 1,000,100 students take an exam.
   Students can score an integer number of points between 0 and 10,000 
   on the exam:
   the set of possible test scores is $\{0, 1, \ldots, 9,999, 10,000\}$.
   Miraculously, for each possible point value of the exam, 
   exactly 100 students score that many points.    
   Those students that score above a certain threshold on the exam
   (usually around 5,000 points) receive 
   a scholarship.
   We are interested in estimating the LATE of receiving a scholarship
   on future earnings, for those students that score 5,000 points.
   
\vspace{1em}
\noindent   Let $Y_i(1)$ and $Y_i(0)$ denotes the future earnings of student $i$
   when that student receives (or does not receive) the scholarship.        
   Let $s_i$ denote the test score of student $i$.   
   Let $c_i =1$ if student $i$ receives a scholarship
   and $c_i = 0$ if that student does not receive a scholarship
   Suppose that the distribution of $IQ$ for those students scoring 4,995 points 
   is the same for students scoring 4,996 points, 4,997 points, 
   $\ldots$, 5,004 points, 5,005 points.
   We want to estimate:
   $$
     \text{LATE} = E(Y_i(1) - Y_i(0) | s_i = 5,000)
   $$
   \begin{itemize}
     \item[a.] 
       Suppose that all students that score above 5,000 points receive a scholarship, and
       all students that score below 5,000 points do not receive a scholarship.
       For students that score exactly 5,000 points, half will be randomly selected to receive
       a scholarship, and half will not receive that scholarship.
       Give an unbiased estimate of the LATE.
       Does this estimate require smoothness of $E(Y_i(1))$ and $E(Y_i(0))$ 
       at the $s_i = 5,000$ threshold?
     \item[b.]
       Suppose for parts (b), (c), and (d) that all students scoring 5,000 points or above
       receive a scholarship.
       
       Suppose that future earnings are determined by the following model:
       $$
         Y_i = \alpha + \beta_1 s_i + \beta_2c_i 
                +\beta_3s_ic_i + \beta_4IQ_i + \epsilon_i
       $$
       where $\epsilon_i$ has expectation $0$ and variance $\sigma^2$.
       Give an unbiased estimate of the LATE.
       Under this model, are $E(Y_i(1))$ and $E(Y_i(0))$ smooth at the 
       $s_i = 5,000$ threshold?
       Are these assumptions stronger than those 
       required for regression discontinuity?
     \item[c.]
       Suppose that, for scores between 4,995 and $5,000-\epsilon$ points, 
       future earnings are determined by
       $$
         Y_i = 50,000 + 5,000(s_i - 5,000) + \epsilon_i
       $$
       and for scores between 5,000 and 5,005 points, future earnings are determined by
       $$
         Y_i = 80,000 - 6,000(s_i - 5,000) + \epsilon_i.
       $$
       Give an unbiased estimate of the LATE.
       In what sense are these assumptions stronger or weaker than those in (b)?
       Are these assumptions stronger than those required for regression discontinuity?
     \item[d.]
       Suppose some students that would ordinarily receive low test scores cheat off of good students.
       All students that cheat score 5,000 points or above, 
       with some students scoring exactly 5,000 points.
       Assume that, had the students not cheated, 
       the assumptions for regression discontinuity analysis would hold. 
       After the students cheat, do these assumptions still hold?
       Why or why not?
       What if all of the cheating students scored above 5,002 points?
  \end{itemize}

\paragraph{Problem 3} 

This question will involve the RD design controversy in  Lee (2008)
and Caughey and Sekhon (2011).  The
following URL provides David Lee's dataset: 
\url{http://sekhon.berkeley.edu/stuff1/LeeRDdata.zip}

\vspace{1em}
\noindent Note that Lee's dataset is different from that used by
 Caughey and Sekhon (2011). For example, it includes far fewer
variables and it contains some errors and missing values that were
imputed. Both the Caughey and Sekhon
\href{http://sekhon.berkeley.edu/papers/CaugheySekhonRD.pdf}{article}
and an
\href{http://sekhon.berkeley.edu/papers/RDappendix.pdf}{appendix} that
includes details about their dataset are available on Sekhon's
webpage.


\begin{enumerate}
\item[a.] Use David Lee's replication files to replicate the tables
  and figures in Lee's article.

\item[b.] To the extent possible, use Lee's dataset to replicate the
  key tables and figures in Caughey and Sekhon (2010). Which key
  findings differ between Lee (2008) and Caughey and Sekhon (2010)
  because of data differences and which findings are consistent even
  if one uses Lee's original dataset?
\end{enumerate}
%Recall in section that because choice of the bandwidth in an RD design is somewhat
%arbitrary, Imbens (2009) recommends combining local linear regression
%with a ``cross-validation'' procedure for choosing $h$. The idea
%is the following: Consider an observation $i$. To
%see how well a linear regression with a bandwidth $h$ fits the data, we
%run a regression with observation $i$ left out and use the estimates to
%predict the value of $Y$ at $X = x_i$. To emulate the fact that RD
%estimates are based on regression estimates at the boundary, the
%regression is estimated using only observations with values of $X$ on
%the left of $X_i (X_i - h \leq X < X_i)$ for observations on the left of the
%cutpoint $(X_i < c)$. For observations on the right of the cutoff point
%$(X_i \geq c)$, the regression is estimated using only the observations with
%values of $X$ on the right of $X_i (X_i < X \leq X_i + h)$.  After repeating
%this procedure for each and every observation, we will have a
%collection of predicted values of Y that can be compared to the actual
%values of $Y$ . The optimal bandwidth can be picked by choosing the
%value of h that minimizes the mean square of the difference between
%the predicted and actual value of $Y$.  Formally, let $\hat{Y}(X_i)$ be the
%predicted value of Y obtained using the regressions described
%above. The cross validation criterion is defined as
%$$CV_Y(h)=\frac{1}{N}\sum_{i=1}^N (Y_i -\hat{Y}(X_i))^2$$
%with the corresponding cross-validation choice for the bandwidth
%$$ h^{opt}_{CV}=\text{arg min}_h CV_Y(h)$$
\begin{enumerate}
\item[c. ] Use the cross-validation procedure described in section to
  calculate $h_{opt}$ for a trimmed subset of the data compiled by Caughey and Sekhon
  (2011).  Select a range of possible $h$ to check in your procedure
  (say $h = .01, .02, .03, ..., .3.)$ 
\vspace{1em}

For a more detailed discussion of this method, see: \url{http://www.econ.ubc.ca/lemieux/papers/designs.pdf}

\item[d.] Using local linear regression, estimate the local average treatment effect and its associated standard error with your $h_{opt}$ calculated in part (a).

\item[Bonus] On the same data, show whether or not previous {\em incumbent} win margin is
  smooth through the cut-point in the design using McCrary's test of
  smoothness outlined in his 2008 paper:
  \url{http://emlab.berkeley.edu/~jmccrary/mccrary2006_DCdensity.pdf}.
  Now show this for {\em Democratic candidate} previous vote margin.
\end{enumerate}



\end{document}

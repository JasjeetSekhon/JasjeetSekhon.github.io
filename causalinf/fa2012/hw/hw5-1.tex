
\documentclass{article}
\usepackage{amsmath}
\usepackage{amsthm}
\usepackage{color}
\usepackage{setspace}
\usepackage{fullpage}
\usepackage[round]{natbib}
\usepackage[utf8]{inputenc}
\usepackage{amssymb} 

% Setup for fullpage use
\usepackage{fullpage}

% Uncomment some of the following if you use the features
%
% Running Headers and footers
%\usepackage{fancyhdr}
% Multipart figures
%\usepackage{subfigure}
% More symbols
%\usepackage{amsmath}
%\usepackage{amssymb}
%\usepackage{latexsym}
% Surround parts of graphics with box
\usepackage{boxedminipage}

% Package for including code in the document
\usepackage{listings}

% If you want to generate a toc for each chapter (use with book)
\usepackage{minitoc}

% This is now the recommended way for checking for PDFLaTeX:
\usepackage{ifpdf}

%\newif\ifpdf
%\ifx\pdfoutput\undefined
%\pdffalse % we are not running PDFLaTeX
%\else
%\pdfoutput=1 % we are running PDFLaTeX
%\pdftrue
%\fi
\usepackage{natbib} 
\usepackage{times} 
\usepackage{setspace}
\usepackage{subfigure}

\usepackage{hyperref} 

\newcommand\independent{\protect\mathpalette{\protect\independenT}{\perp}} 
\def\independenT#1#2{\mathrel{\rlap{$#1#2$}\mkern2mu{#1#2}}} 



\ifpdf 
\usepackage[pdftex]{graphicx} \else 
\usepackage{graphicx} \fi 

\title{PS C236A / Stat C239A \\ Problem Set 5 \\ Due: Nov. 12, 2012}
\date{}

\begin{document}

\maketitle
\vspace{-4em}
\section*{Instructions}
This assignment is due {\bf 4 pm Monday, Nov. 12.}  You may submit your
analytical work either electronically or in paper form.  Electronic
versions must be sent as a .pdf to
$<$\texttt{jahenderson[at]berkeley.edu}$>$. Paper copies should be
placed in my mailbox in 210 Barrows.  For the computing portion of the
assignment, you \underline{must} submit a fully executable version of
all .R code, along with any data used in the code (excepting that
provided through the course webpage) to the email above.  All files
for each assignment sent electronically should be included in one
omnibus email, with the subject line containing the course and
homework number, and your last name (e.g., PS239A/STAT236A: HW5 - Pelosi).\\


\noindent You are encouraged to work together in groups to complete
the assignments. However, you must hand in your own individual
answers. Photocopies and other reproductions of someone else’s answers
are not acceptable. Please also list the names of everyone with
whom you have collaborated on this assignment.




\paragraph{Problem 1}
 An alternative distance metric to the propensity score is
Mahalanobis distance. This metric reduces the multidimensional problem
of multivariate matching to a unidimensional problem. Although
Mahalanobis distance was originally developed for use with
multivariate Normal data, we often encounter covariates that are not
normally distributed. This problem will explore the implications of
these non-normal variables on this distance metric.
\vspace{1em}
\noindent 
   \begin{itemize}
     \item[a.]  When including a binary variable in a Mahalanobis distance metric,
will a binary variable with $p=1/2$ or a binary variable with $p$ near
zero be given greater weight by this distance metric? Prove why this
is true mathematically.
     \item[b.]
     How will this distance metric treat covariates with outliers? How
about covariates that have long-tailed distributions?
     \item[c.] Should we or shouldn’t we be concerned by the behavior of the
Mahalanobis distance metric for the covariate distributions described
in parts (a) and (b)? Why?
   \end{itemize}   



\paragraph{Problem 2}

For this problem, you will critique ``The Fox News Effect: Media Bias
and Voting'', by Stefano Della Vigna and Ethan Kaplan. The paper can
be found here:
\url{http://sekhon.berkeley.edu/causalinf/papers/DellaVignaFoxNews.pdf}. Please
write a page or two addressing the following questions:

   \begin{itemize}
   \item[a.]  Describe and discuss the identification strategy of the
     paper. What are the weaknesses? What parts do you find
     convincing?
   \item[b.] Explain the importance of section III.A in the article?
     Would you do it any differently?
   \item[c.] Perform the following thought experiment: hold the
     estimation procedure in section III.B constant, and assuming that
     you have access to all existing data in the US, what data would
     you include to the improve the validity of the estimates? Now do
     the reverse. Holding the data constant, discuss what parts you
     would change and add to the estimation procedures to increase
     confidence in the validity of the results.
   \item[d.] Overall, are you convinced that their conclusions are
     correct?
\end{itemize}




\paragraph{Problem 3} 


This question will analyze the following dataset:
\url{http://sekhon.berkeley.edu/causalinf/data/cross_section_wfl.csv}.
The variables are described in the following file:
\url{http://sekhon.berkeley.edu/causalinf/data/codebook.water.txt}

\vspace{1em}
\noindent
The dataset has 435 observations and was used in the article ``Water
for Life: The Impact of the Privatization of Water Services on Child
Mortality'', by S. Galiani, P. Gertler, and E. Schargrodsky (2005,
Journal of Political Economy, volume 113).  The paper is here: \url{http://sekhon.berkeley.edu/causalinf/papers/GalianiWater.pdf}.

\vspace{1em}
\noindent The units of observation are municipalities in Argentina, and the
treatment under study is the privatization of municipal water
services. All 435 municipalities in this sample had public water
services in the year 1990, but by the year 1999, 123 municipalities
had privatized their water services. Of the 123 municipalities which
privatized between 1990 and 1999, 83 municipalities did so between
1998 and 1999. The original panel structure of the dataset has been
simplified to a cross-section: for each municipality, the dataset you
will be working with has the covariates for each year between 1990 and
1999.

\vspace{1em}
\noindent
The treatment indicator has been defined as equal to one if the
municipality privatized its water services sometime between 1991 and
1999, and equal to zero if a municipality whose water services were
public in 1990 never privatized between 1991 and 1999. The outcomes of
interest are total child mortality and child mortality from infectious
parasitic diseases, i.e., water-borne diseases. Perinatal mortality
is also of interest for the reasons discussed in Galiani, Gertler, and
Schargrodsky (2005).  
% For the purposes of all questions on this exam except for the bonus
% question, when trying to find optimal balance, you can safely
% restrict yourself to the following baseline covariates and the
% nonlinear functions listed here:

\vspace{1em}
\noindent For parts (b) - (e) below, be sure to explicitly set seeds
to ensure that GenMatch recovers reproducible results,
i.e. \texttt{set.seed} in general, and in GenMatch \texttt{unif.seed}, \texttt{int.seed}.

\begin{itemize}
\item[a.] Select a set of covariates to condition on. Be sure to
  include higher order terms and interactions you think are
  appropriate. Using these variables, perform Mahalanobis distance
  matching, with the average treatment effect for the treated (ATT) as
  your estimand. Report your balance statistics before and after
  matching using \texttt{MatchBalance}.

\item[b.] With the same set of covariates, use GenMatch to generate
  weights that optimize balance. Use the default setting for the loss
  function, but feel free to adjust other parameters of the
  function. (Make sure you {\bf do not} drop any treated units in
  matching here however.)  Present balance before and after matching
  using \texttt{MatchBalance}.  Produce two QQ-Plots illustrating
  improvement in balance on one important continuous covariate before
  and after matching.

\item[c.] Estimate the ATT of privatizing
  water services between 1991 and 1999 on child mortality in 1999,
  using your matched data from part (b).  Is the ATT estimate
  statistically different from zero?  Plot the density of the unit effects of treatment on
  the treated municipalities.  Does your interpretation of the effect
  of privatization change when examining average versus unit effects?    
%  What is the most interesting summary statistic when comparing child
%  mortality across the two groups? How informative are mean
%  differences? What are the mean differences?

\item[d.] Using the same covariates from (a) and (b), rerun GenMatch, this
  time dropping at most 10\% of the treated units. Does your balance
  improve with respect to the balance you found in part (b)? Now what
  is the ATT of privatization in this matched data?  Is the ATT
  significantly different from zero?  


\item[e.] Now find the best balance with GenMatch using your own loss
  function.  In doing so, retain every other specification you used in
  GenMatch in part (d).  Explain the logic behind your choice of loss
  function. (An example loss function would maximize the median
  $p$-value from a vector of \texttt{t.test} and \texttt{ks.test}
  results). In your loss function, you may want to prioritize
  important selection variables, for instance pre-treatment mortality
  rates. Present balance statistics after matching using
  \texttt{MatchBalance}.

\item[f.] Overall, how do your results differ from those in Galiani,
  Gertler, and Schargrodsky (2005)?  In particular, are your results
  in part (b) and (c), and their published findings comparable? 

\end{itemize}

\end{document}


\item[a.] Create a loss function that ensures that GenMatch will not return a matched data set with worse balance on any variable in BalanceMatrix than the balance obtained by your matching method used in part a—as judged by eQQ-plots and difference of means. Do this so that this property holds by design—i.e., it holds regardless of the dataset used. In order to make this happen, you will have to both write a custom loss function (you may alter the one created in question “c” or write a new one), and provide GenMatch with starting.values so that it starts with the pscore matched dataset. Report your balance statistics after using this loss function.

\item[a.] Pick the matching method that produced, in your judgement, the best
balance. Estimate treatment effects and report them. How do they
differ from the reported estimates?

\item[a.] Select a set of covariates to condition on. Be sure to consider if any higher order terms and interactions are appropriate. Using these variables, perform Mahnolobis distance matching on a propensity score and “orthogo- nalized” covariates, with ATT as your estimand. Report your balance statistics, preferably using a plot.

\item[a.] Now using the same set of covariates (propensity score and “orthogonalized” covariates), use GenMatch to generate weights that optimize balance. Use the default setting for the loss function, but feel free to adjust other parameters of the function. Present balance before and after matching.

\item[a.] Now find balance with GenMatch using your own loss function. Explain the logic behind your choice of loss function. You may want to prioritize important selection variables, such as age. Present balance statistics after matching.
1

\item[a.] Create a loss function that ensures that GenMatch will not return a matched data set with worse balance on any variable in BalanceMatrix than the balance obtained by pscore matching—as judged by eQQ-plots and difference of means. Do this so that this property holds by design—i.e., it holds regardless of the dataset used. In order to make this happen, you will have to both write a custom loss function (you may alter the one created in question “c” or write a new one), and provide GenMatch with starting.values so that it starts with the pscore matched dataset. Report your balance statistics after using this loss function.

\item[a.] Pick the matching method that produced, in your judgement, the best
balance. Estimate treatment effects and report them. How do they
differ from the reported estimates?
\end{itemize}




\end{document}


For this problem, you will perform several matching exercises using
the ``Fox News Effect'' data.  The unit of observation are towns in
the US, and the treatment under study is the availability of Fox News
during the 2000 election season. The outcome (\texttt{reppresfv2p00m96}) is the
change in the Republican presidential vote share between 1996 and
2000. The dataset for this assignment only includes those towns with
pre-treatment outcome data, i.e. the change in the Republican
presidential vote share between 1988 and 1992 (\texttt{reppresfv2p92m88}). The
treatment indicator (\texttt{foxnews2000}) has been defined as equal to one if
the town’s cable system carried the Fox News network before the 2000
election. The dataset includes a set of demographic covariates from
the 2000 and 1990 census

\begin{itemize}
\item[a.] Select a set of covariates to condition on. Be sure to
  consider if any higher order terms and interactions that are
  appropriate. Using these variables, perform Mahnolobis distance
  matching on a propensity score and ``orthogonalized'' covariates
  (simultaneously),
  with ATT as your estimand. Report your treatment effect estimates
  and your balance statistics, either in a table or a plot. Is it
  similar to the estimate reported in the original paper, in
  particular the estimate reported in columns 6 and 7 in Table IV?
   \item[b.] Now with the same set of covariates, use GenMatch to
     generate weights that optimize balance. Use the default setting
     for the loss function, but feel free to adjust other parameters
     of the function. Present balance before and after matching, as
     well as your effect estimates. How is your estimate different
     from the results reported in the paper and your findings in part
     a?
   \item[c.] Now match using either the method from part (a) or the
     method from part (b), using only demographic covariates. Estimate
     the ``treatment effect'' of the introduction of Fox news prior to
     the 2000 election on pre-treatment outcome of the change in
     Republican presidential vote share between 1988 and 1992. This is
     known as a ``placebo test''. Can you recover a 0 ATT estimate using
     only demographic covariates as the conditioning set?
\end{itemize}






\begin{itemize}
\item[a.] Select a set of covariates to condition on. Be sure to consider if
any higher order terms and interactions are appropriate. Using these
variables, perform Mahalanobis distance matching, with ATT as your
estimand. Report your balance statistics.

\item[a.] Now using the same set of covariates, use GenMatch to generate weights that optimize balance. Use the default setting for the loss function, but feel free to adjust other parameters of the function. Present balance before and after matching.

\item[a.] Now find balance with GenMatch using your own loss function. Explain the logic behind your choice of loss function. You may want to prioritize important selection variables, such as age. Present balance statistics after matching.

\item[a.] Create a loss function that ensures that GenMatch will not return a matched data set with worse balance on any variable in BalanceMatrix than the balance obtained by your matching method used in part a—as judged by eQQ-plots and difference of means. Do this so that this property holds by design—i.e., it holds regardless of the dataset used. In order to make this happen, you will have to both write a custom loss function (you may alter the one created in question “c” or write a new one), and provide GenMatch with starting.values so that it starts with the pscore matched dataset. Report your balance statistics after using this loss function.

\item[a.] Pick the matching method that produced, in your judgement, the best
balance. Estimate treatment effects and report them. How do they
differ from the reported estimates?

\item[a.] Select a set of covariates to condition on. Be sure to consider if any higher order terms and interactions are appropriate. Using these variables, perform Mahnolobis distance matching on a propensity score and “orthogo- nalized” covariates, with ATT as your estimand. Report your balance statistics, preferably using a plot.

\item[a.] Now using the same set of covariates (propensity score and “orthogonalized” covariates), use GenMatch to generate weights that optimize balance. Use the default setting for the loss function, but feel free to adjust other parameters of the function. Present balance before and after matching.

\item[a.] Now find balance with GenMatch using your own loss function. Explain the logic behind your choice of loss function. You may want to prioritize important selection variables, such as age. Present balance statistics after matching.
1

\item[a.] Create a loss function that ensures that GenMatch will not return a matched data set with worse balance on any variable in BalanceMatrix than the balance obtained by pscore matching—as judged by eQQ-plots and difference of means. Do this so that this property holds by design—i.e., it holds regardless of the dataset used. In order to make this happen, you will have to both write a custom loss function (you may alter the one created in question “c” or write a new one), and provide GenMatch with starting.values so that it starts with the pscore matched dataset. Report your balance statistics after using this loss function.

\item[a.] Pick the matching method that produced, in your judgement, the best
balance. Estimate treatment effects and report them. How do they
differ from the reported estimates?
\end{itemize}


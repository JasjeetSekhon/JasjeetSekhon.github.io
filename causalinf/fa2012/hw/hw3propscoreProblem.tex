\documentclass{article}
\usepackage{amsmath}
\usepackage{amsthm}
\usepackage{color}
\usepackage{setspace}
\usepackage{fullpage}
\usepackage[round]{natbib}
\usepackage[utf8]{inputenc}
 
\usepackage{fullpage}
\usepackage{boxedminipage}

\usepackage{listings}

\usepackage{minitoc}

\usepackage{ifpdf}

\usepackage{natbib} 
\usepackage{times} 
\usepackage{setspace}
\usepackage{subfigure}

\usepackage{hyperref} 

\newcommand\independent{\protect\mathpalette{\protect\independenT}{\perp}} 
\def\independenT#1#2{\mathrel{\rlap{$#1#2$}\mkern2mu{#1#2}}} 
\newcommand{\var}[0]{\text{var}}
\newcommand{\cov}[0]{\text{cov}}
\ifpdf 
\usepackage[pdftex]{graphicx} \else 
\usepackage{graphicx} \fi 

\begin{document}
\begin{itemize}
  \item[1)]
    Suppose there are 10,000 people.
    Some of these people are assigned to treatment, and the rest are assigned to control.
    The exact mechanism for assigning units to treatment is unknown,
    but is known to depend only on the values of three dichotomous variables:
    the sex of the person, 
    whether or not the person exercises 30 minutes a day, 
    and whether or not the person watches TV for more than an hour a day.
    For each of the $2^3 = 8$ configurations of the dichotomous variables, 
    there is at least one treated person and at least one non-treated person.
  \begin{itemize}
    \item[a)]
      Let $t_i = 1$ if person $i$ is treated, and let $t_i = 0$ if the person is not.
      Let $t = (t_1,t_2,\ldots,t_{10000})$ denote the 
      observed treatment assignment for all 10000 units.
      The probability that person $i$ is assigned treatment is $p_i$, which is unknown.
      Express the probability of observing the treatment assignment $t$ in terms of $t$ and $p$.
      What estimated values of $p_i$ maximize this probability, 
      under the assumption that treatment
      assignment only depends on the dichotomous variables?     
    \item[b)]  
      Suppose that the propensity score depends on all three of these dichotomous
      variables.  
      Will the estimated probabilities found in a) converge to the true propensity score?
      What if the propensity score only depends on two of the
      original three dichotomous variables?
    \item[c)]
      Suppose the propensity score is known (and again, depends only on the values of 
      these three variables).  
      Write out unbiased estimates for the ATE and the ATT, 
      and prove that they are unbiased (assuming that the propensity score is fixed).
      How do these estimates change if the propensity score is not fixed? 
    \item[d)]
      Suppose the heaviest person of the 10,000 people weighs 500 pounds.
      Moreover suppose the propensity score is:
      $$
        P(T_i =1 | X) = 1/2 - 1/4(\text{does not exercise more than 30 minutes a day}) - 
        \text{weight}/2000
      $$
      Can both the ATT and the ATE be estimated without bias by
      conditioning responses on the propensity score?  
      For each quantity that can be estimated unbiasedly, 
      give a description or formula on how to compute the estimate (supposing the
      propensity score is known and fixed).
    \item[e)]
      Suppose it is known that the propensity score is linear 
      in the exercise dichotomous
      variable and in weight, but the exact coefficients are unknown.
      Will regressing treatment assignment on 
      exercise and weight using OLS produce unbiased estimates of these coefficients?      
  \end{itemize}
\end{itemize}    
\end{document}

\documentclass{article}
\usepackage{amsmath}
\usepackage{amsthm}
\usepackage{color}
\usepackage{setspace}
\usepackage{fullpage}
\usepackage[round]{natbib}
\usepackage[utf8]{inputenc}
\usepackage{amssymb} 

% Setup for fullpage use
\usepackage{fullpage}

% Uncomment some of the following if you use the features
%
% Running Headers and footers
%\usepackage{fancyhdr}
% Multipart figures
%\usepackage{subfigure}
% More symbols
%\usepackage{amsmath}
%\usepackage{amssymb}
%\usepackage{latexsym}
% Surround parts of graphics with box
\usepackage{boxedminipage}

% Package for including code in the document
\usepackage{listings}

% If you want to generate a toc for each chapter (use with book)
\usepackage{minitoc}

% This is now the recommended way for checking for PDFLaTeX:
\usepackage{ifpdf}

%\newif\ifpdf
%\ifx\pdfoutput\undefined
%\pdffalse % we are not running PDFLaTeX
%\else
%\pdfoutput=1 % we are running PDFLaTeX
%\pdftrue
%\fi
\usepackage{natbib} 
\usepackage{times} 
\usepackage{setspace}
\usepackage{subfigure}

\usepackage{hyperref} 

\newcommand\independent{\protect\mathpalette{\protect\independenT}{\perp}} 
\def\independenT#1#2{\mathrel{\rlap{$#1#2$}\mkern2mu{#1#2}}} 

\newcommand{\indep}{\perp\!\!\!\perp}
\newcommand{\cov}[0]{\text{cov}}
\newcommand{\var}[0]{\text{var}}
\newcommand{\E}[0]{\mathbb{E}}



\ifpdf 
\usepackage[pdftex]{graphicx} \else 
\usepackage{graphicx} \fi 

\title{PS C236A / Stat C239A \\  Midterm Exam \\ Due: December 4, 2012}
\date{}

\begin{document}

\maketitle
\vspace{-4em}
\section*{Instructions}




%This is an ungraded practice exam.  The following instructions outline
%the expectations for the upcoming midterm. \vspace{1em}

\noindent This exam is due at the beginning of class (2:10 pm) on
Tuesday, December 4. The questions below will be graded as follows:
True/False (I) 10\%, analytical section (II, III, IV) 40\%, and the
empirical section (V) 50\%.  You \underline{\bf must} submit your
midterm answers in paper form in class.  This material should include
your analytical answers, plus all .R output, figures, and tables
needed to answer the computing portion of the exam.  We will not read
computer code to find your answer, however, you \underline{\bf must}
submit a fully executable version of all .R code to
$<$\texttt{jahenderson[at]berkeley.edu}$>$. If you do not send an
electronic version of your .R code, that portion of the midterm
\underline{\bf will not} be graded.  All files sent electronically
should be included in one omnibus email, with the subject line containing the course number and your last name (e.g., PS239A/STAT236A: Midterm - Rice).\\


\noindent Note: This exam is open book.  However, during the exam,
you are not allowed to communicate or cooperate with anyone in
any way about the exam. Any questions should be asked directly to the
Professor or the GSI. To repeat: you may not use study groups, online
help forms, the writing center, or any other form of external help.
If in doubt, ask.



\paragraph{\Large I. True or False}
Answer {\em True} or {\em False}.  Explain your answer in a sentence or two.
\begin{itemize}


\item[1.]   
      A scientist has a large population of $4n$ people;
      $2n$ of the people are men and $2n$ of the people are women.
      The scientist assigns treatment to exactly half of the men and half of the women,
      and assigns the rest of the people to control.
      Let $Y_i(1)$ and $Y_i(0)$ denote the potential outcomes
      of person $i$ given treatment and control respectively.
      The scientist wants to estimate the average treatment effect ($ATE$):
      $$
        ATE = \frac{1}{4n}\sum_{i=1}^{4n}Y_i(1) - \sum_{i=1}^{4n} Y_i(0)
      $$
      The scientist assumes the following model:
      $$
        Y_i = \alpha + \beta_1 T_i + \beta_2 M_i + \epsilon_i
      $$
      Here, $T_i$ is a treatment indicator, and
      $M_i$ is an indicator variable for whether
      subject $i$ is a male.
      The scientists estimates the $ATE$ by computing $\widehat{ATE} = \hat\beta_1$ 
      using OLS. 
      Under these assumptions, the OLS estimate is unbiased for the $ATE$.
%      A researcher  runs an experiment, and assigns people to 
%      treatment groups and control groups randomly.
%      The researcher estimates the Average Treatment Effect (ATE) in two ways: 
%      \begin{enumerate}
%        \item[i.] First, the researcher  computes
%          $$
%            \widehat{ATE}_1 = \sum_{i=1}^N \frac{Y_iT_i}{N_1} -\sum_{i=1}^N \frac{Y_i(1-T_i)}{N_0} 
%          $$
%           where $T_i$ are treatment indicators and 
%           $N_1$ and $N_0$ denote how many people were assigned to 
%           treatment and control respectively.
%           This estimate is unbiased for the ATE.
%         \item[ii.] Second, the researcher  assumes that responses were generated by the model
%           $$
%             Y_i = \alpha + \beta T_i + X_i\gamma+ \epsilon_i
%           $$
%           where $X_i$ are pre-treatment covariates, and $\epsilon_i$
%           are independent and identically distributed with
%           $\E(\epsilon_i) = 0$.  The researcher obtains
%           $\widehat{ATE}_2 = \hat\beta $ through OLS from this linear
%           model.  Since OLS
%           estimates coefficients unbiasedly, this method obtains an
%           unbiased estimate for the ATE.
%      \end{enumerate}
%      Both methods find that the average treatment effect estimate is
%      large and positive. The finding that both methods obtain large
%      and positive estimates of the ATE provides more evidence that
%      the ATE is positive than if only one of these methods were used.


    \item[2.]  An affinely invariant matching method was used to
      produce a matched sample in terms of a single covariate $x$, which
      is assumed to be the only confounder. The imbalance before and
      after matching was measured comparing the difference-in-mean,
      the variance ratio and the entire distribution of $x$ (using the
      Kolmogorov-Smirnov test). Before matching, all three measures
      showed a strong imbalance in $x$ among treated and control
      units. After matching, the variance ratio and the results of the
      Kolmogorov Smirnov test remained unchanged but the mean of $x$
      became identical in both groups. Since matching completely
      balanced the means and did not worsen the balance in all other
      moments of the distribution of $x$, this matching method removed
      some of the bias caused by the confounder $x$.

%A researcher is analyzing the effect of a treatment in a
%  randomized experiment and uses a two-sample $t$-test (with unequal
 % variances) to reject the null hypothesis of a 0 average treatment
 % effect. The researcher could have tested the same null hypothesis
 % with a randomization (permutation) test and his inference would not
 % depend on any large sample approximations.

 % \item[3.] A group of researchers begins a drug trial to study the
 %   effectiveness of a particular psychostimulant in treating
 %   Attention Deficit Hyperactivity Disorder (ADHD).  Before the
 %   study, individuals were surveyed to measure their likelihood to
 %   comply with the experimental protocol, resulting in a `compliance
 %   score' $S_i$ that is an unbiased estimate of
 %   $Pr(Complier_i|T_i=1)$, or the true probability of taking the
 %   drug if assigned to it in an experiment. In this particular
 %   experimental design, the researchers choose to assign the drug
 %   based on the scoring $S_i$ itself, using a weighted
 %   coin flip, so that $\pi_i=\pi_j$, if and only if, $S_i=S_j$,
 %   where $\pi_i$ is the probability of being assigned the drug
 %   treatment
 %   $(T_i=1)$, and $1-\pi_i$ is the probabilty of being assigned a
 %   sugar pill ($T_i=0$).
%  \vspace{.5em}
%
%  The researchers believe that the Effect of the Treatment on Treated
%  (ETT) (i.e., the average drug response for compliers) will be
%  positive, and that the average drug response among individuals in
%  the treatment group who refused treatment will
%  be zero.  Given these assumptions about the responses under
%  treatment (and given a known treatment assignment probability), it
%  is
%  better to assign treatment so that $\pi_i=1-S_i$, rather than
%  $\pi_i=S_i$, since it is more conservative (i.e., likely to recover
%  an estimate biased towards zero) to have a higher rate of
%  non-compliance in the treatment group rather than in the control
%  group here.


\item[3.]A political scientist is estimating the effect of a natural
  experiment, where the treatment under study is assigned by nature at
  the level of a US state. Fortunately, her outcome variable is
  measured at the level of the individual voter, so she has millions
  of observations to use. Even if the treatment effect is small, the
  test she considers should have considerable power to detect this
  small effect due to her large sample size.

\end{itemize}


\paragraph{\Large II. Regression Discontinuity Design \\ \\}
\vspace{1em} 

\noindent Consider a large medical trial for a new weight loss drug.  
      Before the trial, 
      each patient has their weight, height, and body fat percentage measured.
      A goodness-of-health score 
      is calculated for each patient based on these measurements
      (higher scores are a proxy for worse health).
      Assume that patients do not have time to manipulate their weight or body fat
      once selected to participate in the trial.
      Historically, a histogram of patient goodness-of-health 
      scores closely follows a normal
      distribution with mean $c$.
      It is thought that the effect of the drug varies with the value of this score.
      \begin{itemize}
        \item[a)]  
%          Consider the following mechanism for treatment assignment:
%          Before being assigned to treatment or control, each patient rolls a 6-sided die.
%          For a patient with a score above $c$, 
%          If the die comes up as 1, 2, 3, or 4 and the patient has a score of $c$ or above,
%          that patient takes the weight loss drug, otherwise they receive a placebo.
%          If the die comes up as a 5 or 6 and the patient has a score below $c$,
%          that patient takes the weight loss drug, otherwise they receive a placebo.
%          Suppose that the die roll is known to the experimenter.
          Suppose that the trial is conducted so that people with a 
          goodness-of-health score of $c$ or above 
          are given the drug and that people with a score below $c$ are not given the drug.
          Under this setup,
          what meaningful inferences can be made about the effect of the drug on weight loss? 
          Discuss the parameter of interest and the 
          methods used to estimate this parameter.
          What assumptions are required to estimate this parameter?
        \item[b)]
          Consider the following mechanism for treatment assignment:
          Before being assigned to treatment or control, each patient rolls a 6-sided die.
          If the die comes up as 1, 2, 3, or 4 and the patient has a score of $c$ or above,
          that patient takes the weight loss drug, otherwise they receive a placebo.
          If the die comes up as a 5 or 6 and the patient has a score below $c$,
          that patient takes the weight loss drug, otherwise they receive a placebo.
          Suppose that the die roll is unknown to the experimenter.
          
          What does the estimate in a) measure now?
          Can the parameter of interest in part a) still be estimated?
          If so, how?  If not, why not?
        \item[c)]
          Suppose that the effect of the drug is thought to be the same
          for all patients with goodness-of-health scores within the interval
          $(c-5,c+5)$.
          Suppose that patients with scores below $c$ 
          are ineligible to receive the weight loss drug.
          Patients with scores of $c$ or above are given an appointment to receive 
          the drug.
          The drug is administered only once during the trial, and only at this appointment.
          Some patients fail to arrive at their appointment.
          Under this setup,
          discuss at least two types of inference 
          possible for measuring the effect of the drug on weight loss? 
          Discuss the parameters of interest and the methods used to estimate these parameters.
          What assumptions are required to estimate these parameters?
          Which estimate will be larger (in absolute value)?
      \end{itemize}

\paragraph{\Large III. Sensitivity Analysis \\ \\}

  Suppose that there is a study with 
      a total of $2n$ subjects.
      Exactly $n$ of these people are smokers.
      A height and weight are measured for
      each subject.
      Suppose that there are enough people so that
      the joint distribution of the heights and weights is extremely close to a
      multivariate normal distribution.
      The researcher wants to test whether smoking affects 40-yard dash times.
      \begin{itemize}
        \item[a)]
          A statistician notices some imbalance in the average weight and height 
          between the smokers and the non-smokers.         
          To fix the imbalance, the statistician matches smokers
          to non-smokers by matching 
          on the Mahalanobis distance with height and weight covariates          
          (with replacement, nearest neighbor).
          Will the differences in average height and average
          weight between the smokers
          and matched non-smokers            
          be as small or smaller as they were before matching?  
          Why or why not?
        \item[b)]
          Suppose instead that all subjects in the study are twins.
          For each set of twins, one twin is a smoker and one twin is a non-smoker,
          and both twins in each set have the same height and weight.
          In his analysis, the statistician believes that 
          the smoking sibling in a twin pair 
          is essentially random, though he concedes
          that some unobserved trait may help explain a twin's 
          propensity for being a smoker.
           
          For each set of twins $s$, let $(1,s)$ denote the twin that smokes,
          and let $(2,s)$ denote the non-smoking twin.
          Let $T_{is}$ denote random smoking indicators;
          $T_{is} = 1$ if the $i$th unit in the $s$th twin pair smokes, $i = 1,2$.
          For this study, for each pair $s$, 
          the smoking indicators are observed to be $T_{1s} = 1$
          and $T_{2s} = 0$.
          The statistician models the probability that a subject smokes in the following way:
          \begin{equation}
            \log\left( 
              \frac{P(T_{is} = 1)}{1 - P(T_{is} = 1)}
            \right) = \alpha + \kappa_1 h_{is} + \kappa_2 w_{is} + \gamma u_{is}
            \label{probassign}
          \end{equation}
          where $h_{is}$ and $w_{is}$ are the height and weight of twin $(i,s)$,
          and $u_{is}$ is the value of an unobserved covariate for that twin.
          The statistician also assumes that any subject cannot influence
          any other subject to smoke or not smoke
          (smoking is independent across all subjects).
           
          Show that, under this model, the probability
          that subject $(1,s)$ is a smoker is:
%          of the obtaining
%          the observed smoking assignment is:
          \begin{equation}
            P(T_{1s} = 1| T_{1s} + T_{2s} = 1) 
            = \frac{e^{\gamma u_{1s}}}{e^{\gamma u_{1s}} + e^{\gamma u_{2s}}}
            \label{aprobques}
          \end{equation}
          Hint: Use $P(A | B) = P(A\cap B)/P(B)$, and find an expression for
          $$
            \frac{P(T_{1s} = 1 \cap T_{2s} = 0)}{P(T_{1s} = 0 \cap T_{2s} = 1)} = 
            \left(\frac{P(T_{1s} = 1)}{1 - P(T_{1s} = 1)}\right)
            \left(\frac{P(T_{2s} = 1)}{1 - P(T_{2s} = 0)}\right)
          $$
        \item[c)]
          Suppose that $0 \leq u_{is} \leq 1$ and that $\gamma > 0$.
          Find an upper and lower bound (sharper than just 1 and 0) for the probability
          $P(T_{1s} = 1| T_{1s} + T_{2s} = 1).$
          Denote these bounds by $p^+_s$ and $p^-_s$ respectively.
          Do the same for $P(T_{1s} = 0| T_{1s} + T_{2s} = 1).$
          Comment, in one sentence, on how these bounds change if $\gamma < 0$.
        \item[d)]
          Let $y_{is}$ denote the 40-yard dash time of subject $(i,s)$ in milliseconds.
          Let $Z_s$ denote an indicator variable for the smoker having the faster 40-yard dash time:
          $Z_s = 1$ if and only if the smoking twin 
          has a faster 40-yard dash time than the non-smoking twin.
          Let $d_s$ denote the rank of $|y_{1s} - y_{2s}|$; 
          higher ranks denote larger absolute values.
          Assume there are no ties between $y_{1s}$ and $y_{2s}$ within any 
          twin pair $s$,
          and that $|y_{1s} - y_{2s}| \neq |y_{1t} - y_{2t}|$ for all distinct twin pairs $s,t$.
          
          The Wilcoxon signed rank statistic is:
          $$
            W = \sum_{s=1}^n d_sZ_s.
          $$
          Let $Z^+_s$ and $Z^-$ be independent and identically distributed 
          bernoulli random variables (or indicator variables) with $P(Z^+_s = 1) = p^+_s$
          and $P(Z^-_s = 1) = p^-_s$.
          Consider the following random variables:
          \begin{eqnarray*}
            W^+ &=& \sum_{s=1}^n d_s Z^+_s \\
            W^- & = & \sum_{s=1}^n d_sZ^-_s
          \end{eqnarray*}
          Show that, under the null hypothesis that smoking does not effect 40-yard dash times, 
          the following property holds:
          $$
           \E(W^-)\leq  \E(W| T_{1s} + T_{2s} =1) \leq \E(W^+) 
          $$
        \item[e)]
          In fact, it can be shown that under this null hypothesis, for any $a$:
          \begin{equation}
            P(T^- \geq a) \leq P(T \geq a | Z_{1s} + Z_{2s} = 1) \leq P(T^+ \geq a)
            \label{probabs}
          \end{equation}
          Discuss, in about 3 -5 sentences or so, how property~\eqref{probabs} can be
          exploited to test the exact null of no treatment effect. 
        \item[f)] [BONUS QUESTION]
          Prove property~\eqref{probabs}. 
      \end{itemize}

\paragraph{\Large IV. Media Bias \\ \\}

For this problem, you will compare the research design from three
papers studying the effects of media bias on political attitudes and
choices.  The first paper is ``The Fox News Effect'', by S.
DellaVigna and E. Kaplan (DVK), and can be found here
\url{http://sekhon.berkeley.edu/causalinf/papers/DellaVignaFoxNews.pdf}. The
second paper is ``Exploiting a Rare Shift in Communication Flows to
Document News Media Persuasion'', by J. Ladd and G. Lenz (LL),
and can be found here
\url{http://sekhon.berkeley.edu/causalinf/papers/LaddLenzBritish.pdf}. And
the third paper is ``Does the Media Matter? A Field Experiment
Measuring the Effect of Newspapers on Voting Behavior and Political
Opinions'', by A. Gerber, D. Karlan, and D. Bergan (GKB), and
can be found here
\url{http://sekhon.berkeley.edu/causalinf/papers/GerberNewspapers.pdf}.

\vspace{1em}

Please write a page or two addressing the following questions:


   \begin{itemize}
   \item[a.]  Compare the identification strategies of the
     three papers.  Which strategy do you find the most convincing?
     The least?  Why?

   \item[b.] Given the different types of interventions being studied
     (e.g., biased media exposure v. change in media bias, television v. newspaper media,
     etc), in what sense are the findings across these three studies
     `comparable'?  Do these studies give us useful information to
     test the same theoretical claim or different theoretical claims?  

   \item[c.] Imagine at a future point in time, Fox News expanded to
    every major cable and media market in the US. Would we 
     expect to see a similar media effect as measured by DVK as a
     result of this national expansion?  Why or why
    not?  What would be an analogous type of issue in the studies
    conducted by LL and GKB?  Do any of the three studies seem more
    robust to this type of issue than the others?

   \item[d.]  Which paper do you find the most 
     interesting, weighting both the scope and significance of the
     effect being estimated, as well as the {\em external} and {\em internal} validity
     of the respective estimates?  Generally speaking, which study is
     {\em most} informative about the substantive impact of media bias on 
public opinion or vote choice?  


\end{itemize}




\paragraph{\Large V. Data and Matching \\ \\}

For this problem, you will perform several matching exercises using
Ladd and Lenz's ``Exploiting a Rare Shift'' data.  The unit of
observation is the individual respondent in a UK election survey, and
the treatment under study is whether an individual is a reader of a
newspaper that switched its party endorsement from Tory to Labour in
the run-up to the 1997 election.  The main outcome is change in Labour
party vote support between 1992 and 1997.  To control for confounding,
the authors condition on a number of covariates (listed in Table 3
and Table 1A of their paper) that may predict both readership and party voting
behavior.
\\

\noindent The Ladd and Lenz data is available here:
\url{http://sekhon.berkeley.edu/causalinf/data/midterm.dta}.
The variables are described in the following file:
\url{http://sekhon.berkeley.edu/causalinf/data/midterm_codebook.xsls}


\vspace{1em}
\noindent For parts (a) - (e) below, be sure to explicitly set seeds
to ensure that GenMatch recovers reproducible results,
i.e. \texttt{set.seed} in general, and in GenMatch \texttt{unif.seed}, \texttt{int.seed}.

\begin{itemize}
\item[a.]  Estimate the causal effect of being a typical reader of a newspaper
  that switched party endorsement (from Tory to Labour) on the {\em
    change} in Labour party vote support between 1992 and 1997.  In
  doing so, select a set of relevant covariates to condition on.  In
  matching, first use a custom loss function and then use GenMatch’s
  default loss function. Provide some justification for your custom
  loss function. Choose the matched dataset with the best balance on
  the relevant covariates.  Are the media effects on voting you
  estimate significantly different from zero? What are the mean
  differences in change in party vote suport you recover after
  matching?  What are the three {\em worst} balanced covariates in
  this best-matched dataset?  What are the standardized mean
  differences across matched treated and control on these three
  covariates?


\item[b.] Using the best-matched dataset from part (a), restrict your
  analysis of the matched-pairs to include only those treated
  individuals who are habitual readers of a newspaper that switched
  its party endorsement.  Check balance on this `restricted' dataset
  using \texttt{MatchBalance}.  Does balance change considerably in
  this dataset, compared with that recovered in (a)?  Now, use
  GenMatch to match on the same covariates used in (a), utilizing
  habitual readers as the treatment indicator.  Does balance in this
  matched dataset improve compared to that recoverd in the
  `restricted' matched dataset?  What media effects do you recover in these
  two matched datasets?  Are these different from that found in (a)?

\item[c.] Choose one matched dataset from (a) or (b) that you think is
  the most convincing in recovering conditional exchangeability (for
  either habitual or typical readers), and conduct two robustness
  tests of the conditional exchangeability assumption.  The first
  robustness check should be a Rosenbaum sensitivity test using the
  \texttt{rbounds} package in {R}.  The second robustness check either
  should be a post-matching parametric bias adjustment on the matched
  data (e.g., a probit regression including covariates and treatment
  to model the outcome on the matched data), or a placebo test of the
  effect of treatment on a prior party vote outcome before and after
  matching.  What is the $\Gamma$ magnitude of confounding due to an
  unobserved covariate in the Rosenbaum sensitivity test at which the
  estimated treatment effect is indistinguishable from zero?  How does
  this $\Gamma$ compare to the imbalance recovered on the three
  worst-balanced covariates in the best-balanced dataset in (a)?  Are
  these robustness tests convincing that conditional exchangeability
  holds?


\item[d.] Repeat the analysis in part (a), this time using the {\em
    level} of Labour party vote support in 1997 (rather than change in
  vote support). Is this estimate consistent with the one recovered in
  (a)?  Is this causal estimate more or less persuasive than the
  difference-in-difference estimate you recovered in (a)?   Overall, what do we learn about the effects of media from this
  analysis?  





%\item[e.] In this study, does SUTVA limit what we can say? If so, how?
%  If not, why not?  

\item[e.] [BONUS QUESTION] Fully replicate the regression analysis in
  Table 1A (excluding the 1992 instrument column), on both the {\em
    level} and {\em change} in party vote support.  That is, do the
  bivariate analysis, the exact matching on the same covariates used
  by Ladd and Lenz, and the GenMatch analysis on the same coveriates,
  and also perform linear adjustment on each matched data set.  Can
  you replicate the table exactly? If not, which parts can you
  replicate exactly? How confident are you that this analysis is
  recovering an unbiased estimate of the persuasive effect of media on
  vote choice behavior? Does replicating the analysis change your
  assessment?







\end{itemize}

\end{document}


\item[a.] Create a loss function that ensures that GenMatch will not return a matched data set with worse balance on any variable in BalanceMatrix than the balance obtained by your matching method used in part a—as judged by eQQ-plots and difference of means. Do this so that this property holds by design—i.e., it holds regardless of the dataset used. In order to make this happen, you will have to both write a custom loss function (you may alter the one created in question “c” or write a new one), and provide GenMatch with starting.values so that it starts with the pscore matched dataset. Report your balance statistics after using this loss function.

\item[a.] Pick the matching method that produced, in your judgement, the best
balance. Estimate treatment effects and report them. How do they
differ from the reported estimates?

\item[a.] Select a set of covariates to condition on. Be sure to consider if any higher order terms and interactions are appropriate. Using these variables, perform Mahnolobis distance matching on a propensity score and “orthogo- nalized” covariates, with ATT as your estimand. Report your balance statistics, preferably using a plot.

\item[a.] Now using the same set of covariates (propensity score and “orthogonalized” covariates), use GenMatch to generate weights that optimize balance. Use the default setting for the loss function, but feel free to adjust other parameters of the function. Present balance before and after matching.

\item[a.] Now find balance with GenMatch using your own loss function. Explain the logic behind your choice of loss function. You may want to prioritize important selection variables, such as age. Present balance statistics after matching.
1

\item[a.] Create a loss function that ensures that GenMatch will not return a matched data set with worse balance on any variable in BalanceMatrix than the balance obtained by pscore matching—as judged by eQQ-plots and difference of means. Do this so that this property holds by design—i.e., it holds regardless of the dataset used. In order to make this happen, you will have to both write a custom loss function (you may alter the one created in question “c” or write a new one), and provide GenMatch with starting.values so that it starts with the pscore matched dataset. Report your balance statistics after using this loss function.

\item[a.] Pick the matching method that produced, in your judgement, the best
balance. Estimate treatment effects and report them. How do they
differ from the reported estimates?
\end{itemize}




\end{document}


For this problem, you will perform several matching exercises using
the ``Fox News Effect'' data.  The unit of observation are towns in
the US, and the treatment under study is the availability of Fox News
during the 2000 election season. The outcome (\texttt{reppresfv2p00m96}) is the
change in the Republican presidential vote share between 1996 and
2000. The dataset for this assignment only includes those towns with
pre-treatment outcome data, i.e. the change in the Republican
presidential vote share between 1988 and 1992 (\texttt{reppresfv2p92m88}). The
treatment indicator (\texttt{foxnews2000}) has been defined as equal to one if
the town’s cable system carried the Fox News network before the 2000
election. The dataset includes a set of demographic covariates from
the 2000 and 1990 census

\begin{itemize}
\item[a.] Select a set of covariates to condition on. Be sure to
  consider if any higher order terms and interactions that are
  appropriate. Using these variables, perform Mahnolobis distance
  matching on a propensity score and ``orthogonalized'' covariates
  (simultaneously),
  with ATT as your estimand. Report your treatment effect estimates
  and your balance statistics, either in a table or a plot. Is it
  similar to the estimate reported in the original paper, in
  particular the estimate reported in columns 6 and 7 in Table IV?
   \item[b.] Now with the same set of covariates, use GenMatch to
     generate weights that optimize balance. Use the default setting
     for the loss function, but feel free to adjust other parameters
     of the function. Present balance before and after matching, as
     well as your effect estimates. How is your estimate different
     from the results reported in the paper and your findings in part
     a?
   \item[c.] Now match using either the method from part (a) or the
     method from part (b), using only demographic covariates. Estimate
     the ``treatment effect'' of the introduction of Fox news prior to
     the 2000 election on pre-treatment outcome of the change in
     Republican presidential vote share between 1988 and 1992. This is
     known as a ``placebo test''. Can you recover a 0 ATT estimate using
     only demographic covariates as the conditioning set?
\end{itemize}






\begin{itemize}
\item[a.] Select a set of covariates to condition on. Be sure to consider if
any higher order terms and interactions are appropriate. Using these
variables, perform Mahalanobis distance matching, with ATT as your
estimand. Report your balance statistics.

\item[a.] Now using the same set of covariates, use GenMatch to generate weights that optimize balance. Use the default setting for the loss function, but feel free to adjust other parameters of the function. Present balance before and after matching.

\item[a.] Now find balance with GenMatch using your own loss function. Explain the logic behind your choice of loss function. You may want to prioritize important selection variables, such as age. Present balance statistics after matching.

\item[a.] Create a loss function that ensures that GenMatch will not return a matched data set with worse balance on any variable in BalanceMatrix than the balance obtained by your matching method used in part a—as judged by eQQ-plots and difference of means. Do this so that this property holds by design—i.e., it holds regardless of the dataset used. In order to make this happen, you will have to both write a custom loss function (you may alter the one created in question “c” or write a new one), and provide GenMatch with starting.values so that it starts with the pscore matched dataset. Report your balance statistics after using this loss function.

\item[a.] Pick the matching method that produced, in your judgement, the best
balance. Estimate treatment effects and report them. How do they
differ from the reported estimates?

\item[a.] Select a set of covariates to condition on. Be sure to consider if any higher order terms and interactions are appropriate. Using these variables, perform Mahnolobis distance matching on a propensity score and “orthogo- nalized” covariates, with ATT as your estimand. Report your balance statistics, preferably using a plot.

\item[a.] Now using the same set of covariates (propensity score and “orthogonalized” covariates), use GenMatch to generate weights that optimize balance. Use the default setting for the loss function, but feel free to adjust other parameters of the function. Present balance before and after matching.

\item[a.] Now find balance with GenMatch using your own loss function. Explain the logic behind your choice of loss function. You may want to prioritize important selection variables, such as age. Present balance statistics after matching.
1

\item[a.] Create a loss function that ensures that GenMatch will not return a matched data set with worse balance on any variable in BalanceMatrix than the balance obtained by pscore matching—as judged by eQQ-plots and difference of means. Do this so that this property holds by design—i.e., it holds regardless of the dataset used. In order to make this happen, you will have to both write a custom loss function (you may alter the one created in question “c” or write a new one), and provide GenMatch with starting.values so that it starts with the pscore matched dataset. Report your balance statistics after using this loss function.

\item[a.] Pick the matching method that produced, in your judgement, the best
balance. Estimate treatment effects and report them. How do they
differ from the reported estimates?
\end{itemize}




\item[1(a).]   A researcher runs an experiment, and assigns people to 
      treatment groups and control groups randomly.
      The researcher estimates the Average Treatment Effect (ATE) in two ways: 
      \begin{enumerate}
        \item[i.] First, the researcher  computes
          $$
            \widehat{ATE}_1 = \sum_{i=1}^N \frac{Y_iT_i}{N_1} -\sum_{i=1}^N \frac{Y_i(1-T_i)}{N_0} 
          $$
           where $T_i$ are treatment indicators and 
           $N_1$ and $N_0$ denote how many people were assigned to 
           treatment and control respectively.
           This estimate is unbiased for the ATE.
         \item[ii.] Second, the researcher assumes that responses were generated by the model
           $$
             Y_i = \alpha + \beta T_i + \epsilon_i
           $$
           where $\epsilon_i$ are independent
           and identically distributed with $\E(\epsilon_i) = 0$.
           The researcher obtains $\widehat{ATE}_2 = \hat\beta $ through OLS.
           Since OLS estimates coefficients unbiasedly, this 
           method obtains an unbiased estimate for the ATE.
      \end{enumerate}
      Both methods find that the average treatment effect estimate is
      large and positive.  The finding that both methods obtain large
      and positive estimates of the ATE provides more evidence that
      the ATE is positive than if only one of these methods were used.





\documentclass{article}
\usepackage{amsmath}
\usepackage{amsthm}
\usepackage{color}
\usepackage{setspace}
\usepackage{fullpage}
\usepackage[round]{natbib}
\usepackage[utf8]{inputenc}
\usepackage{amssymb} 

% Setup for fullpage use
\usepackage{fullpage}

% Uncomment some of the following if you use the features
%
% Running Headers and footers
%\usepackage{fancyhdr}
% Multipart figures
%\usepackage{subfigure}
% More symbols
%\usepackage{amsmath}
%\usepackage{amssymb}
%\usepackage{latexsym}
% Surround parts of graphics with box
\usepackage{boxedminipage}

% Package for including code in the document
\usepackage{listings}

% If you want to generate a toc for each chapter (use with book)
\usepackage{minitoc}

% This is now the recommended way for checking for PDFLaTeX:
\usepackage{ifpdf}

%\newif\ifpdf
%\ifx\pdfoutput\undefined
%\pdffalse % we are not running PDFLaTeX
%\else
%\pdfoutput=1 % we are running PDFLaTeX
%\pdftrue
%\fi
\usepackage{natbib} 
\usepackage{times} 
\usepackage{setspace}
\usepackage{subfigure}

\usepackage{hyperref} 

\newcommand\independent{\protect\mathpalette{\protect\independenT}{\perp}} 
\def\independenT#1#2{\mathrel{\rlap{$#1#2$}\mkern2mu{#1#2}}} 



\ifpdf 
\usepackage[pdftex]{graphicx} \else 
\usepackage{graphicx} \fi 

\title{PS C236A / Stat C239A \\ Problem Set 2 \\ Due: Oct. 1, 2012}
\date{}

\begin{document}

\maketitle
\vspace{-4em}
\section*{Instructions}
This assignment is due {\bf 4 pm Monday, Oct. 1.}  You may submit your
analytical work either electronically or in paper form.  Electronic
versions must be sent as a .pdf to
$<$\texttt{jahenderson[at]berkeley.edu}$>$. Paper copies should be
placed in my mailbox in 210 Barrows.  For the computing portion of the
assignment, you \underline{must} submit a fully executable version of
all .R code, along with any data used in the code (excepting that
provided through the course webpage) to the email above.  All files
for each assignment sent electronically should be included in one
omnibus email, with the subject line containing the course and
homework number, and your last name (e.g., PS239A/STAT236A: HW2 - Romney).\\







\noindent You are encouraged to work together in groups to complete
the assignments. However, you must hand in your own individual
answers. Photocopies and other reproductions of someone else’s answers
are not acceptable. Please also list the names of everyone with
whom you have collaborated on this assignment.


%test statistic question; maybe in R

%permutation inference in R;

%univariate matching 

%\section*{Permutation Inference}

\paragraph{Problem 1: The Lady Tasting Tea }
Consider the following variation of the Lady Tasting Tea example that
we discussed in class. The Lady tastes eight cups of tea, four of
which have milk added first and four of which have tea added
first. The cups are organized into matched pairs and for each pair, a
fair coin is flipped to determine which gets milk first. The Lady
knows the design, meaning that she knows there is one milk-first cup
and one tea-first cup in each matched pair.

\begin{itemize}


\item[a.]   Consider the following hypothesis test:  The null hypothesis 
  is that the Lady has no ability to
  discriminate the order in which milk is added to tea. 
  The alternative is that the Lady's ability to discriminate 
  the order is better than random chance.
  In the case where the Lady makes one mistake (classifies
  one milk-first cup as a tea-first cup), what is the $p$-value for this hypothesis test?

\item[b.] Consider the same null and alternative as in part (a).
  Suppose now that the cups are no longer paired; instead milk-first
  or tea-first assignment is completely randomized, with four cups
  receiving each assignment.  The Lady is told that exactly four cups
  are milk-first, but is given no additional information.  If the Lady
  makes one mistake (classifies one milk-first cup as a tea-first
  cup), what is the $p$-value for the hypothesis test?  Is this
  $p$-value different from the one calculated in part (a)?  Why or why
  not?  If you are trying to discern whether the Lady can correctly
  identify milk-first and tea-first cups, which design would you
  prefer, the one in (a) or the one in (b)?

\item[c.]  You believe that the Lady guesses ``milk-first'' 2/3 of the time.  
  Suppose you have a coin that
  lands heads 2/3 of the time.  
  For each of the eight cups, you flip the coin and pour
  milk first or tea first depending on the outcome of the coin.
  Which would you prefer:  
  Pour milk first on heads or pour tea first on heads?
  
\item[d.]  
  Consider the same null and alternative hypotheses as in part (a).
  Suppose that for each of the eight cups, you flip a fair coin, 
  and you pour milk into that cup first
  if that coin lands heads (otherwise, you pour tea first).
  By chance, seven of the cups are milk-first, and only one of the cups is tea-first.
  The Lady, when told about the randomization mechanism, 
  states that she will choose at most six cups to be milk-first and at most six cups to 
  be tea-first, as any more than that ``is far too unlikely to happen.''
  Suppose that the Lady makes exactly two mistakes.
  What is the $p$-value for this hypothesis test?

\end{itemize}

%\section*{Observational Studies}



\paragraph{Problem 2: Catholic School -- I}  In an observational study of the effects of
attending a Catholic school, the central dependent variable of
interest is a binary variable, $Y_i$, which indicates whether or not
student $i$ graduated from high school. The treatment variable, $T_i$,
indicates Catholic school attendance. In a very large sample of
students, half attended Catholic school and half did not. You observe
that the treated students have a graduation rate of .7 and the control
students have a graduation rate of .5. You wish to estimate the
average treatment effect of attending Catholic school. Assume that
your sample is large enough to make sampling variability negligible.

\begin{itemize}
\item[a.] Without making any assumptions about the relationship
  between the students’ potential outcomes and treatment assignment,
  what is the largest possible value of the ATE? What is the smallest
  possible value of the ATE? What is the difference between these two
  values? Will this difference between the maximum and minimum
  possible ATE always be the same, irrespective of the specific
  observed values of the outcome variable?
\item[b.] Again making no assumptions about treatment assignment,
  assume that Catholic school does not prevent any student from
  graduating. What is the largest possible value of the ATE? What is
  the smallest possible value of the ATE?
\end{itemize}

\paragraph{Problem 3: Catholic School -- II}

To address this question more precisely, researchers randomly sample 6
students, measuring some covariate $X$ for students who attend
Catholic ($T=1$) and non-Catholic school ($T=0$).  Assume
unconfoundedness holds conditionally on $X$ {\em and} $U$.
%or
%$\{Y_1,Y_0 \perp T | X,U\}$.
% Thus, potential outcomes are exchangeable across Catholic and
% non-Catholic attendance, given $X$ {\em and} $U$. 
Moreover, assume the conditionality in $T_i$ follows the logit
distribution, $\pi_i/(1-\pi_i) = \exp(X_i\beta+\gamma U_i)$, where
$\pi_i$ is student $i$'s probability of attending Catholic school, and
$\gamma$ and $\beta$ are additive parameters.  
% (Also note that $Y_0=f(X,U)$, that is, the potential outcomes under
% treatment assignment are some function of $X$ and $U$.)
In this study, $Y$, $T$, and $X$ are observed, while $U$ is not.
Though unobserved for the researchers, however, the ``true'' values of
$U$ are given to us below in $U_{true}$.


\begin{table}[!h]
		\caption{Catholic School Graduation Data}
	\begin{center}
		\begin{tabular}{cccc|cc}
                  Unit & $Y$ & $T$ & $X$ & $U_{true}$ & $U_{obs}$\\ \hline  
                  1    & 1     & 1   &\ 1.37   & 1   & 1  \\ 
                  2    & 0     & 1   &\ 0.16   & 1    & -\\ 
                  3    & 0    &  0    &\ 0.51   & 0   & 0\\ 
                  4    & 1    &  1    &\ 0.99   & 1  & - \\ 
                  5   &  1    &  0    &\ 1.53   & 1   & 1\\ 
                  6  &  1     &  0     &-0.46   & 0  & -\\ 

		\end{tabular}
		\label{}
	\end{center}
\end{table}

\begin{itemize}
  % \item[a.] Assume random assignment to treatment occurs with
  %   $Pr(T_i=1)=\pi_i$, with $\gamma=0$ and $\beta=0$ from above.
  %   Define a test statistic $t(T,Y)$ to be the number of treated
  %   units
  %   with a response equal to 1.  Compute this statistic for the data
  %   collected in the above table.  Using your estimated test
  %   statistic
  %   and these assumptions about $\pi_i$, what is the $p$-value
  %   associated with this test, under the null hypothesis of no
  %   effect of
  %   attending Catholic school on graduation?

\item[a.] Match each treated unit to one control unit on $X$ without
  replacement,
% (i.e., use each control only once), 
  to minimize: $\sum_{s=1}^{S}(X_{si}-X_{sj})^2$, where $T_{si}=1$ and
  $T_{sj}=0$. List which units are matched together in each $s$.  What
  is the resulting McNemar test statistic after matching?  What is the
  estimated ATT after matching?

\item[b.] Assume that after matching,
  $Pr(T_{si}=1)=Pr(T_{sj}=1)=\pi_{s}$, but that $\pi_s$ varies across
  each strata.  Using the McNemar test statistic from part (a),
  % , again with $\gamma=0$ and $\beta=0$.
  what is the $p$-value that this statistic occurred by chance, under
  the null hypothesis of no effect of attending Catholic school on
  graduation (in the stratified design)?

\item[c.] Now assume that $\gamma=0.62$ and $\beta=1.49$.
  % What is $\pi_i$ for each $i$?
  Given your matches in (a), does $\pi_{si}=\pi_{sj}$ for each of the
  $s$ strata (where again $T_{si}=1$ and $T_{sj}=0$)?  If these are
  different, which strata has the largest difference in the
  probability of attending Catholic school for treated and control
  units?  Now what is the $p$-value that the McNemar statistic,
  estimated from your matches in part (a), occurred by chance under
  the null of no effect of attending Catholic school on graduation,
  given this new information about $X$, $U_{true}$, $\gamma$, and
  $\beta$ in the stratified design?  Is this $p$-value the same as you
  calculated in part (b)? Why or why not?

\end{itemize}

%\begin{table}[h!!!!!!!]
%		\caption{Catholic School Graduation Data}
%	\begin{center}
%		\begin{tabular}{cccc|c}
%    Unit & $Y$ & $T$ & $X$ &$U_{obs}$\\ \hline
%    1 & 1 & 1 &\ 1.37 & 1 \\
%    2 & 0 & 1 &\ 0.16 & - \\
%    3 & 0 & 0 &\ 0.51 & 0 \\
%    4 & 1 & 1 &\ 0.99 & - \\
%    5 & 1 & 0 &\ 1.53 & 1 \\
%    6 & 0 & 0 &-0.46 & - \\

% \end{tabular}
%		\label{}
%	\end{center}
%\end{table}

\begin{itemize}
\item[d.] The researchers find additional money to collect some
  information about $U$, but can only do so for units 1, 3, and 5.
  (This data is presented as $U_{obs}$ in Table 1 above; in the table,
  a ``dash'' means missing data.)  Create a new stratification, this
  time matching with replacement on $X$, again to minimize
  $\sum_{s=1}^{S}(X_{si}-X_{sj})^2$.  Calculate the ATT as your test
  statistic after matching.  Again assume that $\gamma=0.62$ and
  $\beta=1.49$.  Under the null hypothesis of no effect in this
  stratification design, what are the {\em largest} and the {\em
    smallest} $p$-values possible, associated with the probability
  that this statistic occurred by chance, given our remaining
  ignorance in $U_{obs}$ about $U_{true}$?





\end{itemize}

%\paragraph{Problem 4}
%Consider an observational study, where $Z_i=1$ if unit $i$ is in the treatment group and
%$Z_i=0$ if unit $i$ is in  the control group. Let $X$ be
%a vector of observed pretreatment covariates. Write
%$X_{Z=1}$ for the observed covariates of the units in the
%treatment group. Similarly, let $X_{Z=0}$ be the observed
%covariates in the control group.  Let $r_{1}$ be outcome under treatment
%and $r_{0}$ be the outcome under control.  Assume the following:
%$$r_0 \independent Z|X_{Z=1}$$
%$$ P(Z=1|X_{Z=1})<1$$

%\noindent Suppose you know the propensity score $e(X)=P(Z=1)$ for all
%units $i$.  With these assumptions, can conditioning on the propensity
%score estimate the ATT without bias? Prove it mathematically and
%describe your logic in words.  What additional assumption would we
%need in order to estimate the ATE without bias? 

% First show that conditioning on the propensity score is equivalent
% to conditioning on $X_{Z=1}$. Then show that conditioning on the
% propensity score can produce unbiased ATT estimates under the
% assumptions above.



\paragraph{Problem 4:}
In this problem, you will analyze a famous experiment conducted by Leonard Wantchekon in Benin in 2001. Wantchekon wanted to examine the effectiveness of different types of campaign messages on voting behavior in a presidential election.
For details, see:
\begin{quote}
  \url{http://www.princeton.edu/~lwantche/Clientelism_and_Voting_Behavior_Wantchekon.pdf}
\end{quote}

Wantchekon convinced the campaigns of the major presidential
candidates to randomize the messages they employed in 24 villages. The
two treatment conditions were as follows:
\begin{enumerate}
\item \textit{Public Policy:} Wantchekon describes this treatment condition as: ``It was decided that any public policy platform would raise issues pertaining to national unity and peace, eradicating corruption, alleviating poverty, developing agriculture and industry, protecting the rights of women and children, developing rural credit, providing access to the judicial system, protecting the environment, and/or fostering educational reforms.''
\item \textit{Clientelist}: Wantchekon describes this treatment as: ``A clientelist message, by contrast, would take the form of a specific promise to the village, for example, for government patronage jobs or local public goods, such as establishing a new local university or providing financial support for local fishermen or cotton producers.''
%\item \textit{Both}: These villages received both types of messages. 
\end{enumerate}

The data has been modified for the assignment, but the basic structure
of the experiment was \textit{block} randomization. For the purposes
of the assignment, villages were divided into groups of 2 based on
geography and treatment status was randomized within the 8 groups of
2. The outcome variable is the vote share of the candidate
participating in the experiment. The only covariate is the number of
registered voters. In the dataset, \texttt{block} indicates block
group, \texttt{reg.voters} is the registered voters
covariate,\texttt{vote.pop}is the outcome variable, \texttt{treat} is
a variable indicating treatment status.The data can be found here: 
\begin{quote}
  \url{http://sekhon.berkeley.edu/causalinf/data/hw2data.RData} \\
\end{quote} 


\noindent In this problem, we are interested in the difference between
the clientelist and public policy conditions.  
\begin{itemize}
\item[a.] Estimate the effect the clientelist message compared to the
  public policy message, using the ITT estimator and the regression
  estimator. For the regression estimate, include block level dummy
  variables in your regression equation.  
\item[a.] Now test the sharp null
  of no treatment effect using randomization inference. Use two test
  statistics: Wilcoxon’s signed rank test (Rosenbaum 2002, pg. 32) and
  the difference in means. What are the two sided $p$-values under these
  two tests?  
\item[a.] Under the assumption of a constant, additive, treatment
  effect, use randomization inference to find a 95\% confidence
  interval of the treatment effect. Use the signed rank as your test
  statistic. See pages 44-46 in Rosenbaum (2002).
\item[d.] What can you conclude about the effectiveness of clientelistic
  appeals in Benin?
\item[e.] Bonus: Perform randomization inference with covariance adjustment. How does this effect your results? For a very good article on  covariance adjustment with randomization inference, see: 
\begin{quote}
Rosenbaum, Paul. 2002. “Covariance Adjustment in Randomized Experiments and Observational Studies.” \textit{Statistical Science} 17(3): 286-327. 
\end{quote}
\end{itemize}



\end{document}
% LocalWords:  texttt head.edu rnorm noindent Olken


\documentclass{article}
\usepackage{amsmath}
\usepackage{amsthm}
\usepackage{color}
\usepackage{setspace}
\usepackage{fullpage}
\usepackage[round]{natbib}
\usepackage[utf8]{inputenc}
\usepackage{amssymb} 

% Setup for fullpage use
\usepackage{fullpage}

% Uncomment some of the following if you use the features
%
% Running Headers and footers
%\usepackage{fancyhdr}
% Multipart figures
%\usepackage{subfigure}
% More symbols
%\usepackage{amsmath}
%\usepackage{amssymb}
%\usepackage{latexsym}
% Surround parts of graphics with box
\usepackage{boxedminipage}

% Package for including code in the document
\usepackage{listings}

% If you want to generate a toc for each chapter (use with book)
\usepackage{minitoc}

% This is now the recommended way for checking for PDFLaTeX:
\usepackage{ifpdf}

%\newif\ifpdf
%\ifx\pdfoutput\undefined
%\pdffalse % we are not running PDFLaTeX
%\else
%\pdfoutput=1 % we are running PDFLaTeX
%\pdftrue
%\fi
\usepackage{natbib} 
\usepackage{times} 
\usepackage{setspace}
\usepackage{subfigure}

\usepackage{hyperref} 

\newcommand\independent{\protect\mathpalette{\protect\independenT}{\perp}} 
\def\independenT#1#2{\mathrel{\rlap{$#1#2$}\mkern2mu{#1#2}}} 



\ifpdf 
\usepackage[pdftex]{graphicx} \else 
\usepackage{graphicx} \fi 

\title{PS C236A / Stat C239A \\ Problem Set 3 \\ Due: Oct. 15, 2012}
\date{}

\begin{document}

\maketitle
\vspace{-4em}
\section*{Instructions}
This assignment is due {\bf 4 pm Monday, Oct. 15.}  You may submit your
analytical work either electronically or in paper form.  Electronic
versions must be sent as a .pdf to
$<$\texttt{jahenderson[at]berkeley.edu}$>$. Paper copies should be
placed in my mailbox in 210 Barrows.  For the computing portion of the
assignment, you \underline{must} submit a fully executable version of
all .R code, along with any data used in the code (excepting that
provided through the course webpage) to the email above.  All files
for each assignment sent electronically should be included in one
omnibus email, with the subject line containing the course and
homework number, and your last name (e.g., PS239A/STAT236A: HW3 - Clinton).\\


\noindent You are encouraged to work together in groups to complete
the assignments. However, you must hand in your own individual
answers. Photocopies and other reproductions of someone else’s answers
are not acceptable. Please also list the names of everyone with
whom you have collaborated on this assignment.





\paragraph{Problem 1}

Consider an observational study, where $Z_i=1$ if unit $i$ is in the
treatment group and $Z_i=0$ if unit $i$ is in the control group. Let
$X$ be a vector of observed pretreatment covariates. Write $X_{Z=1}$
for the observed covariates of the units in the treatment
group. Similarly, let $X_{Z=0}$ be the observed covariates in the
control group.  Let $r_{1}$ be outcome under treatment and $r_{0}$ be
the outcome under control.  Assume the following:
$$r_0 \independent Z|X_{Z=1}$$
$$ P(Z=1|X_{Z=1})<1$$

\noindent Suppose you know the propensity score $e(X)=P(Z=1)$ for all
units $i$.  With these assumptions, can conditioning on the propensity
score estimate the ATT without bias? Prove it mathematically and
describe your logic in words.  What additional assumption would we
need in order to estimate the ATE without bias?  First show that
conditioning on the propensity score is equivalent to conditioning on
$X_{Z=1}$. Then show that conditioning on the propensity score can
produce unbiased ATT estimates under the assumptions above.




\paragraph{Problem 2}

%  \item[1)]

    Suppose there are 10,000 people in a study.
    Some of these people are assigned to treatment, and the rest are assigned to control.
    The exact mechanism for assigning units to treatment is unknown,
    but is known to depend only on the values of three dichotomous variables:
    the sex of the person, 
    whether or not the person exercises 30 minutes a day, 
    and whether or not the person watches TV for more than an hour a day.
    For each of the $2^3 = 8$ configurations of the dichotomous variables, 
    there is at least one treated person and at least one non-treated person.
  \begin{itemize}
  \item[a.] Let $T_i = 1$ if person $i$ is treated, and let $T_i = 0$
    if the person is not.  Let $T = (T_1,T_2,\ldots,T_{10000})$ denote
    the observed treatment assignment for all 10000 units.  The
    probability that person $i$ is assigned treatment is $p_i$, which
    is unknown.  Express the probability of observing the treatment
    assignment $T$ (i.e., $Pr\{T=t | m\}$) in terms of $T$ and $p$.
    What estimated values of $p_i$ maximize this probability, under
    the assumption that treatment assignment only depends on the
    dichotomous variables?  [Hint: The probability of the given
    treatment assignment is found in Rosenbaum and Rubin (1983).  Take the log of this probability and use calculus to
    maximize these probabilities.]
    \item[b.]  
      Suppose that the propensity score depends on all three of these dichotomous
      variables.  
      Will the estimated probabilities found in (a) converge to the true propensity score?
      What if the propensity score only depends on two of the
      original three dichotomous variables?
    \item[c.]
      Suppose the propensity score is known (and again, depends only on the values of 
      these three variables).  
      Write out unbiased estimates for the ATE and the ATT, 
      and prove that they are unbiased (assuming that the propensity score is fixed).
      How do these estimates change if the propensity score is not fixed? 
    \item[d.]
      Suppose the heaviest person of the 10,000 people weighs 500 pounds.
      Moreover suppose the propensity score is:
      $$
        P(T_i =1 | X) = 1/2 - 
        \text{weight}/1000
      $$
      Can both the ATT and the ATE be estimated without bias by
      conditioning responses on the propensity score?  
      For each quantity that can be estimated unbiasedly, 
      give a description or formula on how to compute the estimate (supposing the
      propensity score is known and fixed).
    \item[e)]
      Suppose it is known that the propensity score is linear 
      in weight, but the exact coefficients are unknown.
      Will regressing treatment assignment on 
      exercise and weight using OLS produce unbiased estimates of these coefficients?      
  \end{itemize}


\paragraph{Problem 3}

    Suppose that an award is given to anyone who scores above some threshold $c$ on a test.
    A student wants to evaluate the effect of obtaining the award on future income.
    The student uses an RD model to evaluate this effect:
    The student fits the model
    $$
      \text{Future Income}_i = \alpha + \beta(\text{Test Score}_i) + \epsilon_i
    $$
    separately to people scoring at most five points below the threshold  
    and to those scoring at most five points above the threshold.
    The student estimates the effect of winning the award on future income by 
    taking the difference between the 
    regression estimates
    when $\text{Test Score} = c$.
  \begin{itemize}
    \item[a.] Show that the variance of the regression estimates 
      are larger when further away from the mean test score.  
      How does this effect the estimate of the LATE?   
    \item[b.] Suppose that, before a person takes a test, they flip a coin.
      If the coin lands tails, the person loses 10 points on his test; otherwise, the person 
      does not lose any points on the exam. 
      The result of the coin flip is not known to the student. (Assume that the distribution of scores on the interval $\{c-5,c\}$ is the same as the distribution of scores on the interval $\{c+5,c+10\}$.)
      Can the LATE still be estimated?  
      If so, how?
      If not, what condition necessary to estimate the LATE is violated?
    \item[c.] Suppose that, in reality, future income is related to the test score and winning the award
    in the following way: 
    $$
       \text{Future Income}_i = \alpha + \beta(\text{Test Score Before Coin Flip}_i) 
         + \gamma(\text{Win award}_i) + \epsilon_i
    $$
    where both $\beta$ and $\gamma$ are greater than zero.
    Under the scenario in part (b), where a coin flip can cause the test taker to lose 10 points,
    show that the estimate of the LATE in part (a) gives a lower bound
    (asymptotically) of the true
    LATE.
\end{itemize}


\paragraph{Problem 4}

This problem is based on Sekhon's analysis of the voting
irregularities in the 2004 election in Florida. There was a lot of
speculation that ``the optical voting machines that [were] used in a
majority of Florida counties caused John Kerry to receive fewer votes
than `Direct Recording Electronic' (DRE) voting machines''. The paper
can be downloaded at:
\url{http://sekhon.berkeley.edu/papers/SekhonOpticalMatch.pdf}.  And
the data is available here:
\url{http://sekhon.berkeley.edu/causalinf/data/hw3data.RData}.

\begin{itemize}
%\item[a.] First, using Bush’s vote percentage in 2004 as an outcome,
%  run three linear models of the effect of using an electronic voting
 % machine. In each model, include different explanatory variables. How
 % does changing the model change the estimate of the effect of having
 % an electronic voting machine as well as the significance of this
 % estimate. Does the point estimate move? Should we be concerned, and
  %if so, why? [JAH: could skip this part]
\item[a.] Calculate a propensity score for assignment to
  treatment (defined as having an electronic voting machine), using no
  more than 3 of the following 5 covariates: \texttt{income},
  \texttt{votePer00.dem}, \texttt{regPer00.dem}, \texttt{black00}, 
  \texttt{lowEduc00}. Provide
  some justification for your pscore model. Make boxplots showing the
  distribution of the propensity score for both treated and control
  groups.% [JAH: could supply our own score]
\item[b.] Write your own univariate nearest-neighbor matching function
  (with replacement). In this function, include an option to pass in a
  caliper. Run this function twice, once without an enforced caliper
  and once with an enforced caliper. How small must your caliper be
  before it changes your results in a significant way? 
%[Hint: You can  effectively not enforce a caliper if you pass in a
%very large number   for your caliper]
\item[c.] Check balance, after matching with a caliper, and after
  matching without a caliper. What can you conclude?  What is the
  difference between running your function with a caliper and without
  a caliper? Which method should we prefer? Why?
\item[d.] Using your matched data set, calculate the ATE and ATT of
  having an electronic voting machine on Bush's presidential vote in 2004.  How do these two estimates
  differ?  Should we prefer one or the other?  Why?  
\item[e.] Estimate the ATE and ATT using inverse probability weighting on the basis of the same
  propensity score you estimated in part (a).  How does this estimate
  compare to the one you recovered in (d)?
\item[Bonus] Again, using the propensity score estimated in (a),
  estimate your estimands (ATE and ATT) using weighted OLS regression.
\end{itemize}

%\paragraph{Problem }
%Rd theory question?

%\paragraph{Problem }%
%mccrary smoothness quesion ... for RD 

\newpage



\end{document}
% LocalWords:  texttt head.edu rnorm noindent Olken

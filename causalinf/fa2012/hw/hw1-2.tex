
\documentclass{article}
\usepackage{amsmath}
\usepackage{amsthm}
\usepackage{color}
\usepackage{setspace}
\usepackage{fullpage}
\usepackage[round]{natbib}
\usepackage[utf8]{inputenc}
 
% Setup for fullpage use
\usepackage{fullpage}

% Uncomment some of the following if you use the features
%
% Running Headers and footers
%\usepackage{fancyhdr}
% Multipart figures
%\usepackage{subfigure}
% More symbols
%\usepackage{amsmath}
%\usepackage{amssymb}
%\usepackage{latexsym}
% Surround parts of graphics with box
\usepackage{boxedminipage}

% Package for including code in the document
\usepackage{listings}

% If you want to generate a toc for each chapter (use with book)
\usepackage{minitoc}

% This is now the recommended way for checking for PDFLaTeX:
\usepackage{ifpdf}

%\newif\ifpdf
%\ifx\pdfoutput\undefined
%\pdffalse % we are not running PDFLaTeX
%\else
%\pdfoutput=1 % we are running PDFLaTeX
%\pdftrue
%\fi
\usepackage{natbib} 
\usepackage{times} 
\usepackage{setspace}
\usepackage{subfigure}

\usepackage{hyperref} 

\newcommand\independent{\protect\mathpalette{\protect\independenT}{\perp}} 
\def\independenT#1#2{\mathrel{\rlap{$#1#2$}\mkern2mu{#1#2}}} 

\ifpdf 
\usepackage[pdftex]{graphicx} \else 
\usepackage{graphicx} \fi 

\title{PS C236A / Stat C239A \\ Problem Set 1 \\ Due: Sept. 21, 2012}
\date{}

\begin{document}

\maketitle
\vspace{-4em}
\section*{Instructions}
This assignment is due {\bf 4 pm Friday, Sept. 21.}  You may submit
your analytical work either electronically or in paper form.
Electronic versions must be sent as a .pdf to
$<$\texttt{jahenderson[at]berkeley.edu}$>$. Paper copies should be
placed in my mailbox in 210 Barrows.  For the computing portion of the
assignment, you \underline{must} submit a fully executable version of
all .R code, along with any data used in the code (excepting that
provided through the course webpage) to the email above.  All files
for each assignment sent electronically should be included in one
omnibus email, with the subject line containing the course and
homework number, and your last name (e.g., PS239A/STAT236A: HW1 - Obama).\\







\noindent You are encouraged to work together in groups to complete
the assignments. However, you must hand in your own individual
answers. Photocopies and other reproductions of someone else’s answers
are not acceptable. Please also list the names of everyone with
whom you have collaborated on this assignment.



\section*{Potential Outcomes}

\paragraph{Problem 1:}
Hooke's Law of elasticity for the restoring force of a spring out of
equilibrium is $F=-kx$, with $x$ being a measure of displacement, and
$k$ being a spring constant rate.  It is an approximation. Compare
Hooke's Law to a model of `force' in political science, where a
person's party identification, $PID_i$ (e.g., an ordinal scale of
seven points which ranges from strong Democrat to strong Republican),
influences the rate of $i$'s campaign giving to candidate $j$, denoted
as $d_{ij}$.  Frequently this is modeled as: $d_{ij}=\alpha+\gamma
PID_i+\beta(v_i-m_j)^2 + \epsilon_{ij}$, where $v_i$ and $m_j$ control
for the ideal policies $i$ and $j$ prefer.  In this model, $\alpha,
\beta$ and $\gamma$ are parameters that are estimated, $\epsilon_{ij}$
is a stochastic term, and $PID$ is measured through a random survey.
Assume $v_i$ and $m_j$ are fixed covariates measured through a survey
asking respondents to place themselves on an ideological scale.  In
real data of contributors, this model predicts a person's political
donations extremely well.  Is this sufficient for the donations
equations to provide potential outcomes for $PID$?  Why or why not?
What about Hooke's Law for $x$? What (if any) additional assumptions
are required in either case?

\paragraph{Problem 2:}

Imagine $n$ people who live on the same street are randomly assigned
to some treatment $T_i=\{0,1\}$. 
\begin{itemize}
\item[a.] How many potential outcomes in total are there in this
  experiment {\em without} making the SUTVA assumption?  

\item[b.] Now, assume there is
% symmetrical (i.e. full)
interference only if $\sum_{i=1}^n T_i \geq \frac{n}{2}$. When this
condition is met, how many potential outcomes does every $i$
unit have? 

\item[c.] Define {\em adjacent units} on this street to be each $i$'s
  nearest neighbors $\{i-1,i+1\}$, where location on the street
  defines $i$'s ordering. Assume interference for $i$ occurs {\em only
  } if $T_{i-1}=T_{i+1}=1$, that is both neighbors $i-1$ and $i+1$ are
  assigned treatment. Now how many potential outcomes are there for
  each $i$?
\end{itemize}


\paragraph{Problem 3:} Consider a field experiment that compares
treatments A and B. Suppose there are $N$ subjects, indexed by
$i=1,...,N$. Let $x_i$ be the response of subject $i$ to treatment
$A$; likewise, $y_i$ is the response to B. For each $i$, either $x_i$
or $y_i$ can be observed, but not both. Let $S$ be a random subset of
$\{1,...,N\}$, with $n$ elements; this group gets treatment A, so
$x_i$ is observed for $i$ in $S$. Let $T$ be a random subset of
$\{1,...,N\}$, with $m$ elements, disjoint from $S$. This group gets
treatment $B$, so $y_i$ is observed for $i$ in $T$.

We estimate population means $\bar x$ and $\bar y$ by the sample
means:
\begin{align*}
\bar X = \frac{1}{n} \sum_{i}^nx_i & &\bar Y = \frac{1}{m} \sum_{i}^my_i\end{align*}

Using simple sampling without replacement formulas:

\begin{align*}
\textrm{var}(\bar X) = \frac{N-n}{N-1}\frac{\sigma^2}{n}&
&\textrm{var}(\bar Y) = \frac{N-m}{N-1}\frac{\tau^2}{m}
\end{align*}
\begin{align*}
\textrm{cov}(\bar X, \bar Y)= -\frac{1}{N-1} \textrm{cov}(x,y)
\end{align*}


\begin{itemize}
\item[a.] What is the average treatment effect parameter? Write it
  using the above notation and also explain what it is in words. 
\item[b.] What is the variance of the average treatment effect (ATE), i.e. var$(\bar X - \bar Y)$, using the above notation? 
\item[c.] The usual two sample difference-in-means variance (without
  replacement)  found in sampling textbooks is:  
$$ \frac{N}{N-1} \left( \frac{\sigma^2}{n} + \frac{\tau^2}{m} \right)$$
What is the difference, if any, between the usual two sample
difference-in-means variance and the variance expression you derived
in part (b)?
\item[d.] The variance calculated using the ``usual'' formula can be biased, but only in one direction. What is the direction of the bias in the ``usual'' variance estimate? Prove it. Under what conditions will this bias be 0?
\end{itemize}




\section*{Linear Regression}
\paragraph{Problem 4:}
Suppose you are in a simplified world, and you wish to determine the
returns to education for a group of N workers you have data for.  In
this simplified version of the world, there are two factors that
influence a worker's income, level of education and intelligence.
Assume the correct model is:
\begin{equation} y_{i} = \alpha_{1} + \gamma_{1}*\text{education level}_{i} + \gamma_{2}*\text{intelligence}_{i} + \epsilon_{1i} \end{equation}
Where $y_{i}$ is individual $i$'s income.  However, you naively assume that the only factor that influences income is education level, and you run a regression using the following model:
\begin{equation} y_{i} = \alpha_{2} + \beta_{1}*\text{education level}_{i} + \epsilon_{2i}  \end{equation}
\begin{itemize}

\item[a.] Write down or describe the design matrix for the correct model of the world (model 1) as well as the naive model (model 2).
\item[b.] Show that $\frac{1}{N}\sum_{i = 1}^N y_{i} = \frac{1}{N}\sum_{i = 1}^N \hat{y}_{i}$
\item[c.] Which, if any, assumptions and conditions are necessary for part (b) to be true?
\item[d.] Assume that education level and intelligence are positively correlated.  By using the naive model instead of the true model, what happens to your estimate of $\beta_{1}$?  How would it relate to your estimate of $\gamma_{1}$ if you ran a regression using the true model?  Prove it.
\item[e.] Is this estimate of $\beta_{1}$ from (d) BLUE?  Why or why
  not?
\item[f.]  What is cov$(\hat{\beta}_1|X)$? 
\end{itemize}

% freedman page 66 question 7
%\paragraph{Problem 2:} True or False, and explain: as long as the
%design matrix has full rank, the computer can find the OLS estimator
%$\ha% t{\beta}$.  If so, what are the assumptions good for? Discuss briefly.




\paragraph{Problem 5:} 
Researchers run a randomized experiment to measure the effect of
school vouchers $T_t$ in the 8th grade on student test scores
$S_{t+2}$ by grade 10.
\begin{enumerate}
\item[a.] Researchers first estimate:
  $S_{t+2}=\alpha+\beta_1T_t+\beta_2 S_{t-1}+\epsilon$.  Assume there is successful
  randomization, no compliance problem, and the data are full
  rank. Is $\hat{\beta_1}$ unbiased?  Prove it.
\item[b.] Subsequently, researchers estimate:
  $S_{t+2}=\alpha+\beta_1T_t+\beta_2S_{t-1}+\beta_3S_{t+1}+\epsilon$.
  Again, assume successful randomization, full compliance, and
  full rank data. Is $\hat{\beta_1}$ unbiased?  Prove it.
\item[c.] \textit{Bonus:} In part (b), is $\hat{\beta_1}$ unbiased
  asymptotically?  Prove it.
\end{enumerate}


%\paragraph{Problem 5:} 
%\renewcommand{\theenumi}{\alph{enumi}} Suppose
%$Y_i=au_i+bv_i+\epsilon_i$ for $i=1,...,100$. The $\epsilon_i$ are
%independent $N(0,1)$. The $u$’s and $v$’s are fixed not random; these
%two data variables have mean 0 and variance 1: the correlation between
%them is $r$. If $r = \pm 1$, show that the design matrix has rank
%1. Otherwise, let $\hat{a}$ and $\hat{b}$ be the OLS estimator. Find
%the variance of $\hat{a}$, find the variance of $\hat{b}$, and find
%the variance of $\hat{a} - \hat{b}$. What happens if $r = 0.99$? What
%are the implications for collinearity for applied work? For instance,
%what sort of inferences about a and b are made easier or harder by
%collinearity?



\section*{Applications In R}

\paragraph{Problem 6:} 

Table 1 contains the potential outcomes from a hypothetical experiment
with 6 units.  Complete the following calculations using R.

\begin{table}[!h]
		\caption{Potential Outcomes}
	\begin{center}
		\begin{tabular}{ccc}
			Unit & $Y_T$ & $Y_C$ \\ \hline  
			1    & 2     & 1     \\ 
			2    & 6     & 2     \\ 
			3    & 33    & 13    \\ 
			4    & 17    & 14    \\ 
			5   &  2     &  10     \\ 
			 6  & 54    &  3 \\     
		\end{tabular}
		\label{}
	\end{center}
\end{table}


\begin{itemize}
\item[a.] What are the unit-level treatment effects? What is the ``true''
  average treatment effect? Is the average treatment effect a reasonable way
  of summarizing causal effects in this case? 
\item[b.]   What is the variance of the average treatment effect, using
  the formula you derived in part 3(b) from the above question? What is
  the variance using the ``usual'' formula written in 3(d) from the
  above question?
  \item[c.] Write a function that randomly assigns treatment to three out of
  the six units and then  produces the observed values of the dependent
  variable. The function should also calculate the estimated average
  treatment effect from the observed values, as well as its standard
  errors.
 % [Hint: You may want to look at the help file for the function \texttt{rbinom(n, size, prob)} with size = 1 and prob = 0.5.]
\item[d.] Calculate the estimated treatment effect for
  every possible combination of treatment assignment. Summarize this distribution of estimates using a plot.
  %[Hint: You may want to look at the help file for the function
  %\texttt{combn(x, m)}.]   
\item[e.] What is the ``true'' variance of the treatment effect estimate? Calculate this using your treatment effect estimates from part (d).
\end{itemize}

\subsection*{Olken Data}
For Problems 7 and 8, you will use R to calculate
%  descriptive statistics and 
treatment effect estimates from a dataset used in:
\begin{quote}
  Benjamin A. Olken. 2007. ``Monitoring Corruption: Evidence from a
  Field Experiment in Indonesia.'' \textit{Journal of Political
    Economy} 115: 300-249
\end{quote}
Note: You can download the data file on the class website at:
\paragraph{}
\url{http://sekhon.berkeley.edu/causalinf/data/hw1data.RData}

The data are contained in an object called \texttt{data}.
\paragraph{}
This objective of this experiment was to evaluate two interventions
thought to reduce corruption in road building projects in Indonesian
villages. The two treatments were audits by engineers and efforts to
encourage communities to monitor the projects
themselves. i.e. ``grassroots participation''.  While the actual
experimental design is somewhat involved, in this exercise we will
focus on the intervention designed to increase community
monitoring. The full paper can be found here:
\begin{quote}
  \url{http://econ-www.mit.edu/files/2913}
\end{quote}


Olken describes the intervention to be analyzed as follows:
\begin{quote}
  ...[T]he experiments sought to enhance participation at
  ``accountability meetings'', the village-level meetings in which
  project officials account for how they spent project
  funds. ...[H]undreds of invitations to these meetings were
  distributed throughout the village, to encourage direct
  participation in the monitoring process and to reduce elite
  dominance of the process. 
\end{quote}
Note that residents in treatment villages were notified about these meetings
\textit{before} construction began, but after the total budget
was decided. While the total budget was allocated before assignment to treatment, decisions about how the budget was to be spent was decided after the intervention. 

The main dependent variable is \texttt{pct.missing}, which is a
measure of the difference between what the villages claimed they spent
on road construction and an independent estimate of what the villages
actually spent. Treatment status is indicated by the dummy variable
\texttt{treat.invite}, which takes a value of 1 if the village
received the intervention and 0 if it did not. 

\begin{table*}[h]
  \caption{Variables \label{vars}}
  \centering
  \begin{tabular}{c|c}
    \hline \textbf{Variable} & \textbf{Definition}\\ \hline
    \texttt{pct.missing} & Percent expenditures missing\\
    \texttt{treat.invite} & Treatment assignment \\ 
    \texttt{head.edu} & Village head education \\
    \texttt{mosques} & Mosques per 1,000 \\
    \texttt{pct.poor} & Percent of households below the poverty line\\
    \texttt{total.budget} & Total budget (Rp. million)\\
    \texttt{share.total.unskilled} & Share of road construction expenses spent on
    unskilled labor\\
    \texttt{unskilled.transformed} & Transformed \texttt{share.total.unskilled}
   \end{tabular}
\end{table*}
Other variables in the dataset are listed in Table \ref{vars}. 


%\paragraph{Problem 7:}
%\begin{enumerate}
%\item Check whether the variables in the dataset have missing values, and
%report the number of missing values by variable. 

%\item  Report the minimum, maximum, mean, and standard deviation of the
%\textit{pre}-treatment covariates in the data set, separately for treatment and
%controls.  Are treatment and control units similar in terms of these
%characteristics? Be sure that you only include variables that were
%measured before the treatment was applied. 

%\item Bonus: Use a \texttt{for} loop or the \texttt{apply} function to
 % calculate these summary statistics.
%\end{enumerate}

\paragraph{Problem 7:} Complete the following computations, using your
own custom-written function(s). Where appropriate, your function(s)
should be flexible to input data up to $n\times k$ dimensions, and
produce well-formatted results.  (Note: You should not use any existing
statistical functions, i.e., \texttt{lm(),t.test()}; mathematic
functions are okay, i.e. \texttt{mean(),var()}.  When in doubt, if it
produces a p-value, don't use it.)






\begin{itemize}
\item[a.] Report the average difference in the outcome variable by treatment
assignment status (the ``treatment effect''). What is the standard error of this estimate? 

\item[b.] Now estimate the treatment effect using a regression model with no
covariates. Is this estimate different from the
difference-in-means estimate? Are the standard errors of the two estimates different?

\item[c.] Finally, estimate the treatment effect using a regression model, but
this time include all pre-treatment covariates as additional independent
variables.  What is your estimated treatment effect? What is the
standard error of this estimate? Is this estimate substantively
different from the difference-in-means estimate?

\item[d.] Is there a reason to prefer one of these methods of
  estimating treatment effects over the others?  What can you conclude
  about the effectiveness of this intervention?

  % \item[e.] In a couple of sentences, what can you conclude about
  %   the effectiveness of this intervention?

% \item[f.] Bonus: Write a complete function that inputs the above
%   data (flexible up to $k$ covariates), and produces the
%   calculations in (a)
%   - (c), stores them in a list, and outputs the list with printed
%   text
%   to provide a user with helpful information.



\end{itemize}

%\paragraph{Problem 8:}
%Write your own function in R to estimate a multivariate OLS model with
%beta coefficients {\em and} standard errors, i.e., without using
%\texttt{lm()}.  Then randomly generate your own outcome data $Y$ using
%\texttt{rnorm()}, as some function of at least three $X$ covariates,
%plus random error. (You may generate your own $X$ covariates or use
%any from Table \ref{vars} above.) Then use your function to estimate a
%linear model on the data, and compare these results to those produced
%by using \texttt{lm()}.\\


\paragraph{Problem 8:}

Making a selection on $X$ observables assumption,  the average
treatment effect for the treated (ATT) is defined as: 
$$
\tilde{\tau}|(T_i=1) = E\left\{ E(Y_i|X_i,T_i=1) -E(Y_i|X_i,T_i=0)\  |\ T_i=1\right\} 
$$

%τ ̄|(T =1) = E{E(Yi|Xi,Ti =1)−
%E(Yi|Xi,Ti = 0) | Ti = 1}

\noindent Estimate this quantity by OLS in R, for the following two
models from the Olken experimental data. Unlike the previous question,
feel free to use any function in R. In the models below:
$$
Y =  \texttt{pct.missing}
$$
$$
T = \texttt{treat.invite}
$$
$$
X_{1} = \texttt{unskilled.transformed}
$$
$$
X_{2} = \texttt{mosques}
$$


\begin{itemize}
\item[a.] $Y_i = \alpha + \tau
 T_i + \beta_1X_{1i}+\beta_2X_{2i}+\epsilon_i$
\item[b.] $Y_i = \alpha + \tau
 T_i + \beta_1X_{1i}+\beta_2X_{2i}+\gamma X_{1i}*T_i+\epsilon_i$
\end{itemize}

%\begin{itemize}
%\item[a.] $pct.missing_i = \alpha + \beta_1
 % treat.invite_i + \beta_2head.edu_i+\beta_3mosques_i+\epsilon_i$
%\item[b.]  $pct.missing_i = \alpha + \beta_1
%  treat.invite_i + \beta_2head.edu_i+\beta_3
%  treat.invite_i*head.edu_i+\epsilon_i$%
%\end{itemize}
 





\end{document}
% LocalWords:  texttt head.edu rnorm noindent Olken
